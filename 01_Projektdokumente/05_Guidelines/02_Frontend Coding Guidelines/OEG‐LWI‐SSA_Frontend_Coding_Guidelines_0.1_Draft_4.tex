\documentclass[11pt, a4paper, twoside]{article}

\usepackage[table]{xcolor}
\usepackage{geometry} %Set page layouts
\usepackage{pdfpages} %Include pdf files
\usepackage{fancyhdr} %Header and Footer
\usepackage{graphicx} %Inlcude graphics
\usepackage{lastpage} %Reference the number of pages in your LATEX
\usepackage[german]{babel}
\usepackage{fontspec}
\usepackage{float}
\usepackage{hyperref}
\usepackage{tocloft}
\usepackage{tabularx}
\usepackage{import}
\usepackage{lscape}
\usepackage{pdflscape}
\usepackage{verbatim}
% \usepackage{showframe} %For debugging

\hypersetup{
    colorlinks=true,
    linkcolor=black,
    filecolor=magenta,
    urlcolor=cyan,
}

\geometry{top=0.75cm, bottom=0.8cm, right=2.5cm, left=2.5cm, headheight=35pt, includeheadfoot, portrait}
\setmainfont{Calibri}

%Path relative to the .tex file containing the \includegraphics command
\graphicspath{{01_Grafiken/}}

\tocloftpagestyle{fancy}

%Define variables for document
\newcommand{\shortAuthor}{OEG-LWI-SSA}
\newcommand{\dstate}{Draft} %Draft, Progress, Closed
\newcommand{\dversion}{0.1}
\newcommand{\dname}{\shortAuthor\_Backend\_Coding\_Guidelines\_\dversion\_\dstate\_\pageref{LastPage}}

\begin{document}

    \section*{Java Coding Guidelines - Visual Board GmbH}
    \newpage
    %Define header & footer for twoside documents
    \fancyhead{} %Clear all header fields
    \fancyhead[L]{\includegraphics{header_img_abb2.png}}
    \fancyfoot{} %Clear all footer fields
    \fancyfoot[RO, LE]{\small \thepage \textbar \pageref{LastPage}}
    \fancyfoot[LO, RE]{\small \dname}
    \renewcommand{\footrulewidth}{0.4pt}
    \renewcommand{\headrulewidth}{0.4pt}
    \pagestyle{fancy}


    \section*{Dokumentenmanagement}

    \begin{tabularx}{\columnwidth}{l X}
        \textbf{Erstellungsdatum:} & 22.06.2020 \\
        \textbf{Autoren:} & Oliver Egloff (OEG) \\
        \textbf{Dateiname:} &\dname
    \end{tabularx}

    \section*{Änderungsverzeichnis}
    \begin{tabularx}{\columnwidth}{l l l X}
        \textbf{Version} & \textbf{Datum} & \textbf{Autor} & \textbf{Beschreibung}\\
        0.1 & 22.06.2020 & OEG & Coding Guidelines erstellt \\
    \end{tabularx}

    \renewcommand{\cftsecleader}{\cftdotfill{\cftdotsep}}
    \tableofcontents

    \clearpage

    \section{Code Style}
    In unseren Java Projekten verwenden wir den Code Styleguide von Google.
    Dieser kann für IntelliJ hier heruntergeladen werden https://github.com/google/styleguide/blob/gh-pages/intellij-java-google-style.xml
    Den Styleguide kann in IntelliJ wie folgt eingerichtet werden: https://www.jetbrains.com/help/idea/configuring-code-style.html
    \\

    \section{Logging}
    Abweichung zu Konstanten-Style → Klein- und Kurzschreibung
    \begin{verbatim}
    private final static Logger log = new Logger(....class);
    \end{verbatim}

    \section{Validierung}
    Für die Validierung sind folgende 2 Konstrukte zulässig: \\
    \textbf{boolean is}Valid / \textbf{has}Valid...(Email) \\
    \textbf{void validate}... / \textbf{check}...(Email) \textbf{throws} ...Exception

    \section{Variablenbennenung}
    Für die Validierung sind folgende 2 Konstrukte zulässig: \\
    \begin{tabularx}{\columnwidth}{l X}
        \textbf{Regel} & \textbf{Beispiel} \\
        Programmieren immer auf Englisch \\ auch bei Variablennamen & spielerName → nickname \\
        ganze Wörter & pId → playerId \\
        camelCase & playername → playerName \\
    \end{tabularx}
    \\
    \\
    Ausnahme: ``Exception e`` in einem Catch-Block

    

\end{document}

