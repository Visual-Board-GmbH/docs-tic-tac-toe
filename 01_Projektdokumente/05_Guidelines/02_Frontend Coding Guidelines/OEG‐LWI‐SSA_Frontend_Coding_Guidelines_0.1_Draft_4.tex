\documentclass[11pt, a4paper, twoside]{article}

\usepackage[table]{xcolor}
\usepackage{geometry} %Set page layouts
\usepackage{pdfpages} %Include pdf files
\usepackage{fancyhdr} %Header and Footer
\usepackage{graphicx} %Inlcude graphics
\usepackage{lastpage} %Reference the number of pages in your LATEX
\usepackage[german]{babel}
\usepackage{fontspec}
\usepackage{float}
\usepackage{hyperref}
\usepackage{tocloft}
\usepackage{tabularx}
\usepackage{import}
\usepackage{lscape}
\usepackage{pdflscape}
\usepackage{verbatim}
% \usepackage{showframe} %For debugging

\hypersetup{
    colorlinks=true,
    linkcolor=black,
    filecolor=magenta,
    urlcolor=cyan,
}

\geometry{top=0.75cm, bottom=0.8cm, right=2.5cm, left=2.5cm, headheight=35pt, includeheadfoot, portrait}
\setmainfont{Calibri}

%Path relative to the .tex file containing the \includegraphics command
\graphicspath{{01_Grafiken/}}

\tocloftpagestyle{fancy}

%Define variables for document
\newcommand{\shortAuthor}{OEG-LWI-SSA}
\newcommand{\dstate}{Draft} %Draft, Progress, Closed
\newcommand{\dversion}{0.2}
\newcommand{\dname}{\shortAuthor\_Frontend\_Coding\_Guidelines\_\dversion\_\dstate\_\pageref{LastPage}}

\begin{document}

    \section*{Frontend Coding Guidelines - Visual Board GmbH}
    \newpage
    %Define header & footer for twoside documents
    \fancyhead{} %Clear all header fields
    \fancyhead[L]{\includegraphics{header_img_abb2.png}}
    \fancyfoot{} %Clear all footer fields
    \fancyfoot[RO, LE]{\small \thepage \textbar \pageref{LastPage}}
    \fancyfoot[LO, RE]{\small \dname}
    \renewcommand{\footrulewidth}{0.4pt}
    \renewcommand{\headrulewidth}{0.4pt}
    \pagestyle{fancy}


    \section*{Dokumentenmanagement}

    \begin{tabularx}{\columnwidth}{l X}
        \textbf{Erstellungsdatum:} & 01.07.2020 \\
        \textbf{Autoren:} & Samuel Salomon (SSA) \\
        \textbf{Dateiname:} &\dname
    \end{tabularx}

    \section*{Änderungsverzeichnis}
    \begin{tabularx}{\columnwidth}{l l l X}
        \textbf{Version} & \textbf{Datum} & \textbf{Autor} & \textbf{Beschreibung}\\
        0.1 & 01.07.2020 & SSA & Dokument erstellt \\
        0.2 & 06.07.2020 & SSA & Coding Guidelines definiert \\
    \end{tabularx}

    \renewcommand{\cftsecleader}{\cftdotfill{\cftdotsep}}
    \tableofcontents

    \clearpage
    
    \section{Ordnerstruktur}
    	Alle Projektrelevanten Dateien befinden sich im im src Ordner des Projektes. Die verschiedenen Unterordner werden folgenden näher erklärt.
    	In dem Hauptverzeichnis befinden sich bereits zwei Dateien (App.vue, main.js).
    
   
   		\subsection{api}
   		Falls API Anfragen nötig sind werden die dafür erforderlichen Dateien in diesem Ordner abgelegt.
    	
    	\subsection{assets}
    	In diesem Ordner werden Daten wie Bilder, HTML Seiten oder ähnliches gespeichert die Später in die js Dateien importiert werden.
    	
    	\subsection{components}
    	Beinhlatet alle Komponenten des Projektes die später in den Hauptansichten (views) verwendet werden. Komponenten sollten in weitere Unterordner strukturiert werden.
    	
    	\subsection{mixins}
    	Die Mixins sind Teile des Javascript-Codes, die in verschiedenen Komponenten wiederverwendet werden. Demnach befinden sich in diesem Ordner Methoden die in jeder Komponente aus Vue.js wiederverwendet werden können.
    	
    	\subsection{router}
    	Routen werden verwendet um Komponenten auf spezifische Pfade der Web Applikation zu mappen. JS Dateien mit Routen definiert werden in diesem Ordner abgelegt.
    	
    	\subsection{views}
    	Geroutete Komponenten werden als Views definiert. Um das Projekt übersichtlicher zu gestalten werden diese getrennt von den anderen Komponenten in diesem Verzeichnis abgelegt.
    	
    
    	
    	
    	
    
    \section{Dateibenennung}
    Da Vue.js Komponentenbasiert ist sollen Komponenten wie folgt benant werde
   	\\
   	\\
    	 \begin{tabularx}{\columnwidth}{l X}
    		\textbf{Regel} & \textbf{Beispiel} \\
    		Immer in Englisch & brettobjekt → boardobject \\
    		camelCase & boardobject → boardObject \\
    	\end{tabularx}
    \\
    \\
    
    Generell sollte die Namen der Komponenten so gewählt werden, dass sie die Funktion dieser beschreiben.
    	

    \section{Code Style}
   	Vue JS besitzt eigene Code Style Empfehlungen auf die hier verwiesen werden. \href{https://vuejs.org/v2/style-guide/}{Vue.JS Style Guide}

    \section{Validierung}
    Für die Validierung sind folgende 2 Konstrukte zulässig: \\
    \textbf{boolean is}Valid / \textbf{has}Valid...(Email) \\
    \textbf{void validate}... / \textbf{check}...(Email)

    \section{Variablenbennenung}
    Für die Validierung sind folgende 2 Konstrukte zulässig: \\
    \begin{tabularx}{\columnwidth}{l X}
        \textbf{Regel} & \textbf{Beispiel} \\
        Programmieren immer in Englisch \\ auch bei Variablennamen & spielerName →playername \\
        ganze Wörter & pId → playerId \\
        camelCase & playername → playerName \\
    \end{tabularx}
    \\
    \\
    Ausnahme: ``Exception e`` in einem Catch-Block

    

\end{document}

