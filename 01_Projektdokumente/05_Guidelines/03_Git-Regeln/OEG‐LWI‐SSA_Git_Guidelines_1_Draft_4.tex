\documentclass[11pt, a4paper, twoside]{article}

\usepackage[table]{xcolor}
\usepackage{geometry} %Set page layouts
\usepackage{pdfpages} %Include pdf files
\usepackage{fancyhdr} %Header and Footer
\usepackage{graphicx} %Inlcude graphics
\usepackage{lastpage} %Reference the number of pages in your LATEX
\usepackage[german]{babel}
\usepackage{fontspec}
\usepackage{float}
\usepackage{hyperref}
\usepackage{tocloft}
\usepackage{tabularx}
\usepackage{import}
\usepackage{lscape}
\usepackage{pdflscape}
\usepackage{verbatim}
% \usepackage{showframe} %For debugging

\hypersetup{
    colorlinks=true,
    linkcolor=black,
    filecolor=magenta,
    urlcolor=cyan,
}

\geometry{top=0.75cm, bottom=0.8cm, right=2.5cm, left=2.5cm, headheight=35pt, includeheadfoot, portrait}
\setmainfont{Calibri}

%Path relative to the .tex file containing the \includegraphics command
\graphicspath{{01_Grafiken/}}

\tocloftpagestyle{fancy}

%Define variables for document
\newcommand{\shortAuthor}{OEG-LWI-SSA}
\newcommand{\dstate}{Draft} %Draft, Progress, Closed
\newcommand{\dversion}{0.1}
\newcommand{\dname}{\shortAuthor\_Backend\_Coding\_Guidelines\_\dversion\_\dstate\_\pageref{LastPage}}

\begin{document}

    \section*{Git Guidelines - Visual Board GmbH}
    \newpage
    %Define header & footer for twoside documents
    \fancyhead{} %Clear all header fields
    \fancyhead[L]{\includegraphics{header_img_abb2.png}}
    \fancyfoot{} %Clear all footer fields
    \fancyfoot[RO, LE]{\small \thepage \textbar \pageref{LastPage}}
    \fancyfoot[LO, RE]{\small \dname}
    \renewcommand{\footrulewidth}{0.4pt}
    \renewcommand{\headrulewidth}{0.4pt}
    \pagestyle{fancy}


    \section*{Dokumentenmanagement}

    \begin{tabularx}{\columnwidth}{l X}
        \textbf{Erstellungsdatum:} & 08.07.2020 \\
        \textbf{Autoren:} & Leonardo Wiedemeier (LWI) \\
        \textbf{Dateiname:} &\dname
    \end{tabularx}

    \section*{Änderungsverzeichnis}
    \begin{tabularx}{\columnwidth}{l l l X}
        \textbf{Version} & \textbf{Datum} & \textbf{Autor} & \textbf{Beschreibung}\\
        0.1 & 08.07.2020 & LWI & Git Guidelines erstellt \\
    \end{tabularx}

    \renewcommand{\cftsecleader}{\cftdotfill{\cftdotsep}}
    \tableofcontents

    \newpage

    \section{Master Branch}
    Der Master Branch ist ``protected'' und ``default''.
    \newline
    Default bedeuted, er ist der standard Branch bei jedem neuen merge request auf welchen gemerged wird.
    Auf Protected Branches kann nur gemerged werden. Direktes pushen ist auf diese Branches nicht möglich.
    
    \section{Naming}
    Branches werden nach folgendem Schema bennant:
    \begin{itemize}
    	\item Issues \textbf{TTT-<issueNumber>\_<issueName>}
    	\item Teilimplementierungen für einen Issue \textbf{feature/<was-wird-implementiert-oder-angepasst}
    	\item fixes/hotfixes \textbf{fix/<was-wird-gefixed}
    \end{itemize}

    \section{Vorgehen}
    \begin{enumerate}
    	\item Es wird vom Master ausgehend der Issue Branch erstellt.
    	\item Für ``jede'' Teilimplementation wird ein Feature Branch erstellt.
    	\begin{itemize}
    		\item es werden unnötig grosse Codereviews vermieden
    		\item der jeweilige Code Teil kann übersichtlich von anderen Mitarbeiter kontrolliert und verstanden/kritisiert werden
    	\end{itemize}
    	\item Nach dem Codereview von mind. 1 Mitarbeiter, wird der Feature Branch in den Issue Branch gemerged.
    	\item Sobald der Issue abgeschlossen wurde, kann dieser vom Verantwortlich in den Master Branch gemerged werden.
    \end{enumerate}
    	
    	
    	
\end{document}

