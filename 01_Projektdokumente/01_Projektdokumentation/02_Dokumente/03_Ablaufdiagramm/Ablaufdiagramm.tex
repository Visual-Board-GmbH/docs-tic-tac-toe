Im folgenden Ablaufdiagramm wird die Hauptlogik des TicTacToe Spiels als vereinfachten Codeablauf dargestellt. So wurde er geplant um die Struktur zu definieren und wie das Projekt umgesetzt werden soll.

\begin{figure}[!hbt]
	\centering
	\includegraphics[width=0.8\columnwidth]{02_Dokumente/03_Ablaufdiagramm/01_Grafiken/flussdiagramm_main.png}
	\caption{Flussdiagramm Main}
	\label{fig:flussdiagramm_main}
\end{figure}

\newpage

\begin{figure}[!hbt]
	\centering
	\includegraphics[width=0.8\columnwidth]{02_Dokumente/03_Ablaufdiagramm/01_Grafiken/flussdiagramm_login.png}
	\caption{Flussdiagramm Login}
	\label{fig:flussdiagramm_login}
\end{figure}

\newpage

\begin{figure}[!hbt]
	\centering
	\includegraphics[width=0.6\columnwidth]{02_Dokumente/03_Ablaufdiagramm/01_Grafiken/flussdiagramm_spiel-erstellen.png}
	\caption{Flussdiagramm Spiel erstellen}
	\label{fig:flussdiagramm_spiel-erstellen}
\end{figure}

\newpage

\begin{figure}[!hbt]
	\centering
	\includegraphics[width=0.3\columnwidth]{02_Dokumente/03_Ablaufdiagramm/01_Grafiken/flussdiagramm_eigenem-spiel-beitreten.png}
	\caption{Flussdiagramm Eigenem Spiel beitreten}
	\label{fig:flussdiagramm_eigenem-spiel-beitreten}
\end{figure}

\newpage

\begin{figure}[!hbt]
	\centering
	\includegraphics[width=0.9\columnwidth]{02_Dokumente/03_Ablaufdiagramm/01_Grafiken/flussdiagramm_spiellogik.png}
	\caption{Flussdiagramm Spiellogik}
	\label{fig:flussdiagramm_spiellogik}
\end{figure}

\newpage

\begin{figure}[!hbt]
	\centering
	\includegraphics[width=0.6\columnwidth]{02_Dokumente/03_Ablaufdiagramm/01_Grafiken/flussdiagramm_spiel-endet.png}
	\caption{Flussdiagramm Spiel endet}
	\label{fig:flussdiagramm_spiel-endet}
\end{figure}
