\documentclass[11pt, a4paper, twoside]{article}

\usepackage[table]{xcolor}
\usepackage{geometry} %Set page layouts
\usepackage{pdfpages} %Include pdf files
\usepackage{fancyhdr} %Header and Footer
\usepackage{graphicx} %Inlcude graphics
\usepackage{lastpage} %Reference the number of pages in your LATEX
\usepackage[german]{babel}
\usepackage{fontspec}
\usepackage{float}
\usepackage{hyperref}
\usepackage{tocloft}
\usepackage{tabularx}
\usepackage{import}
\usepackage{lscape}
\usepackage{pdflscape}
\usepackage{multirow}
%\usepackage{showframe} %For debugging

\hypersetup{
	colorlinks=true,
	linkcolor=black,
	filecolor=magenta,
	urlcolor=cyan,
}

\geometry{top=0.75cm, bottom=0.8cm, right=2.5cm, left=2.5cm, headheight=35pt, includeheadfoot, portrait}
%\setmainfont{Calibri}

%Path relative to the .tex file containing the \includegraphics command
\graphicspath{{01_Grafiken/}}

\tocloftpagestyle{fancy}

%Define variables for document
\newcommand{\shortAuthor}{OEG-LWI-SSA}
\newcommand{\dstate}{Draft} %Draft, Progress, Closed
\newcommand{\dversion}{0.1}
\newcommand{\dname}{\shortAuthor\_20200428\_\dversion\_\dstate\_\pageref{LastPage}}

\begin{document}


%Define header & footer for twoside documents
\fancyhead{} %Clear all header fields
\fancyhead[L]{\includegraphics{header_img_abb2.png}}
\fancyfoot{} %Clear all footer fields
\fancyfoot[RO, LE]{\small \thepage \textbar \pageref{LastPage}}
\fancyfoot[LO, RE]{\small \dname}
\renewcommand{\footrulewidth}{0.4pt}
\renewcommand{\headrulewidth}{0.4pt}
\pagestyle{fancy}

	
	\section*{Dokumentenmanagement}
	
	\begin{tabularx}{\columnwidth}{l X}
		\textbf{Erstellungsdatum:} & 28.04.2020 \\
		\textbf{Autotren:} & Samuel Salomon (SSA), Oliver Egloff (OEG), Leonardo Wiedemeier (LWI) \\
		\textbf{Dateiname:} &\dname
	\end{tabularx}
	
	
	\section*{Änderungsverzeichnis}
	\begin{tabularx}{\columnwidth}{l l l X}
		\textbf{Version} & \textbf{Datum} & \textbf{Autor} & \textbf{Beschreibung}\\
		0.1 & 28.04.2020 & SSA & Dokumentation und Gliederung erstellt \\
		0.2 & 04.05.2020 & LWI & Einfügen der Risikoanalyse \\
		0.3 & 06.05.2020 & LWI & Einfügen der Situationsanalyse \\
		0.4 & 07.05.2020 & LWI & Einfügen der Rollenbeschreibung + kleine Design Änderungen \\
		0.5 & 08.05.2020 & LWI & Einfügen der Ressourcenplanung \\
		0.6 & 11.05.2020 & LWI & Änderung R3/M3 in der Risikoanalyse \\
		0.7 & 12.05.2020 & LWI & Änderung in der Risikoanalyse + Grafik \\
		0.8 & 19.05.2020 & LWI & Risikomatrix mit Texten erweitern \\
		0.9 & 06.07.2020 & LWI & Lösungssuche und Lösungswahl vorbereitet \\
		0.10 & 08.07.2020 & SSA, OEG, LWI & Mind Map für die Lösungssuche erstellt\\
	\end{tabularx}
	
	\renewcommand{\cftsecleader}{\cftdotfill{\cftdotsep}}
	\tableofcontents
	
	\clearpage
	
	\section{Managment Summary}
	\section{Aufgabenstellung}
	\section{Projektmanagement}
    \subsection{Projektorganisation}
    \begin{figure}[!hbt]
    	\centering
    	\includegraphics[width=\columnwidth]{Projektorganisation.png}
    	\caption{Projektorganisation}
    	\label{fig:projektorganisation}
    \end{figure}
	
	\newpage
	
	\subsubsection{Rollenbeschreibung}
	\begin{table}[h]
		\centering
		\rowcolors{1}{gray!30}{gray!10}
		\begin{tabular}{p{0.3\textwidth}|p{0.64\textwidth}}
			\textbf{Kürzel} & \textbf{Beschreibung} \\ \hline
			Auftraggeber & Er ist der Kunde und liefert das Lastenheft. \\ \hline
			
			Projektsteuerungs-Gremium & Trifft wichtige Entscheidungen und ist bei Problemen umgehend zu informieren. \\ \hline
			
			Projektportfolio-Controller & Hat alle laufenden Projekte im Überblick und ist Ansprechpartner für den Projektleiter und die Projektträger. \\ \hline
			
			Projektleiter & Die führende Kraft im Projektteam. Ist Ansprechpartner für die Involvierten und Projektträger. \\ \hline
			
			Projektmitarbeiter & Alle Betroffenen und Involvierten Personen/Organisation, die am Projekt mitarbeiten. \\ \hline
			
			Frontend Entwicklung & Hauptverantwortlicher des Frontend. \\ \hline
			
			Backend Entwicklung & Hauptverantwortlicher des Backend. \\ \hline
			
			Entwicklungsleitung & Hat Übersicht des Zusammenspiels von Front- und Backend. \\ \hline
			
			Software \& System Architekt  &  Übersicht der eingesetzten Tools und des Systemaufbaus. \\ \hline
			
			Datenbank Manager \& Architekt  & Hauptverantwortlicher der Datenbank. \\ \hline
			
			Dokumentenverantwortlicher & Ist verantwortlich für den Aufbau und die Struktur der Dokumente. \\ \hline
			
			Requirements \& Business Engineering  & Hat die Übersicht über alle Anforderungen und ihrer Einhaltung \\ \hline
			
			Quality \& Testing  & Verantwortlich für die Qualität des Codes und das er getestet ist. \\
		\end{tabular}
		\caption{Rollenbeschreibung}
		\label{fig:rollenbeschreibung}
	\end{table}

	\newpage

	\subsubsection{Funktionsdiagramm}
	\begin{table}[h]
		\centering
		\rowcolors{1}{gray!30}{gray!10}
		\begin{tabular}{p{0.1\textwidth}|p{0.34\textwidth}|p{0.1\textwidth}|p{0.1\textwidth}|p{0.1\textwidth}|p{0.1\textwidth}}
			\textbf{Nr.} & \textbf{Leistung} & \textbf{ABBTS} & \textbf{SSA} & \textbf{OEG} & \textbf{LWI} \\ \hline
			L1 & Projektanalyse & & & &  \\ \hline
			AP1.1 & Situationsanalyse & I & M & M & D  \\ \hline
			AP1.2 & Anforderungskatalog  & I & D & M & M  \\ \hline
			AP1.3 & Ziele definieren & I & D & M & M \\ \hline
			AP1.4 & Risikoanalyse & I & M & M & D  \\ \hline
			L2 & Projektplanung  & & & &  \\ \hline
			AP2.1 & Funktionendiagramm  & I & M & M & D  \\ \hline
			AP2.2 & Rollenbeschreibung & I & M & M & D  \\ \hline
			AP2.3 & Kommunikationsplan  & I & M & D & M  \\ \hline
			AP2.4 & Projektorganisation & I & M & D & M  \\ \hline
			AP2.5 & Projektrollen definieren & I & M & M & D \\ \hline
			AP2.6 & Projektstrukturplan & I & D & D & D  \\ \hline
			AP2.7 & Projektzeitplan & I & D & M & D \\ \hline
			AP2.8 & Kostenplanung & I & M & D & M  \\ \hline
			AP2.9 & Ressourcenplanung & I & D & D & D \\ \hline
			L3 & Lösungsevaluation & & & &  \\ \hline
			AP3.1 & Lösungssuche & I & D & D & D  \\ \hline
			AP3.2 & Lösungsevaluation & I & D & D & D  \\ \hline
			AP3.3 & Lösungsentscheid & I & D & D & D \\ \hline
			L4 & Pflichtenheft & & & &  \\ \hline
			AP4.1 & Chancen \& Risiken analysieren & I & M & M & D \\ \hline
			AP4.2 & Massnahmen definieren & I & M & M & D \\ \hline
			AP4.3 & Technische Umsetzung planen & I & D & D & D  \\ \hline
			AP4.4 & Vorbereitung Abnahme Pflichtenheft & I & M & D & D \\ \hline
		\end{tabular}
		\caption{Funktionsdiagramm}
		\label{fig:funktionsdiagramm}
	\end{table}

	\begin{table}[h]
		\centering
		\begin{tabular}{|p{0.46\textwidth} p{0.46\textwidth}|}
			\hline Legende & \\
			E = Entscheiden & K = Koordination, Veranlassung  \\
			I = Information & D = Durchführung, Verantwortung  \\
			M = Mitarbeiter, Beratung &  \\ \hline
		\end{tabular}
	\end{table}

	\subsection{Ressourcenplanung}
	\begin{table}[h]
		\centering
		\rowcolors{1}{gray!30}{gray!10}
		\begin{tabular}{p{0.15\textwidth}|p{0.1\textwidth}|p{0.1\textwidth}|p{0.15\textwidth}|p{0.15\textwidth}|p{0.15\textwidth}}
			\textbf{Ressourcenname} & \textbf{Typ} & \textbf{Initials} &\textbf{Rolle}&\textbf{Stundensatz}&\textbf{Kosten pro Stunde} \\ \hline
			
			Oliver Egloff & Person / Gruppe & OEG & Projektmanagement & CHF 100.00/h & CHF 0.00 \\\hline
			
			Leonardo Wiedemeier & Person / Gruppe & LWI & Projectworker & CHF 100.00/h & CHF 0.00 \\\hline
			
			Samuel Salomon & Person / Gruppe & SSA & Projektworker & CHF 100.00/h & CHF 0.00 \\\hline
			
			Stakeholder & Person / Gruppe & S & Gremium & CHF 0.00 & CHF 0.00 \\
		\end{tabular}
		\caption{Stundensatz}
		\label{fig:stundensatz}
	\end{table}

	\begin{table}[h]
		\centering
		\rowcolors{1}{gray!30}{gray!10}
		\begin{tabular}{p{0.23\textwidth}|p{0.25\textwidth}|p{0.25\textwidth}|p{0.15\textwidth}}
			\textbf{Name} & \textbf{Aktueller Aufwand} & \textbf{Geplanter Aufwand} & \textbf{Differenz} \\ \hline
			Oliver Egloff & 5 hrs  & 100 hrs & 0 hrs \\
			Leonardo Wiedemeier & 5 hrs  & 100 hrs & 0 hrs \\
			Samuel Salomon & 9.5 hrs  & 100 hrs & 0 hrs \\
			Stakeholder & 0 hrs & 0 hrs & 0 hrs \\
		\end{tabular}
	
		\caption{Aufwand}
		\label{fig:aufwand}
	\end{table}

	\newpage
	
	\section{Situationsanalyse}
	\subsection{Vorgänger Projekt}
	Die Visual Board GmbH hat bereits mit der ABBTS (Rolf Hirschi) zusammengearbeitet und einen Prototypen erstellt um Bilder auf einer RGB-LED-Matrix anzuzeigen.
	Die RGB-LED-Matrix wurde funktionsfähig gestellt und es sollte ein Programm geschrieben werden um Bilder wie gewünscht auf der Matrix anzuzeigen. Darunter ist das korrekte transformieren, aufbereiten und weitergeben an die Matrix zu verstehen.
	
	Der Prototyp wurde von der Visual Board GmbH mit einer REST-Api in Java umgesetzt welche auf dem Raspberry Pi der Hardware, oder einem externen Windows Computer läuft.
	So wurde die RGB-LED-Matrix, dem Netzwerk in welchem sie sich befindet hat, zur Verfügung gestellt. Und es ist möglich mit den korrekten "requests" Bilder direkt oder mit dem Webinterface an die Matrix zu schicken.
		
	Der Visual Board GmbH wurde die komplette Hardware durch die ABBTS gestellt wurde.
	\newline
	Teil der Projektorganisiton waren:
	\begin{itemize}
		\item ABBTS (Rolf Hirschi) als Auftraggeber
		\item Visual Board GmbH als Arbeitnehmer. Samuel Salomon hat die Position als Projektleiter eingenommen.
		\item Herr Künzli und Herr Jenzer jeweils als Mitglied des PSG (Projektsteuerungs-Gremium)
		\item Herr Künzli hatte zudem die Position des PPC  (Projektportfolio-Controller
	\end{itemize}

	\subsection{Weiteres Vorgehen}
	Die ABBTS hat sich entschieden auf den Prototypen der Visual Board GmbH aufzubauen und die Spiellogik für ein Tic-Tac-Toe Spiel umzusetzen.
	Es soll die Möglichkeit bestehen das mehrere Spieler, auf verschiedenen Endgeräten, gegeneinander spielen, während die RGB-LED-Matrix das aktuelle Spielfeld ausgibt. Hier für werden wir das komplette Vorgänger Projekt verwenden, da die Matrix bereits einsatzbereit ist. Es werden allenfalls Schnittstellen angepasst oder neue erstellt.
	Zusätzlich zur Implementierung der Spiellogik, soll diese auch auf einen Webserver laufen und so erreichbar gemacht werden.
	
	Die Visual Board GmbH wird daher im Rahmen des Projektes die Spiellogic mit Benutzeroberfläche implementieren und den nötigen Webserver einrichten. Für diese Arbeiten stehen 300 Arbeitsstunden der kompletten Projektabwicklung zur Verfügung.
		
	\section{Zielsetzung}
	\section{Lösungssuche}
	\begin{figure}[!hbt]
		\centering
		\includegraphics[width=\columnwidth]{01_Grafiken/loesungs_mind_map.png}
		\caption{Lösungs Mind Map}
		\label{fig:mind_map}
	\end{figure}
	
	\subsection{Vorstudie zur Erhebung der Programmiersprachen}
	\subsubsection{Backend}
	\paragraph{PHP}
	Eine weit verbreitet Open source Skriptsprache welche zur Laufzeit kompiliert wird. Sie zeichnet sich durch ihre Einfachheit, Geschwindigkeit und Stabilität aus. Es gibt eine umfangreiche Anzahl von Frameworks wie auch eine grosse Community, welche durch Open source ein gewisses Mitspracherecht an der Zukunft der Sprache hat. So können simple Kommandozeilen Scripts wie auch umfangreiche Webapplikationen erstellt werden.  
	\paragraph{Java}
	Java ist eine objektorientierte Programmiersprache, mit welcher sich plattformunabhängige Anwendungen entwickeln lassen können. Die Sprache hat sich gut durchgesetzt und ist weit verbreitet. Die Haupteigenschaften der Programmiersprache sind Sicherheit, Robustheit und Einfachheit. Java hat eine Vielzahl an Frameworks und eine grosse Community.
	\\ \\
	Beide Sprache bieten eine grosse Community und ein umfangreiches Toolset mit dem die Anforderungen an das System erfüllt werden können. Mit Hilfe einer Nutzwertanalyse wird nun entschieden mit welchen Varianten die Sprachen in der Lösungswahl berücksichtigt werden.
	
	\subsubsection{Frontend}
	\paragraph{Vue.js}
	Ist ein clientseitiges JavaScript-Webframework zum Erstellen von Single-Page-Webanwendungen nach dem MVVM-Muster, es kann allerdings auch in Multipage Webseiten für einzelne Abschnitte verwendet werden. Serverseitiges Rendern ist auch unterstützt. \\
	Vue.js ist ein "relativ" junges Framework, hat aber trotzdem eine wachsende Community und wird immer mehr verwendet. Wie zum Beispiel von GitLab.
	\paragraph{React}
	Gibt es bereits etwas länger als Vue.js und bietet ähnliche Funktionalitäten als Frontend Framework. Es wurde damals von Facebook ins leben gerufen und auch noch heute verwendet. Daher geniesst React eine grosse Community und ist weit verbreitet.
	\\ \\
	Beide Sprachen bieten ähnliche Funktionalitäten und Syntax. Vue.js gilt als flexiblere Sprache und daher für Neulinge besser zum einsteigen. React.js geniesst eine grosse Community und Facebook als Mitentwickler hinter der Sprache (ist aber wie Vue.js OpenSource). Auch hier soll durch eine Nutzwertanalyse 
	
	\subsection{Black Box}
	
	\subsubsection{Frontend}
	
	\begin{figure}[!hbt]
		\centering
		\includegraphics[width=\columnwidth]{01_Grafiken/black-box_Frotend.png}
		\caption{Black Box Frontend}
		\label{fig:bb_frontend}
	\end{figure}
	
	\subsubsection{Backend}

	\begin{figure}[!hbt]
		\centering
		\includegraphics[width=\columnwidth]{01_Grafiken/black-box_Backend.png}
		\caption{Black Box Backend}
		\label{fig:bb_backend}
	\end{figure}

	\newpage

	\subsection{Variante 1}
	
	\begin{figure}[!hbt]
		\centering
		\includegraphics[width=\columnwidth]{01_Grafiken/architektur.png}
		\caption{Architektur}
		\label{fig:architektur}
	\end{figure}

	Da es sich um ein Nachfolgerprojekt der LED-RGB-Matrix handelt und es die Matrix als unabhängiger Service programmiert wurde ist dieser Teil bereits "abgeschlossen". Es handelt sich um eine Rest-API welche vom jetzigen Projekt einfach angesprochen werden kann.
	Der Tic-Tac-Toe Service (Spiellogik + Datenbank) wird das Backend bilden. Es wird auf einer Seite via JSON mit der LED-RGB-Matrix API kommunizieren und auf der anderen via MQTT mit den jeweiligen Clients. Hier muss noch entschieden werden mit welcher Technologie, Java oder PHP, das Backend umgesetzt wird. Bei der Datenbank wird es sich um MYSQL handel welche mit der jeweiligen Sprache dann angesprochen wird.
	
	Im Frontend steht die Entscheidung der Programmiersprache auch noch offen, Vue.js oder React.js.
	Das Frontend wird fähig sein via MQTT mit dem Backend zu kommunizieren und so dynamisch die View für den Benutzer anpassen. Weiteres ist die View responsive für Mobile wie auch Computer.

	\subsection{Morphologische Analyse (//TODO Mit Herrn Hirschi absprechen öb nötig)}
	
	\newpage
		
	\section{Lösungswahl}
	Die Lösungswahl wird aufgrund der Zielerreichung, Nutzwertanalyse und den weiteren Risiken getroffen.
	\subsection{Zielerreichung}
	In der Zielerreichung wird analysiert, ob die Lösungsvarianten die geforderten Systemziele erfüllen können. So werden ungeeignete Lösungen bereits in einer frühen Phase aus der Studie herausgefiltert.
	\begin{table}[h]
		\centering
		\rowcolors{1}{gray!30}{gray!10}
		\begin{tabular}{p{0.25\textwidth}|p{0.2\textwidth}|p{0.22\textwidth}|p{0.22\textwidth}}
			\textbf{Systemziel-Nr.} & \textbf{Gewicht (M,S,K)} & \textbf{V1} & \textbf{V2} \\ \hline
			& & &
		\end{tabular}
		\caption{Zielerreichung}
		\label{fig:zielerreichung}
	\end{table}
	
	\newpage

	\begin{landscape}
	\subsection{Nutzwertanalyse}
	Die Gewichtung wird in Form einer Zahl zwischen 1-10 definiert, wobei die 1 die tiefste und 10 die höchste Gewichtung ist.
	\begin{table}[h]
		\centering
		%\rowcolors{1}{gray!30}{gray!10}
		\begin{tabular}{p{0.4\textwidth}|p{0.25\textwidth}|p{0.2\textwidth} | p{0.2\textwidth} | p{0.2\textwidth} | p{0.2\textwidth}}
			\multirow{2}{*}{\textbf{Kriterien}} & \multirow{2}{*}{\textbf{Gewichtung}} & \multicolumn{2}{c}{\textbf{Variante 1: PHP}} & \multicolumn{2}{c}{\textbf{Variante 2: Java}} \\
			& & \textbf{Punkte} & \textbf{Bewertung} & \textbf{Punkte} & \textbf{Bewertung}  \\ \hline
			Risiken & 123 & & & &  \\ \hline
			Nachhaltigkeit & 123 & & & &  \\ \hline
			& X & & & & \\ \hline
			\textbf{Nutzwert} & & & \cellcolor{orange} xxx & & \cellcolor{green} \\ \hline
		\end{tabular}
		\caption{Nutzwerkanalyse}
		\label{fig:nutzwerkanalyse}
	\end{table}
	\end{landscape}

	\newpage
	
	\subsection{Weitere Risiken}
	\begin{table}[h]
		\centering
		\rowcolors{1}{gray!30}{gray!10}
		\begin{tabular}{p{0.24\textwidth}|p{0.34\textwidth}|p{0.34\textwidth}}
			\textbf{Beurteilungskriterien} & \textbf{V1} & \textbf{V2} \\ \hline
			Risiken & & \\ \hline
			Nachhaltigkeit & &
		\end{tabular}
		\caption{Weitere Risiken}
		\label{fig:weitere-risiken}
	\end{table}
	\subsection{Empfehlung}
	
	\section{Detailkonzept}
	\section{Realisierung}
	\newpage
	\section{Wirtschaftlichkeit und Risikoanalyse}
	
	\subsection{Risikoanalyse}
	\subsubsection{Risiken}
	
	\begin{table}[h]
		\centering
		\rowcolors{1}{gray!30}{gray!10}
		\begin{tabular}{p{0.02\textwidth}|p{0.45\textwidth}|p{0.45\textwidth}}
			\textbf{Id} & \textbf{Risiko Beschreibung} & \textbf{Möglicher Schade} \\ \hline
			
			R1 & Kosteneinschätzungen werden überschritten & Projekt wird teurer als ursprünglich geplant. Projektbudget kann nicht eingehalten werden. \\ \hline
			
			R2 & Mitarbeiterausfall & Projektziele können nicht eingehalten werden. Die anderen Mitarbeiter müssen vermehrt Überzeit machen. \\ \hline
			
			R3 & Hardwareausfall & Nach Ausfall muss ausgetauscht / repariert werden was zu Verschiebungen im Zeitplan führen kann \\ \hline
			
			R4 & Projektplan wurde zu sportlich geschätzt. & Meilensteine/Projektziele können nicht eingehalten werden, was zu Verschiebungen im Zeitplan führen kann. \\ \hline
			
			R5 & Naturkatastrophe & Verlust von Hardware, Code, Räumlichkeiten. Ausfall von Mitarbeitern. Es kann vorkommen das der Projektplan nicht eingehalten werden kann oder es zur nicht Durchführung des Projektes kommen. \\ \hline
			
			R6 & Fortschrittsverlust durch falsche Anwendung von Hilfsmitteln (z.B. falsche Versionierung von Dokumenten und Code) & Es geht sehr viel Zeit verloren, um die verlorene Arbeit wieder aufzuholen. Projektziele und Projektplan könnten damit gefährdet sein. \\ \hline
			
			R7 & COVID-19 & Die aktuelle Pandemie kann noch unbekannte Risiken und Einschränkungen mit sich ziehen. \\
		\end{tabular}

		\caption{Risiken}
		\label{fig:risiken}
	\end{table}

	Naturkatastrophen sind in der Schweiz ein kleines Risiko. Können sich aber trotzdem, z.B. Überschwemmungen oder Stürme, stark auf den Projektverlauf auswirken.
	
	Den Coronavirus (COVID-19) haben wir mit als Risiko aufgenommen da es die Aktuelle Lage stark beeinflusst. Wir können nicht vorhersagen welche Folgen der Coronavirus noch mit sich ziehen wird. Zum Beispiel wie stark die Wirtschaft einbrechen und den Fortschritt des Projekts in jeder Hinsicht beeinflussen wird. Oder jeder Beteiligte am Projekt Privat beeinflusst wird.

	\newpage

	\subsubsection{Massnahmen}
	\begin{table}[h]
		\centering
		\rowcolors{1}{gray!30}{gray!10}
		\begin{tabular}{p{0.03\textwidth}|p{0.3\textwidth}|p{0.2\textwidth}|p{0.1\textwidth}|p{0.2\textwidth}}
			\textbf{Nr.} & \textbf{Was?} & \textbf{Wer mit wem?} & \textbf{Wie viel?} & \textbf{Wann?} \\ \hline
			
			M1 & Regelmässige Kontrolle der Kostenfortschritte sowie die Einplanung von Reserven & Oliver Egloff, Leonardo Wiedemeier & & Immer zu Beginn einer neuen Phase \\ \hline
			
			M2 & Einfordern eines Ferienabwesenheitsantrages sowie die Planung aller Ressourcen. Zudem sollte ein regelmässiger Know-How Transfer stattfinden. & Oliver Egloff, Leonardo Wiedemeier & & Bei Ferienabwesenheit \\ \hline
			
			M3 & Hardware angemessen behandel und mit dem Kunden (ABBTS) im Falle eines Ausfalls sofort reagieren & Ganzes Team & & Während des ganzes Projektes \\ \hline
			
			M4 & Einplanen von zeitlichen Reserven & Ganzes Team & 1x pro Woche & Zu Beginn der Planung und am Anfang einer neuen Phase \\ \hline
			
			M6 & Regelmässige Backups der Lösung und klare Definierungen wie gearbeitet wird. & Samuel Salomon, Leonardo Wiedemeier & 1x pro Woche & Ab Beginn Projekt \\ \hline
			
			M7 & Über die Situation mit COVID-19 auf dem Laufenden bleiben und persönliche wie auch geschäftliche Entscheidungen frühzeitig angehen. & Alle am Projekt beteiligten Personen & Täglich & Täglich \\
		\end{tabular}
		\caption{Massnahmen}
		\label{fig:massnahmen}
	\end{table}

	Die meisten Massnahmen haben mit genauer Planung, Kommunikation und mit konzentrierten und genauen Arbeiten zu tun. Naturkatastrophen, Coronavirus und auch Teils Hardwareausfall können uns treffen, dürfen aber nicht überraschen. Mit den hier aufgelisteten Massnahmen gehen wir gegen alle möglichen Risiken gegen an und definieren wie im Falle eins Auftretens reagiert und sich darauf vorbereitet wird.

	\newpage

	\subsubsection{Risikomatrix}
	\begin{figure}[!hbt]
		\centering
		\includegraphics[width=\columnwidth]{risikomatrix.png}
		\caption{Risikomatrix}
		\label{fig:risiko_matrix}
	\end{figure}
	
	Die Risikomatrix zeigt auf wie, die von uns definierten und eingeschätzten, die Risiken (gelbe Kugeln) nach Anwendung der Massnahmen (Schwarze Pfeile) entschärft und neu eingestuft werden (blaue Kugeln).

	\newpage
	
	\section{Abschluss}
	\section{Abbildungsverzeichnis}
	\listoffigures
	
	\section{Tabellenverzeichnis}
	\listoftables
	
	
	\section{Selbstständigkeitserklärung}
	
	\section{Anhang}
	
	
	
\end{document}

