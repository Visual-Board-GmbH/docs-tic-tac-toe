
\documentclass[11pt, a4paper, twoside]{article}

\usepackage{geometry} %Set page layouts
\usepackage{pdfpages} %Include pdf files
\usepackage{fancyhdr} %Header and Footer
\usepackage{graphicx} %Inlcude graphics
\usepackage{lastpage} %Reference the number of pages in your LATEX
\usepackage[german]{babel}
\usepackage{fontspec}
\usepackage{float}
\usepackage{hyperref}
\usepackage{tocloft}
\usepackage{tabularx}
\usepackage{makecell}
%\usepackage{showframe} %For debugging

\hypersetup{
	colorlinks=true,
	linkcolor=black,
	filecolor=magenta,
	urlcolor=cyan,
}

\geometry{top=0.75cm, bottom=0.8cm, right=2.5cm, left=2.5cm, headheight=35pt, includeheadfoot, portrait}
\setmainfont{Calibri}

%Path relative to the .tex file containing the \includegraphics command
\graphicspath{{01_Grafiken/}}

\tocloftpagestyle{fancy}

%Define variables for document
\newcommand{\shortAuthor}{OEG-LWI-SSA}
\newcommand{\dstate}{Draft} %Draft, Progress, Closed
\newcommand{\dversion}{0.1}
\newcommand{\dname}{\shortAuthor\_20200428\_\dversion\_\dstate\_\pageref{LastPage}}
\renewcommand\theadfont{\normalsize}

\begin{document}


%Define header & footer for twoside documents
\fancyhead{} %Clear all header fields
\fancyhead[L]{\includegraphics{header_img_abb2.png}}
\fancyfoot{} %Clear all footer fields
\fancyfoot[RO, LE]{\small \thepage \textbar \pageref{LastPage}}
\fancyfoot[LO, RE]{\small \dname}
\renewcommand{\footrulewidth}{0.4pt}
\renewcommand{\headrulewidth}{0.4pt}
\pagestyle{fancy}

	
	\section*{Dokumentenmanagement}
		\begin{tabularx}{\columnwidth}{l X}
			\textbf{Erstellungsdatum:} & 28.04.2020 \\
			\textbf{Autotren:} & Samuel Salomon (SSA), Oliver Egloff (OEG), Leonardo Wiedemeier (LWI) \\
			\textbf{Dateiname:} &\dname
		\end{tabularx}
	
	
	\section*{Änderungsverzeichnis}
		\begin{tabularx}{\columnwidth}{l l l X}
			\textbf{Version} & \textbf{Datum} & \textbf{Autor} & \textbf{Beschreibung}\\
			0.1 & 28.04.2020 & SSA & Dokumentation und Gliederung erstellt
		\end{tabularx}
	
	\renewcommand{\cftsecleader}{\cftdotfill{\cftdotsep}}
	\tableofcontents
	
	\clearpage
	
	\section{Managment Summary}
	\section{Aufgabenstellung}
	\section{Projektmanagement}
	\section{Situationsanalyse}
	\section{Zielsetzung}
		Das Ziel dieses Projektes ist ein webbasiertes Tic-Tac-Toe Spiel zu realisieren. Bereits bestehende Erfahrungen sowie Ergebnisse des RGB-LED-Matrix Prototypen werden wiederverwendet, um z.B. den Spielstand auf der RGB-LED-Matrix darzustellen. Ausgehend vom Projektziel werden Ziele untenstehend strukturiert nach
		\begin{itemize}
			\item Systemzielen
			\item Prozesszielen
			\item Mehrwert Zielen
		\end{itemize}
		
		aufgelistet. Sämtliche Ziele sind nach den SMART Kriterien definiert:
		\begin{itemize}
		
			\item Spezifisch
			\item Messbar
			\item Akzeptiert
			\item Realistisch
			\item Terminiert
		\end{itemize}
		\subsection{Zieldefinition}
			Die Ziele werden in System-, Prozess- und Mehrwert-Ziele unterteilt. Die Wichtigkeit der Ziele wird anhand der Priorität eingestuft: Muss (M), Soll (S) und Kann (K)
			\subsubsection{Systemziele}
				\begin{table}[h]
					\centering
					
					\begin{tabularx}{\columnwidth}{l X r}
						\textbf{\thead{Nr}} & \textbf{\thead{Zielbeschreibung}} & \textbf{\thead{Priorität \\ (M, S, K)}}\\ \Xhline{2pt}
	
					
					\end{tabularx}
					\caption{Systemziele}
				\end{table}
				
			
			\subsubsection{Prozessziele}
				\begin{table}[h]
					\centering
					\begin{tabularx}{\columnwidth}{l X r}
						\textbf{\thead{Nr}} & \textbf{\thead{Zielbeschreibung}} & \textbf{\thead{Priorität \\ (M, S, K)}}\\ \Xhline{2pt}
						1 & Der Prototyp soll bis zum 27.01.2020 fertig sein. & M \\ \Xhline{1pt}
						2 & Der Prototyp soll am 10.02.2020 den Stakeholdern präsentiert werden. & M \\ \Xhline{1pt}
						3 & Das Projekt wird nach dem Wasserfallmodell umgesetzt. & M \\ \Xhline{1pt}
						4 & Am 11.11.2019 findet das Kick-off-Meeting statt, in welchem die Initialisierungsphase besprochen wird und der Übergang in die Konzeptphase stattfindet.& M \\ \Xhline{1pt}
						5 &Bis zum 09.12.2019 muss die Konzeptionelle Planung abgeschlossen sein, um anschliessend mit der Umsetzung zu beginnen & M \\ \Xhline{1pt}
						6 & Das Projekt wird im Wasserfall-Modell in Kombination mit Agile Projektmethoden umgesetzt. & M \\ \Xhline{1pt}
						7 & Folgende Personen sind zu 100\% in das Projekt involviert:
						-  Oliver Egloff
						-	Leonardo Wiedemeier
						-	Samuel Salomon & M \\ \Xhline{1pt}
						8 & Es muss mindestens jede zweite Woche mit dem Kunden kommuniziert werden. & M  \\ \Xhline{1pt}
						9 &	Die angefallenen Kosten sind zu jedem Zeitpunkt aktuell zu halten und mit dem Soll-Zustand zu vergleichen. & M
						
						
					\end{tabularx}
					\caption{Prozessziele}
				\end{table}
			
			\subsubsection{Mehrwert Ziele}
				\begin{table}[h]
					\centering
					
					\begin{tabularx}{\columnwidth}{l X r}
						\textbf{\thead{Nr}} & \textbf{\thead{Zielbeschreibung}} & \textbf{\thead{Priorität \\ (M, S, K)}}\\ \Xhline{2pt}
						1 & Know-How in der Entwicklung von Web Applikation ausweiten & M \\ \Xhline{1pt}
						2 &	Projektmethoden richtig anwenden & M \\ \Xhline{1pt}
						3 &	Agile Methoden im GIT Umfeld anwenden &	M \\ \Xhline{1pt}
						4 &	Know-How in Projektmitarbeit ausweiten & M \\ \Xhline{1pt}
						5 &	Umgang mit MQTT-Brokern sowie weiteren Web-Technologien wie HTML, CCS und Javascript verbessern & M \\ \Xhline{1pt}
						6 &	Erfahrungen mit Projektmethoden (Wasserfall und Agile) sammeln & M
						
						
						
					\end{tabularx}
					\caption{Mehrwert Ziele}
				\end{table}
	\section{Lösungssuche}
	\section{Lösungswahl}
	\section{Detailkonzept}
	\section{Realisierung}
	\section{Wirtschaftlichkeit und Risikoanalyse}
	\section{Abschluss}
	
	
	\renewcommand\listfigurename{}
	\section{Abbildungsverzeichnis}
	\listoffigures
	
	\renewcommand\listtablename{}
	\section{Tabellenverzeichnis}
	\listoftables
	
	
	\section{Selbstständigkeitserklärung}
	
	\section{Anhang}
	
	
	
\end{document}

