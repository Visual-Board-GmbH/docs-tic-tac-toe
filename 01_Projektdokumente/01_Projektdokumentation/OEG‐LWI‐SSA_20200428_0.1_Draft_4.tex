\documentclass[11pt, a4paper, twoside]{article}

\usepackage[table]{xcolor}
\usepackage{geometry} %Set page layouts
\usepackage{pdfpages} %Include pdf files
\usepackage{fancyhdr} %Header and Footer
\usepackage{graphicx} %Inlcude graphics
\usepackage{lastpage} %Reference the number of pages in your LATEX
\usepackage[german]{babel}
\usepackage{fontspec}
\usepackage{float}
\usepackage{hyperref}
\usepackage{tocloft}
\usepackage{tabularx}
\usepackage{import}
\usepackage{lscape}
\usepackage{pdflscape}
% \usepackage{showframe} %For debugging

\hypersetup{
    colorlinks=true,
    linkcolor=black,
    filecolor=magenta,
    urlcolor=cyan,
}

\geometry{top=0.75cm, bottom=0.8cm, right=2.5cm, left=2.5cm, headheight=35pt, includeheadfoot, portrait}
\setmainfont{Calibri}

%Path relative to the .tex file containing the \includegraphics command
\graphicspath{{01_Grafiken/}}

\tocloftpagestyle{fancy}

%Define variables for document
\newcommand{\shortAuthor}{OEG-LWI-SSA}
\newcommand{\dstate}{Draft} %Draft, Progress, Closed
\newcommand{\dversion}{0.1}
\newcommand{\dname}{\shortAuthor\_20200428\_\dversion\_\dstate\_\pageref{LastPage}}


\begin{document}


%Define header & footer for twoside documents
    \fancyhead{} %Clear all header fields
    \fancyhead[L]{\includegraphics{header_img_abb2.png}}
    \fancyfoot{} %Clear all footer fields
    \fancyfoot[RO, LE]{\small \thepage \textbar \pageref{LastPage}}
    \fancyfoot[LO, RE]{\small \dname}
    \renewcommand{\footrulewidth}{0.4pt}
    \renewcommand{\headrulewidth}{0.4pt}
    \pagestyle{fancy}


    \section*{Dokumentenmanagement}

    \begin{tabularx}{\columnwidth}{l X}
        \textbf{Erstellungsdatum:} & 28.04.2020 \\
        \textbf{Autotren:} & Samuel Salomon (SSA), Oliver Egloff (OEG), Leonardo Wiedemeier (LWI) \\
        \textbf{Dateiname:} &\dname
    \end{tabularx}


    \section*{Änderungsverzeichnis}
    \begin{tabularx}{\columnwidth}{l l l X}
        \textbf{Version} & \textbf{Datum} & \textbf{Autor} & \textbf{Beschreibung}\\
        0.1 & 28.04.2020 & SSA & Dokumentation und Gliederung erstellt \\
        0.2 & 04.05.2020 & LWI & Einfügen der Risikoanalyse \\
        0.3 & 06.05.2020 & LWI & Einfügen der Situationsanalyse \\
        0.4 & 07.05.2020 & LWI & Einfügen der Rollenbeschreibung + kleine Design Änderungen \\
        0.5 & 08.05.2020 & LWI & Einfügen der Ressourcenplanung \\
        0.6 & 11.05.2020 & LWI & Änderung R3/M3 in der Risikoanalyse \\
        0.7 & 12.05.2020 & LWI & Änderung in der Risikoanalyse + Grafik \\
        0.8 & 13.05.2020 & OEG & Einfügen des Kommunikationsplans \\
    \end{tabularx}

    \renewcommand{\cftsecleader}{\cftdotfill{\cftdotsep}}
    \tableofcontents

    \clearpage


    \section{Managment Summary}


    \section{Aufgabenstellung}


    \section{Projektmanagement}

    \subsection{Projektorganisation}
    \begin{figure}[!hbt]
        \centering
        \includegraphics[width=\columnwidth]{Projektorganisation.png}
        \caption{Projektorganisation}
        \label{fig:projektorganisation}
    \end{figure}

    \newpage
    \begin{landscape}
        \subsection{Meeting- und Kommunikationsplan}
        \documentclass[11pt, a4paper, twoside]{article}

\usepackage{geometry} %Set page layouts
\usepackage{pdfpages} %Include pdf files
\usepackage{fancyhdr} %Header and Footer
\usepackage{graphicx} %Inlcude graphics
\usepackage{lastpage} %Reference the number of pages in your LATEX
\usepackage[german]{babel}
\usepackage{fontspec}
\usepackage{float}
\usepackage{hyperref}
\usepackage{tocloft}
%\usepackage{showframe} %For debugging

\hypersetup{
    colorlinks=true,
    linkcolor=black,
    filecolor=magenta,
    urlcolor=cyan,
}

\geometry{top=0.75cm, bottom=0.8cm, right=2.5cm, left=2.5cm, headheight=35pt, includeheadfoot, landscape}
\setmainfont{Calibri}

%Path relative to the .tex file containing the \includegraphics command
\graphicspath{ {01_Grafiken/}}

%Set the pagestyle of content tables from plain to fancy
\tocloftpagestyle{fancy}

%Define variables for document
\newcommand{\shortAuthor}{OEG-LWI-SSA}
\newcommand{\dstate}{Draft} %Draft, Progress, Closed
\newcommand{\dversion}{0.1}
\newcommand{\dname}{\shortAuthor\_YYYYMMDD\_\dversion\_\dstate\_\pageref{LastPage}}

\begin{document}
    %Define header & footer for twoside documents
    \fancyhead{} %Clear all header fields
    \fancyhead[L]{\includegraphics{header_img_abb2.png}}
    \fancyfoot{} %Clear all footer fields
    \fancyfoot[RO, LE]{\small \thepage \textbar \pageref{LastPage}}
    \fancyfoot[LO, RE]{\small \dname}
    \renewcommand{\footrulewidth}{0.4pt}
    \renewcommand{\headrulewidth}{0.4pt}

    %Define header & footer for oneside documents
    %\fancyhead{} %Clear all header fields
    %\fancyhead[L]{\includegraphics{01_Grafiken/header_img_abb2.png}}
    %\fancyfoot{}
    %\fancyfoot[R]{\small \thepage \textbar \pageref{LastPage}}
    %\fancyfoot[L]{\dname}
    %\renewcommand{\footrulewidth}{0.4pt}
    %\renewcommand{\headrulewidth}{0.4pt}
    \pagestyle{fancy}

    \begin{table}[h]
        \centering
        \rowcolors{1}{gray!30}{gray!10}
        \begin{tabular}{p{0.02\textwidth}|p{0.45\textwidth}|p{0.45\textwidth}}
            \textbf{Id} & \textbf{Risiko Beschreibung} & \textbf{Möglicher Schade} \\ \hline

            R1 & Kosteneinschätzungen werden überschritten & Projekt wird teurer als ursprünglich geplant. Projektziele können nicht eingehalten werden. \\ \hline

            R2 & Mitarbeiterausfall & Projektziele können nicht eingehalten werden. Die anderen Mitarbeiter müssen vermehrt Überzeit machen. \\ \hline

            R3 & Probleme mit Hostinsg / Hardware & Unstimmigkeiten oder Probleme vom Hosting Partner können zu Verschiebungen im Zeitplan oder Mehrkosten führen. \\ \hline

            R4 & Projektplan wurde zu sportlich geschätzt. & Verlust von Hardware, Code, Räumlichkeiten. Ausfall von Mitarbeitern. Es kann vorkommen das der Projektplan nicht eingehalten werden kann oder es zur nicht Durchführung des Projektes kommen. \\ \hline

            R5 & Naturkatastrophe & Verlust von Hardware, Code, Räumlichkeiten. Ausfall von Mitarbeitern. Es kann vorkommen das der Projektplan nicht eingehalten werden kann oder es zur nicht Durchführung des Projektes kommen. \\ \hline

            R6 & Fortschrittsverlust durch falsche Anwendung von Hilfsmitteln (z.B. falsche Versionierung von Dokumenten und Code) & Es geht sehr viel Zeit verloren, um die verlorene Arbeit wieder aufzuholen. Projektziele und Projektplan könnten damit gefährdet sein. \\ \hline

            R7 & COVID-19 & Die aktuelle Pandemie kann noch unbekannte Risiken und Einschränkungen mit sich ziehen. \\
        \end{tabular}

        \caption{Risiken}
        \label{fig:risiken}
    \end{table}

\end{document}




    \end{landscape}
    \begin{figure}[!hbt]
        \centering
        \includegraphics[width=\columnwidth]{02_Dokumente/02_Kommunikationsplan/Kommunikationsplan.png}
        \caption{Kommunikationswege}
        \label{fig:kommunikations_wege}
    \end{figure}

    \newpage

    \subsubsection{Rollenbeschreibung}
    \begin{table}[h]
        \centering
        \rowcolors{1}{gray!30}{gray!10}
        \begin{tabular}{p{0.3\textwidth}|p{0.65\textwidth}}
            \textbf{Kürzel} & \textbf{Beschreibung} \\ \hline
            Auftraggeber & Er ist der Kunde und liefert das Lastenheft. \\ \hline

            Projektsteuerungs-Gremium & Trifft wichtige Entscheidungen und ist bei Problemen umgehend zu informieren. \\ \hline

            Projektportfolio-Controller & Hat alle laufenden Projekte im Überblick und ist Ansprechpartner für den Projektleiter und die Projektträger. \\ \hline

            Projektleiter & Die führende Kraft im Projektteam. Ist Ansprechpartner für die Involvierten und Projektträger. \\ \hline

            Projektmitarbeiter & Alle Betroffenen und Involvierten Personen/Organisation, die am Projekt mitarbeiten. \\ \hline

            Frontend Entwicklung & Hauptverantwortlicher des Frontend. \\ \hline

            Backend Entwicklung & Hauptverantwortlicher des Backend. \\ \hline

            Entwicklungsleitung & Hat Übersicht des Zusammenspiels von Front- und Backend. \\ \hline

            Software \& System Architekt & Übersicht der eingesetzten Tools und des Systemaufbaus. \\ \hline

            Datenbank Manager \& Architekt & Hauptverantwortlicher der Datenbank. \\ \hline

            Dokumentenverantwortlicher & Ist verantwortlich für den Aufbau und die Struktur der Dokumente. \\ \hline

            Requirements \& Business Engineering & Hat die Übersicht über alle Anforderungen und ihrer Einhaltung \\ \hline

            Quality \& Testing & Verantwortlich für die Qualität des Codes und das er getestet ist. \\
        \end{tabular}
        \caption{Rollenbeschreibung}
        \label{fig:rollenbeschreibung}
    \end{table}


    \section{Situationsanalyse}

    \subsection{Vorgänger Projekt}
    Die Visual Board GmbH hat bereits mit der ABBTS (Rolf Hirschi) zusammengearbeitet und einen Prototypen erstellt um Bilder auf einer RGB-LED-Matrix anzuzeigen.
    Die RGB-LED-Matrix wurde funktionsfähig gestellt und es sollte ein Programm geschrieben werden um Bilder wie gewünscht auf der Matrix anzuzeigen. Darunter ist das korrekte transformieren, aufbereiten und weitergeben an die Matrix zu verstehen.

    Der Prototyp wurde von der Visual Board GmbH mit einer REST-Api in Java umgesetzt welche auf dem Raspberry Pi der Hardware, oder einem externen Windows Computer läuft.
    So wurde die RGB-LED-Matrix, dem Netwerk in welchem sie sich befindet hat, zur verfügung gsetellt. Und es ist möglich mit den korrekten "requests" Bilder direkt oder mit dem Webinterface an die Matrix zu schicken.

    Der Visual Board GmbH wurde die komplette Hardware durch die ABBTS gestellt wurde.
    \newline
    Teil der Projektorganisiton waren:
    \begin{itemize}
        \item ABBTS (Rolf Hirschi) als Auftraggeber
        \item Visual Board GmbH als Arbeitnehmer. Samuel Salomon hat die Position als Projektleiter eingenommen.
        \item Herr Künzli und Herr Jenzer jeweils als Mitglied des PSG (Projektsteuerungs-Gremium)
        \item Herr Künzli hatte zudem die Position des PPC  (Projektportfolio-Controller
    \end{itemize}

    \subsection{ Weiteres Vorgehen}
    Die ABBTS hat sich entschieden auf den Prototypen der Visual Board GmbH aufzubauen und die Spiellogic für ein Tic-Tac-Toe Spiel umzusetzen.
    Es soll die Möglichkeit bestehen das mehrere Spieler, auf verschiedenen Endgeräten, gegeneinander spielen, während die RGB-LED-Matrix das aktuelle Spielfeld ausgiebt. Hier für werden wir das komplette Vorgänger Projekt verwenden, da die Matrix bereits einsatzbereit ist. Es werden allenfalls Schnittstellen angepasst oder neue erstellt.
    Zusätzlich zur implementierung der Spiellogic, soll diese auch auf einen Webserver laufen und so erreichbar gemacht werden.

    Die Visual Board GmbH wird daher im Rahmen des Projektes die Spiellogic mit Benutzeroberfläche implementieren und den nötigen Webserver einrichten. Für diese Arbeiten stehen 300 Arbeitsstunden der kompletten Projektabwicklung zur Verfügung.


    \section{Zielsetzung}


    \section{Lösungssuche}


    \section{Lösungswahl}


    \section{Detailkonzept}


    \section{Realisierung}
    \newpage


    \section{Wirtschaftlichkeit und Risikoanalyse}

    \subsection{Risikoanalyse}

    \subsubsection{Risiken}
    \begin{table}[h]
        \centering
        \rowcolors{1}{gray!30}{gray!10}
        \begin{tabular}{p{0.02\textwidth}|p{0.45\textwidth}|p{0.45\textwidth}}
            \textbf{Id} & \textbf{Risiko Beschreibung} & \textbf{Möglicher Schade} \\ \hline

            R1 & Kosteneinschätzungen werden überschritten & Projekt wird teurer als ursprünglich geplant. Projektziele können nicht eingehalten werden. \\ \hline

            R2 & Mitarbeiterausfall & Projektziele können nicht eingehalten werden. Die anderen Mitarbeiter müssen vermehrt Überzeit machen. \\ \hline

            R3 & Probleme mit Hosting / Hardware & Unstimmigkeiten oder Probleme vom Hosting Partner können zu Verschiebungen im Zeitplan oder Mehrkosten führen. \\ \hline

            R4 & Projektplan wurde zu sportlich geschätzt. & Verlust von Hardware, Code, Räumlichkeiten. Ausfall von Mitarbeitern. Es kann vorkommen das der Projektplan nicht eingehalten werden kann oder es zur nicht Durchführung des Projektes kommen. \\ \hline

            R5 & Naturkatastrophe & Verlust von Hardware, Code, Räumlichkeiten. Ausfall von Mitarbeitern. Es kann vorkommen das der Projektplan nicht eingehalten werden kann oder es zur nicht Durchführung des Projektes kommen. \\ \hline

            R6 & Fortschrittsverlust durch falsche Anwendung von Hilfsmitteln (z.B. falsche Versionierung von Dokumenten und Code) & Es geht sehr viel Zeit verloren, um die verlorene Arbeit wieder aufzuholen. Projektziele und Projektplan könnten damit gefährdet sein. \\ \hline

            R7 & COVID-19 & Die aktuelle Pandemie kann noch unbekannte Risiken und Einschränkungen mit sich ziehen. \\
        \end{tabular}

        \caption{Risiken}
        \label{fig:risiken}
    \end{table}

    \newpage

    \subsubsection{Massnahmen}
    \begin{table}[h]
        \centering
        \rowcolors{1}{gray!30}{gray!10}
        \begin{tabular}{p{0.03\textwidth}|p{0.3\textwidth}|p{0.2\textwidth}|p{0.1\textwidth}|p{0.2\textwidth}}
            \textbf{Nr.} & \textbf{Was?} & \textbf{Wer mit wem?} & \textbf{Wie viel?} & \textbf{Wann?} \\ \hline

            M1 & Regelmässige Kontrolle der Kostenfortschritte sowie die Einplanung von Reserven & Oliver Egloff & & Immer zu Beginn einer neuen Phase \\ \hline

            M2 & Einfordern eines Ferienabwesenheitsantrages sowie die Planung aller Ressourcen. Zudem sollte ein regelmässiger Know-How Transfer stattfinden. & Oliver Egloff, Leonardo Wiedemeier & & Bei Ferienabwesenheit \\ \hline

            M3 & Hardware angemessen behandel und mit dem Kunden (ABBTS) im Falle eines Ausfalls sofort reagieren & Ganzes Team & & Während des ganzes Projektes \\ \hline

            M4 & Einplanen von zeitlichen Reserven & Ganzes Team & 1x pro Woche & Zu Beginn der Planung und am Anfang einer neuen Phase \\ \hline

            M6 & Regelmässige Backups der Lösung und klare Definierungen wie gearbeitet wird. & Samuel Salomon, Leonardo Wiedemeier & 1x pro Woche & Ab Begin Projekt \\ \hline

            M7 & Über die Situation mit COVID-19 auf dem Laufenden bleiben und persönliche wie auch Geschäftliche Entscheidungen Frühzeitig angehen. & Alle am Projekt beteiligten Personen & Täglich & Täglich \\
        \end{tabular}
        \caption{Massnahmen}
        \label{fig:massnahmen}
    \end{table}

    \newpage

    \subsubsection{Risikomatrix}
    \begin{figure}[!hbt]
        \centering
        \includegraphics[width=\columnwidth]{risikomatrix.png}
        \caption{Risikomatrix}
        \label{fig:risiko_matrix}
    \end{figure}

    \newpage


    \section{Abschluss}


    \section{Abbildungsverzeichnis}
    \listoffigures


    \section{Tabellenverzeichnis}
    \listoftables


    \section{Selbstständigkeitserklärung}


    \section{Anhang}


\end{document}

