\documentclass[11pt, a4paper, twoside]{article}

\usepackage[table]{xcolor}
\usepackage{geometry} %Set page layouts
\usepackage{pdfpages} %Include pdf files
\usepackage{fancyhdr} %Header and Footer
\usepackage{graphicx} %Inlcude graphics
\usepackage{lastpage} %Reference the number of pages in your LATEX
\usepackage[german]{babel}
\usepackage{fontspec}
\usepackage{float}
\usepackage{hyperref}
\usepackage{tocloft}
\usepackage{tabularx}
\usepackage{import}
\usepackage{lscape}
\usepackage{multirow}
\usepackage{makecell}
\usepackage{pdflscape}
\usepackage{longtable}
% \usepackage{showframe} %For debugging

\hypersetup{
    colorlinks=true,
    linkcolor=black,
    filecolor=magenta,
    urlcolor=cyan,
}

\geometry{top=0.75cm, bottom=0.8cm, right=2.5cm, left=2.5cm, headheight=35pt, includeheadfoot, portrait}
\setmainfont{Calibri}

%Path relative to the .tex file containing the \includegraphics command
\graphicspath{{01_Grafiken/}}

\tocloftpagestyle{fancy}

%Define variables for document
\newcommand{\shortAuthor}{OEG-LWI-SSA}
\newcommand{\dstate}{Draft} %Draft, Progress, Closed
\newcommand{\dversion}{0.1}
\newcommand{\dname}{\shortAuthor\_20200428\_\dversion\_\dstate\_\pageref{LastPage}}
%\renewcommand\font{\normalsize}
\begin{document}


%Define header & footer for twoside documents
\fancyhead{} %Clear all header fields
\fancyhead[L]{\includegraphics{header_img_abb2.png}}
\fancyfoot{} %Clear all footer fields
\fancyfoot[RO, LE]{\small \thepage \textbar \pageref{LastPage}}
\fancyfoot[LO, RE]{\small \dname}
\renewcommand{\footrulewidth}{0.4pt}
\renewcommand{\headrulewidth}{0.4pt}
\pagestyle{fancy}


	\section{Dokumentenmanagement}

	\begin{tabularx}{\columnwidth}{l X}
		\textbf{Erstellungsdatum:} & 28.04.2020 \\
		\textbf{Autotren:} & Samuel Salomon (SSA), Oliver Egloff (OEG), Leonardo Wiedemeier (LWI) \\
		\textbf{Dateiname:} &\dname
	\end{tabularx}


\section{Änderungsverzeichnis}
	\begin{tabularx}{\columnwidth}{l l l X}
		\textbf{Version} & \textbf{Datum} & \textbf{Autor} & \textbf{Beschreibung}\\
		0.1 & 28.04.2020 & SSA & Dokumentation und Gliederung erstellt \\
		0.2 & 04.05.2020 & LWI & Einfügen der Risikoanalyse \\
		0.3 & 06.05.2020 & LWI & Einfügen der Situationsanalyse \\
		0.4 & 07.05.2020 & LWI & Einfügen der Rollenbeschreibung + kleine Design Änderungen \\
		0.5 & 08.05.2020 & LWI & Einfügen der Ressourcenplanung \\
		0.6 & 11.05.2020 & LWI & Änderung R3/M3 in der Risikoanalyse \\
		0.7 & 12.05.2020 & LWI & Änderung in der Risikoanalyse + Grafik \\
		0.8 & 13.05.2020 & OEG & Einfügen des Kommunikationsplans \\
		0.9 & 15.05.2020 & OEG & Einfügen Anhang Kick-Off Meeting \\
		0.10 & 19.05.2020 & LWI & Risikomatrix mit Texten erweitern \\
		0.11 & 06.06.2020 & LWI & Lösungssuche und Lösungswahl vorbereitet \\
		0.12 & 08.06.2020 & SSA, OEG, LWI & Mind Map für die Lösungssuche erstellt\\
		0.13 & 16.06.2020 & LWI & fertigstellen der Lösungssuche und Lösungswahl \\
		0.14 & 17.06.2020 & SSA & Zieldefinition \\
		0.15 & 23.06.2020 & LWI & Funktionendiagramm \\
	\end{tabularx}
	
	\renewcommand{\cftsecleader}{\cftdotfill{\cftdotsep}}
	
	\tableofcontents


\clearpage


\section{Managment Summary}
	\section{Aufgabenstellung}
	\section{Projektmanagement}
    \subsection{Projektorganisation}
    \begin{figure}[!hbt]
        \centering
        \includegraphics[width=\columnwidth]{Projektorganisation.png}
        \caption{Projektorganisation}
        \label{fig:projektorganisation}
    \end{figure}

	\newpage
   
    \begin{landscape}
        \subsection{Meeting- und Kommunikationsplan}
        \documentclass[11pt, a4paper, twoside]{article}

\usepackage{geometry} %Set page layouts
\usepackage{pdfpages} %Include pdf files
\usepackage{fancyhdr} %Header and Footer
\usepackage{graphicx} %Inlcude graphics
\usepackage{lastpage} %Reference the number of pages in your LATEX
\usepackage[german]{babel}
\usepackage{fontspec}
\usepackage{float}
\usepackage{hyperref}
\usepackage{tocloft}
%\usepackage{showframe} %For debugging

\hypersetup{
    colorlinks=true,
    linkcolor=black,
    filecolor=magenta,
    urlcolor=cyan,
}

\geometry{top=0.75cm, bottom=0.8cm, right=2.5cm, left=2.5cm, headheight=35pt, includeheadfoot, landscape}
\setmainfont{Calibri}

%Path relative to the .tex file containing the \includegraphics command
\graphicspath{ {01_Grafiken/}}

%Set the pagestyle of content tables from plain to fancy
\tocloftpagestyle{fancy}

%Define variables for document
\newcommand{\shortAuthor}{OEG-LWI-SSA}
\newcommand{\dstate}{Draft} %Draft, Progress, Closed
\newcommand{\dversion}{0.1}
\newcommand{\dname}{\shortAuthor\_YYYYMMDD\_\dversion\_\dstate\_\pageref{LastPage}}

\begin{document}
    %Define header & footer for twoside documents
    \fancyhead{} %Clear all header fields
    \fancyhead[L]{\includegraphics{header_img_abb2.png}}
    \fancyfoot{} %Clear all footer fields
    \fancyfoot[RO, LE]{\small \thepage \textbar \pageref{LastPage}}
    \fancyfoot[LO, RE]{\small \dname}
    \renewcommand{\footrulewidth}{0.4pt}
    \renewcommand{\headrulewidth}{0.4pt}

    %Define header & footer for oneside documents
    %\fancyhead{} %Clear all header fields
    %\fancyhead[L]{\includegraphics{01_Grafiken/header_img_abb2.png}}
    %\fancyfoot{}
    %\fancyfoot[R]{\small \thepage \textbar \pageref{LastPage}}
    %\fancyfoot[L]{\dname}
    %\renewcommand{\footrulewidth}{0.4pt}
    %\renewcommand{\headrulewidth}{0.4pt}
    \pagestyle{fancy}

    \begin{table}[h]
        \centering
        \rowcolors{1}{gray!30}{gray!10}
        \begin{tabular}{p{0.02\textwidth}|p{0.45\textwidth}|p{0.45\textwidth}}
            \textbf{Id} & \textbf{Risiko Beschreibung} & \textbf{Möglicher Schade} \\ \hline

            R1 & Kosteneinschätzungen werden überschritten & Projekt wird teurer als ursprünglich geplant. Projektziele können nicht eingehalten werden. \\ \hline

            R2 & Mitarbeiterausfall & Projektziele können nicht eingehalten werden. Die anderen Mitarbeiter müssen vermehrt Überzeit machen. \\ \hline

            R3 & Probleme mit Hostinsg / Hardware & Unstimmigkeiten oder Probleme vom Hosting Partner können zu Verschiebungen im Zeitplan oder Mehrkosten führen. \\ \hline

            R4 & Projektplan wurde zu sportlich geschätzt. & Verlust von Hardware, Code, Räumlichkeiten. Ausfall von Mitarbeitern. Es kann vorkommen das der Projektplan nicht eingehalten werden kann oder es zur nicht Durchführung des Projektes kommen. \\ \hline

            R5 & Naturkatastrophe & Verlust von Hardware, Code, Räumlichkeiten. Ausfall von Mitarbeitern. Es kann vorkommen das der Projektplan nicht eingehalten werden kann oder es zur nicht Durchführung des Projektes kommen. \\ \hline

            R6 & Fortschrittsverlust durch falsche Anwendung von Hilfsmitteln (z.B. falsche Versionierung von Dokumenten und Code) & Es geht sehr viel Zeit verloren, um die verlorene Arbeit wieder aufzuholen. Projektziele und Projektplan könnten damit gefährdet sein. \\ \hline

            R7 & COVID-19 & Die aktuelle Pandemie kann noch unbekannte Risiken und Einschränkungen mit sich ziehen. \\
        \end{tabular}

        \caption{Risiken}
        \label{fig:risiken}
    \end{table}

\end{document}




    \end{landscape}
    \begin{figure}[!hbt]
        \centering
        \includegraphics[width=\columnwidth]{02_Dokumente/02_Kommunikationsplan/01_Grafiken/Kommunikations- und Dokumentationssytem.png}
        \caption{Kommunikationswege}
        \label{fig:kommunikations_wege}
    \end{figure}

    \newpage

    \subsubsection{Rollenbeschreibung}
    \begin{table}[h]
        \centering
        \rowcolors{1}{gray!30}{gray!10}
        \begin{tabular}{p{0.3\textwidth}|p{0.65\textwidth}}
            \textbf{Kürzel} & \textbf{Beschreibung} \\ \hline
            Auftraggeber & Er ist der Kunde und liefert das Lastenheft. \\ \hline

            Projektsteuerungs-Gremium & Trifft wichtige Entscheidungen und ist bei Problemen umgehend zu informieren. \\ \hline

            Projektportfolio-Controller & Hat alle laufenden Projekte im Überblick und ist Ansprechpartner für den Projektleiter und die Projektträger. \\ \hline

            Projektleiter & Die führende Kraft im Projektteam. Ist Ansprechpartner für die Involvierten und Projektträger. \\ \hline

            Projektmitarbeiter & Alle Betroffenen und Involvierten Personen/Organisation, die am Projekt mitarbeiten. \\ \hline

            Frontend Entwicklung & Hauptverantwortlicher des Frontend. \\ \hline

            Backend Entwicklung & Hauptverantwortlicher des Backend. \\ \hline

            Entwicklungsleitung & Hat Übersicht des Zusammenspiels von Front- und Backend. \\ \hline

            Software \& System Architekt & Übersicht der eingesetzten Tools und des Systemaufbaus. \\ \hline

            Datenbank Manager \& Architekt & Hauptverantwortlicher der Datenbank. \\ \hline

            Dokumentenverantwortlicher & Ist verantwortlich für den Aufbau und die Struktur der Dokumente. \\ \hline

            Requirements \& Business Engineering & Hat die Übersicht über alle Anforderungen und ihrer Einhaltung \\ \hline

            Quality \& Testing & Verantwortlich für die Qualität des Codes und das er getestet ist. \\
        \end{tabular}
        \caption{Rollenbeschreibung}
        \label{fig:rollenbeschreibung}
    \end{table}

	\newpage

	\subsubsection{Funktionendiagramm}
	\begin{center}
		\rowcolors{1}{gray!30}{gray!10}
		\begin{longtable}{|p{0.1\textwidth}|p{0.34\textwidth}|p{0.1\textwidth}|p{0.1\textwidth}|p{0.1\textwidth}|p{0.1\textwidth}|}
			\hline \textbf{Nr.} & \textbf{Leistung} & \textbf{ABBTS} & \textbf{SSA} & \textbf{OEG} & \textbf{LWI} \\ \hline
			\textbf{L1} & \textbf{Projektanalyse} & & & &  \\ \hline
			AP1.1 & Situationsanalyse & I & M & M & D  \\ \hline
			AP1.2 & Anforderungskatalog  & I & D & M & M  \\ \hline
			AP1.3 & Ziele definieren & I & D & M & M \\ \hline
			AP1.4 & Risikoanalyse & I & M & M & D  \\ \hline
			
			\textbf{L2} & \textbf{Projektplanung}  & & & &  \\ \hline
			AP2.1 & Funktionendiagramm  & I & M & M & D  \\ \hline
			AP2.3 & Kommunikationsplan  & I & M & D & M  \\ \hline
			AP2.2 & Rollenbeschreibung & I & M & M & D  \\ \hline
			AP2.4 & Projektorganisation & I & M & D & M  \\ \hline
			AP2.5 & Projektrollen definieren & I & M & M & D \\ \hline
			AP2.6 & Projektstrukturplan & I & D & D & D  \\ \hline
			AP2.7 & Projektzeitplan & I & D & M & D \\ \hline
			AP2.8 & Kostenplanung & I & M & D & M  \\ \hline
			AP2.9 & Ressourcenplanung & I & D & D & D \\ \hline
			
			\textbf{L3} & \textbf{Lösungsevaluation} & & & &  \\ \hline
			AP3.1 & Lösungssuche & I & D & D & D  \\ \hline
			AP3.2 & Lösungsevaluation & I & D & D & D  \\ \hline
			AP3.3 & Lösungsentscheid & I & D & D & D \\ \hline
			
			\textbf{L4} & \textbf{Pflichtenheft} & & & &  \\ \hline
			AP4.1 & Chancen \& Risiken analysieren & I & M & M & D \\ \hline
			AP4.2 & Massnahmen definieren & I & M & M & D \\ \hline
			AP4.3 & Technische Umsetzung planen & I & D & D & D  \\ \hline
			AP4.4 & Vorbereitung Abnahme Pflichtenheft & I & M & D & D \\ \hline
			AP4.5 & Pflichtenheft Dokument erstellen & I & M & D & M \\ \hline
			AP4.6 & Mockups für das Pflichtenheft erstellen & I & D & M & M \\ \hline
			
			\textbf{L5} & \textbf{Kommunikationsschnittstellen definieren (Backend)} & & & & \\ \hline
			AP5.1 & Datenstrukturen definieren & I & D & D & D \\ \hline
			AP5.2 & Datenbankanbindung & I & D & D & D \\ \hline
			AP5.3 & Kommunikation mit LED-RGB-Matrix & I & D & M & M \\ \hline
			AP5.4 & MQTT (Client) Kommunikation & I & D & D & D \\ \hline
			
			\textbf{L6} & \textbf{Spiellogik definieren (Backend)} & & & & \\ \hline
			AP6.1 & Datenbank Schema & I & M & M & D \\ \hline
			AP6.2 & Businesslogik (Spiellogik) & I & D & D & D \\ \hline
			
			\textbf{L7} & \textbf{Design \& Logik definieren (Frontend)} & & & & \\ \hline
			AP7.1 & Frontend nach Mockups erstellen & I & D & D & D \\ \hline
			AP7.2 & Spiellogik in Frontend einbinden & I & D & D & D \\ \hline
			
			\textbf{L8} & \textbf{Kommunikationsschnittstellen definieren (Frontend)} & & & & \\ \hline
			AP8.1 & Kommunikation mit Backend (MQTT) & I & D & D & D \\ \hline
			
			\textbf{L9} & \textbf{Projektdokumentation} & & & & \\ \hline
			AP9.1 & Klassendiagramm & I & M & D & M \\ \hline
			AP9.1 & Lösungsdokumentation & I & D & D & D \\ \hline
			\rowcolor{white} \caption{Funktionsdiagramm}
			\label{fig:funktionsdiagramm}
		\end{longtable}
	\end{center}
	
	\begin{table}[h]
		\centering
		\begin{tabular}{|p{0.46\textwidth} p{0.46\textwidth}|}
			\hline Legende & \\
			E = Entscheiden & K = Koordination, Veranlassung  \\
			I = Information & D = Durchführung, Verantwortung  \\
			M = Mitarbeiter, Beratung &  \\ \hline
		\end{tabular}
	\end{table}

	\newpage

	\subsection{Ressourcenplanung}
	\begin{table}[h]
		\centering
		\rowcolors{1}{gray!30}{gray!10}
		\begin{tabular}{p{0.15\textwidth}|p{0.1\textwidth}|p{0.1\textwidth}|p{0.15\textwidth}|p{0.15\textwidth}|p{0.15\textwidth}}
			\textbf{Ressourcenname} & \textbf{Typ} & \textbf{Initials} &\textbf{Rolle}&\textbf{Stundensatz}&\textbf{Kosten pro Stunde} \\ \hline

            Oliver Egloff & Person / Gruppe & OEG & Projektmanagement & CHF 100.00/h & CHF 0.00 \\\hline

            Leonardo Wiedemeier & Person / Gruppe & LWI & Projectworker & CHF 100.00/h & CHF 0.00 \\\hline

            Samuel Salomon & Person / Gruppe & SSA & Projektworker & CHF 100.00/h & CHF 0.00 \\\hline

            Stakeholder & Person / Gruppe & S & Gremium & CHF 0.00 & CHF 0.00 \\
		\end{tabular}
		\caption{Stundensatz}
		\label{fig:stundensatz}
	\end{table}

	\newpage
	
	\section{Situationsanalyse}
	\subsection{Vorgänger Projekt}
	Die Visual Board GmbH hat bereits mit der ABBTS (Rolf Hirschi) zusammengearbeitet und einen Prototypen erstellt um Bilder auf einer RGB-LED-Matrix anzuzeigen.
	Die RGB-LED-Matrix wurde funktionsfähig gestellt und es sollte ein Programm geschrieben werden um Bilder wie gewünscht auf der Matrix anzuzeigen. Darunter ist das korrekte transformieren, aufbereiten und weitergeben an die Matrix zu verstehen.
	
	Der Prototyp wurde von der Visual Board GmbH mit einer REST-Api in Java umgesetzt welche auf dem Raspberry Pi der Hardware, oder einem externen Windows Computer läuft.
	So wurde die RGB-LED-Matrix, dem Netzwerk in welchem sie sich befindet hat, zur Verfügung gestellt. Und es ist möglich mit den korrekten "requests" Bilder direkt oder mit dem Webinterface an die Matrix zu schicken.
		
	Der Visual Board GmbH wurde die komplette Hardware durch die ABBTS gestellt wurde.
	\newline
	Teil der Projektorganisiton waren:
	\begin{itemize}
		\item ABBTS (Rolf Hirschi) als Auftraggeber
		\item Visual Board GmbH als Arbeitnehmer. Samuel Salomon hat die Position als Projektleiter eingenommen.
		\item Herr Künzli und Herr Jenzer jeweils als Mitglied des PSG (Projektsteuerungs-Gremium)
		\item Herr Künzli hatte zudem die Position des PPC  (Projektportfolio-Controller
	\end{itemize}

	\subsection{Weiteres Vorgehen}
	Die ABBTS hat sich entschieden auf den Prototypen der Visual Board GmbH aufzubauen und die Spiellogik für ein Tic-Tac-Toe Spiel umzusetzen.
	Es soll die Möglichkeit bestehen das mehrere Spieler, auf verschiedenen Endgeräten, gegeneinander spielen, während die RGB-LED-Matrix das aktuelle Spielfeld ausgibt. Hier für werden wir das komplette Vorgänger Projekt verwenden, da die Matrix bereits einsatzbereit ist. Es werden allenfalls Schnittstellen angepasst oder neue erstellt.
	Zusätzlich zur Implementierung der Spiellogik, soll diese auch auf einen Webserver laufen und so erreichbar gemacht werden.
   
    Die Visual Board GmbH wird daher im Rahmen des Projektes die Spiellogic mit Benutzeroberfläche implementieren und den nötigen Webserver einrichten. Für diese Arbeiten stehen 300 Arbeitsstunden der kompletten Projektabwicklung zur Verfügung.


\section{Zielsetzung}
Das Ziel dieses Projektes ist ein webbasiertes Tic-Tac-Toe Spiel zu realisieren. Bereits bestehende Erfahrungen sowie Ergebnisse des RGB-LED-Matrix Prototypen werden wiederverwendet, um z.B. den Spielstand auf der RGB-LED-Matrix darzustellen. Ausgehend vom Projektziel werden Ziele untenstehend strukturiert nach
\begin{itemize}
	\item Systemzielen
	\item Prozesszielen
	\item Mehrwert Zielen
\end{itemize}

aufgelistet. Sämtliche Ziele sind nach den SMART Kriterien definiert:
\begin{itemize}
	
	\item Spezifisch
	\item Messbar
	\item Akzeptiert
	\item Realistisch
	\item Terminiert
\end{itemize}
\subsection{Zieldefinition}
Die Ziele werden in System-, Prozess- und Mehrwert-Ziele unterteilt. Die Wichtigkeit der Ziele wird anhand der Priorität eingestuft: Muss (M), Soll (S) und Kann (K)
\subsubsection{Systemziele}
\begin{table}[h]
	\centering
	\begin{tabularx}{\columnwidth}{l X r}
		\textbf{\thead{Nr}} & \textbf{\thead{Zielbeschreibung}} & \textbf{\thead{Priorität \\ (M, S, K)}}\\ \Xhline{2pt}
		1 & Das Tic-Tac-Toe spiel muss Multiplayer fähig sein & M \\ \Xhline{1pt}
		2 & Spielhistorien müssen vom benutzer abgerufen werden können & M \\ \Xhline{1pt}
		3 & Aktuelle Spielzüge müssen auf der RGB-LED-Matrix dargestellt werden & M \\ \Xhline{1pt}
		4 & Objekte sollen in Universellen Formaten gespeichert werden um Sie für späterew Projekte wieder zu verwenden & S \\ \Xhline{1pt}
		5 & Dem Benutzer muss ein Web-GUI zur Verfügung gestellt werden & M \\ \Xhline{1pt}
		6 & Vom Benutzer erzeugte Daten müssen in einer zentralen MySQL Datenbank abgelegt werden & M \\ \Xhline{1pt}
		7 & Der Benutzer muss sich am System anmelden können & M \\ \Xhline{1pt}
		8 & Die allgemeinen Speilregelen des offiziellen Tic-Tac-Toe Spiels müssen abgebildet werden & M \\ \Xhline{1pt}
	\end{tabularx}
	\caption{Systemziele}
\end{table}


\subsubsection{Prozessziele}
\begin{table}[h]
	\centering
	\begin{tabularx}{\columnwidth}{l X r}
		\textbf{\thead{Nr}} & \textbf{\thead{Zielbeschreibung}} & \textbf{\thead{Priorität \\ (M, S, K)}}\\ \Xhline{2pt}
		1 & Das System muss bis zum 26.08.2020 fertiggestellt werden. & M \\ \Xhline{1pt}
		2 & Der Prototyp soll am 26.08.2020 den Stakeholdern präsentiert werden. & M \\ \Xhline{1pt}
		3 & Das Projekt wird nach dem Wasserfallmodell unter Berücksichtigung Agiler Projektmethoden für die Entwicklung umgesetzt. & M \\ \Xhline{1pt}
		4 & Am 13.05.2020 findet das Kick-off-Meeting statt, in welchem die Initialisierungsphase besprochen wird und der Übergang in die Konzeptphase stattfindet. Mit der Freigabe der Stakeholder können erste Ideen geplant und Lösungen evaluiert werden& M \\ \Xhline{1pt}
		5 &Bis zum 26.06.2020 muss die Konzeptionelle Planung abgeschlossen sein, um anschliessend mit der Realisierung zu beginnen & M \\ \Xhline{1pt}
		6 & Folgende Personen sind zu 100\% in das Projekt involviert:
		-  Oliver Egloff
		-	Leonardo Wiedemeier
		-	Samuel Salomon & M \\ \Xhline{1pt}
		7 &	Die angefallenen Kosten sind zu jedem Zeitpunkt aktuell zu halten und mit dem Soll-Zustand zu vergleichen. Dies erfordert ein ständiges Nachführen der geleisteten Stunden aller Projektmitarbeiter & M
		
		
	\end{tabularx}
	\caption{Prozessziele}
\end{table}

\subsubsection{Mehrwert Ziele}
\begin{table}[h]
	\centering
	
	\begin{tabularx}{\columnwidth}{l X r}
		\textbf{\thead{Nr}} & \textbf{\thead{Zielbeschreibung}} & \textbf{\thead{Priorität \\ (M, S, K)}}\\ \Xhline{2pt}
		1 & Know-How in der Entwicklung von Web Applikationen ausweiten & M \\ \Xhline{1pt}
		2 &	Projektmethoden richtig anwenden & M \\ \Xhline{1pt}
		3 &	Agile Methoden im GIT Umfeld sowie im Projektmanagement anwenden &	M \\ \Xhline{1pt}
		4 &	Know-How in Projektmitarbeit ausweiten & M \\ \Xhline{1pt}
		5 &	Umgang mit MQTT-Brokern sowie weiteren Web-Technologien wie HTML, CCS und Javascript verbessern & M \\ \Xhline{1pt}
		6 & Erlernen aktueller Web-Technologien & S \\ \Xhline{1pt}
		7 &	Erfahrungen mit Projektmethoden (Wasserfall und Agile) sammeln & M
	\end{tabularx}
	\caption{Mehrwert Ziele}
\end{table}
	\section{Lösungssuche}
	\begin{figure}[!hbt]
		\centering
		\includegraphics[width=\columnwidth]{01_Grafiken/loesungs_mind_map.png}
		\caption{Lösungs Mind Map}
		\label{fig:mind_map}
	\end{figure}
	
	\subsection{Vorstudie zur Erhebung der Programmiersprachen}
	\subsubsection{Backend}
	\paragraph{PHP}
	Eine weit verbreitet Open source Skriptsprache welche zur Laufzeit kompiliert wird. Sie zeichnet sich durch ihre Einfachheit, Geschwindigkeit und Stabilität aus. Es gibt eine umfangreiche Anzahl von Frameworks wie auch eine grosse Community, welche durch Open source ein gewisses Mitspracherecht an der Zukunft der Sprache hat. So können simple Kommandozeilen Scripts wie auch umfangreiche Webapplikationen erstellt werden.  
	\paragraph{Java}
	Java ist eine objektorientierte Programmiersprache, mit welcher sich plattformunabhängige Anwendungen entwickeln lassen können. Die Sprache hat sich gut durchgesetzt und ist weit verbreitet. Die Haupteigenschaften der Programmiersprache sind Sicherheit, Robustheit und Einfachheit. Java hat eine Vielzahl an Frameworks und eine grosse Community.
	\\ \\
	Beide Sprache bieten eine grosse Community und ein umfangreiches Toolset mit dem die Anforderungen an das System erfüllt werden können. Mit Hilfe einer Nutzwertanalyse wird nun entschieden mit welchen Varianten die Sprachen in der Lösungswahl berücksichtigt werden.
	
	\subsubsection{Frontend}
	\paragraph{Vue.js}
	Ist ein clientseitiges JavaScript-Webframework zum Erstellen von Single-Page-Webanwendungen nach dem MVVM-Muster, es kann allerdings auch in Multipage Webseiten für einzelne Abschnitte verwendet werden. Serverseitiges Rendern ist auch unterstützt. \\
	Vue.js ist ein "relativ" junges Framework, hat aber trotzdem eine wachsende Community und wird immer mehr verwendet. Wie zum Beispiel von GitLab.
	\paragraph{React}
	Gibt es bereits etwas länger als Vue.js und bietet ähnliche Funktionalitäten als Frontend Framework. Es wurde damals von Facebook ins leben gerufen und auch noch heute verwendet. Daher geniesst React eine grosse Community und ist weit verbreitet.
	\\ \\
	Beide Sprachen bieten ähnliche Funktionalitäten und Syntax. Vue.js gilt als flexiblere Sprache und daher für Neulinge besser zum einsteigen. React.js geniesst eine grosse Community und Facebook als Mitentwickler hinter der Sprache (ist aber wie Vue.js OpenSource). Auch hier soll durch eine Nutzwertanalyse 
	
	\subsection{Black Box}
	
	\subsubsection{Frontend}
	
	\begin{figure}[!hbt]
		\centering
		\includegraphics[width=\columnwidth]{01_Grafiken/black-box_Frotend.png}
		\caption{Black Box Frontend}
		\label{fig:bb_frontend}
	\end{figure}
	
	\subsubsection{Backend}

	\begin{figure}[!hbt]
		\centering
		\includegraphics[width=\columnwidth]{01_Grafiken/black-box_Backend.png}
		\caption{Black Box Backend}
		\label{fig:bb_backend}
	\end{figure}

	\newpage

	\subsection{Lösung //TODO needs new name}
	
	\begin{figure}[!hbt]
		\centering
		\includegraphics[width=\columnwidth]{01_Grafiken/architektur.png}
		\caption{Architektur}
		\label{fig:architektur}
	\end{figure}

	Es wurde eine Variante erstellt welche durch das Vorgängerprojekt, der LED-RGB-Matrix, den Rahmenbedingungen und dem Architekturvorschlag (Abbildung: \ref{fig:architektur}) definiert wurde.
	\\ \\
	Die LED-RGB-Matrix wurde als unabhängiger Service programmiert und kann daher ohne Anpassung in das laufende Projekt integriert werden. Es handelt sich um eine Rest-API, an welche Bilder mit gewünschten Transformationsparamtern gesendet und dargestellt werden können.
	\\ \\
	Das Tic-Tac-Toe Game (Spiellogik + Datenbank) wird das Backend bilden. Dies beinhaltet folgende Aufgaben:
	\begin{itemize}
		\item Ist für die gesamte Spiellogik zuständig
		\begin{itemize}
			\item Spielverlauf, Spielerinformation und Spielstand mit der Datenbank verwalten
			\item Assets verwalten
			\item Reagieren auf Zustandsänderungen des Frontends
		\end{itemize}
		\item Kommunizieren mit dem Frontend (über MQTT) und rendern der View
		\item Aufbereiten und senden des Spielfeldes an die LED-RGB-Matrix (REST)
	\end{itemize}
	
	Bei der Datenbank wird es sich um MYSQL handel, was vom Kunden so gewünscht wurde.
	\\
	Das Frontend wird über MQTT mit dem Backend kommunizieren und die interaktive Spieloberfläche sein über welche dann Tic-Tac-Toe gespielt werden kann. Sie reagiert dynamisch auf die Eingaben des Benutzers und passt die View an.
	\\
	\subsubsection{Varianten}
	Es wurde eine Lösung erarbeitet welche mit Verschiedenen Programmiersprachen umgesetzt werden kann.
	\paragraph{Frontend}
	Im Frontend kann jeweils mit Vue.js oder React.js umgesetzt werden.
	
	\paragraph{Backend}
	Im Backend besteht die Möglichkeit es mit Java oder PHP zum implementieren.

	\newpage
		
	\section{Lösungswahl}
	Die Lösungswahl wird aufgrund der Zielerreichung, Nutzwertanalyse und den weiteren Risiken getroffen.
	\subsection{Zielerreichung}
	In der Zielerreichung wird analysiert, ob die Lösungsvarianten die geforderten Systemziele erfüllen können. So werden ungeeignete Lösungen bereits in einer frühen Phase aus der Studie herausgefiltert.
	\\
	Es wurden jeweils nur die Umgebungsrelevanten Systemziele berücksichtigt.
	
	\subsubsection{Frontend}
	
	\begin{table}[h]
		\centering
		\rowcolors{1}{gray!30}{gray!10}
		\begin{tabular}{p{0.25\textwidth}|p{0.2\textwidth}|p{0.22\textwidth}|p{0.22\textwidth}}
			\textbf{Systemziel-Nr.} & \textbf{Gewicht (M,S,K)} & \textbf{Vue.js} & \textbf{React.js} \\ \hline
			1 & M & \cellcolor{green} Ja & \cellcolor{green} Ja \\ \hline
			2 & M & \cellcolor{green} Ja & \cellcolor{green} Ja \\ \hline
			5 & M & \cellcolor{green} Ja & \cellcolor{green} Ja \\ \hline
			7 & M & \cellcolor{green} Ja & \cellcolor{green} Ja \\ \hline
			8 & M & \cellcolor{green} Ja & \cellcolor{green} Ja \\ \hline
		\end{tabular}
		\caption{Zielerreichung des Frontend}
		\label{fig:zielerreichung_frontend}
	\end{table}

	
	\subsubsection{Backend}
	
	\begin{table}[h]
		\centering
		\rowcolors{1}{gray!30}{gray!10}
		\begin{tabular}{p{0.25\textwidth}|p{0.2\textwidth}|p{0.22\textwidth}|p{0.22\textwidth}}
			\textbf{Systemziel-Nr.} & \textbf{Gewicht (M,S,K)} & \textbf{PHP} & \textbf{Java} \\ \hline
			1 & M & \cellcolor{green} Ja & \cellcolor{green} Ja \\ \hline
			2 & M & \cellcolor{green} Ja & \cellcolor{green} Ja \\ \hline
			3 & M & \cellcolor{green} Ja & \cellcolor{green} Ja \\ \hline
			4 & S & \cellcolor{green} Ja & \cellcolor{green} Ja \\ \hline
			6 & M & \cellcolor{green} Ja & \cellcolor{green} Ja \\ \hline
			7 & M & \cellcolor{green} Ja & \cellcolor{green} Ja \\ \hline
			8 & M & \cellcolor{green} Ja & \cellcolor{green} Ja \\ \hline
		\end{tabular}
		\caption{Zielerreichung des Backend}
		\label{fig:zielerreichung_backend}
	\end{table}
	
	\newpage

	\begin{landscape}
	\subsection{Nutzwertanalyse}
	Die Gewichtung wird in Form einer Zahl zwischen 1-10 definiert, wobei die 1 die tiefste und 10 die höchste Gewichtung ist.
	
	 \subsubsection{Frontend}
	
	\begin{table}[h]
		\centering
		\begin{tabular}{p{0.4\textwidth}|p{0.25\textwidth}|p{0.2\textwidth} | p{0.2\textwidth} | p{0.2\textwidth} | p{0.2\textwidth}}
			\multirow{2}{*}{\textbf{Kriterien}} & \multirow{2}{*}{\textbf{Gewichtung}} & \multicolumn{2}{c}{\textbf{Variante 1: Vue.js}} & \multicolumn{2}{c}{\textbf{Variante 2: React.js}} \\
			& & \textbf{Punkte} & \textbf{Bewertung} & \textbf{Punkte} & \textbf{Bewertung}  \\ \hline
			Vorhandenes Know-How & 6 & 2 & 12 & 2 & 12  \\ \hline
			Geringer Entwicklungsaufwand & 7 & 8 & 56 & 6 & 42 \\ \hline
			Community / Dokumentation & 9 & 9 & 81 & 8 & 72 \\ \hline
			Verfügbare Libraries & 5 & 10 & 50 & 10 & 50 \\ \hline
			Zukunftssicher / Vortlaufende Entwicklung & 8 & 10 & 80 & 10 & 80 \\ \hline
			Erweiterbarkeit & 5 & 9 & 45 & 9 & 45 \\ \hline
			\textbf{Nutzwert} & & & \cellcolor{green} 324 & & \cellcolor{orange} 301 \\ \hline
		\end{tabular}
		\caption{Nutzwerkanalyse Frontend}
		\label{fig:nutzwerkanalyse_frontend}
	\end{table}

	Unsere Recherche wurde hauptsächlich von der folgenden Seite bezogen: \href{https://www.codica.com/blog/react-vs-vue-2019/#react-vs-vue-comparison-summary}{https://codica.com}

	\end{landscape}

	\newpage
	
	\begin{landscape}
		\subsubsection{Backend}
		
		\begin{table}[h]
			\centering
			\begin{tabular}{p{0.4\textwidth}|p{0.25\textwidth}|p{0.2\textwidth} | p{0.2\textwidth} | p{0.2\textwidth} | p{0.2\textwidth}}
				\multirow{2}{*}{\textbf{Kriterien}} & \multirow{2}{*}{\textbf{Gewichtung}} & \multicolumn{2}{c}{\textbf{Variante 1: PHP}} & \multicolumn{2}{c}{\textbf{Variante 2: Java}} \\
				& & \textbf{Punkte} & \textbf{Bewertung} & \textbf{Punkte} & \textbf{Bewertung}  \\ \hline
				Vorhandenes Know-How & 6 & 7 & 42 & 8 & 48  \\ \hline
				Geringer Entwicklungsaufwand & 7 & 7 & 49 & 9 & 63 \\ \hline
				Community / Dokumentation & 9 & 9 & 81 & 8 & 72 \\ \hline
				Verfügbare Libraries & 5 & 10 & 50 & 10 & 50 \\ \hline
				Zukunftssicher / Fortlaufende Entwicklung & 8 & 10 & 80 & 10 & 80 \\ \hline
				Erweiterbarkeit & 5 & 9 & 45 & 9 & 45 \\ \hline
				\textbf{Nutzwert} & & & \cellcolor{orange} 347 & & \cellcolor{green} 358 \\ \hline
			\end{tabular}
			\caption{Nutzwerkanalyse Backend}
			\label{fig:nutzwerkanalyse_backend}
		\end{table}
	\end{landscape}

	\newpage
	
	\subsection{Weitere Risiken}
	
	\subsubsection{Frontend}
	
	\begin{table}[h]
		\centering
		\rowcolors{1}{gray!30}{gray!10}
		\begin{tabular}{p{0.24\textwidth}|p{0.34\textwidth}|p{0.34\textwidth}}
			\textbf{Beurteilungskriterien} & \textbf{Vue.js} & \textbf{React.js} \\ \hline
			Risiken & Durch das geringe Know-How kann es zu Verspätungen in der Entwicklung kommen & Durch das geringe Know-How kann es zu Verspätungen in der Entwicklung kommen \\ \hline
			Nachhaltigkeit & OpenSource & OpenSource
		\end{tabular}
		\caption{Weitere Risiken im Frontend}
		\label{fig:weitere-risiken_frontend}
	\end{table}

	\subsubsection{Backend}

	\begin{table}[h]
		\centering
		\rowcolors{1}{gray!30}{gray!10}
		\begin{tabular}{p{0.24\textwidth}|p{0.34\textwidth}|p{0.34\textwidth}}
			\textbf{Beurteilungskriterien} & \textbf{PHP} & \textbf{Java} \\ \hline
			Risiken & Durch das geringe Know-How gewisser Mitglieder, kann es zu Verspätungen in der Entwicklung kommen & Durch das geringe Know-How gewisser Mitglieder, kann es zu Verspätungen in der Entwicklung kommen \\ \hline
			Nachhaltigkeit & OpenSource & OpenSource und Proprietäre möglich
		\end{tabular}
		\caption{Weitere Risiken im Backend}
		\label{fig:weitere-risiken_backend}
	\end{table}

	\subsection{Empfehlung}
	Das Projektteam hat bereits Know-How in JavaScrit wie auch teils in PHP und Java.
	Da in den Nutzwertanalysen klar heraus geht das für dass Frontend \textbf{Vue.js} und im Backend \textbf{Java} verwendet werden sollte, wird dies auch von uns Empfohlen.
	
	\section{Detailkonzept}
	\section{Realisierung}
	\newpage
	\section{Wirtschaftlichkeit und Risikoanalyse}

\subsection{Risikoanalyse}
	\subsubsection{Risiken}

\begin{table}[h]
		\centering
		\rowcolors{1}{gray!30}{gray!10}
		\begin{tabular}{p{0.02\textwidth}|p{0.45\textwidth}|p{0.45\textwidth}}
			\textbf{Id} & \textbf{Risiko Beschreibung} & \textbf{Möglicher Schade} \\ \hline

            R1 & Kosteneinschätzungen werden überschritten & Projekt wird teurer als ursprünglich geplant. Projektbudget kann nicht eingehalten werden. \\ \hline

            R2 & Mitarbeiterausfall & Projektziele können nicht eingehalten werden. Die anderen Mitarbeiter müssen vermehrt Überzeit machen. \\ \hline

            R3 & Hardwareausfall & Nach Ausfall muss ausgetauscht / repariert werden was zu Verschiebungen im Zeitplan führen kann \\ \hline

            R4 & Projektplan wurde zu sportlich geschätzt. & Meilensteine/Projektziele können nicht eingehalten werden, was zu Verschiebungen im Zeitplan führen kann. \\ \hline

            R5 & Naturkatastrophe & Verlust von Hardware, Code, Räumlichkeiten. Ausfall von Mitarbeitern. Es kann vorkommen das der Projektplan nicht eingehalten werden kann oder es zur nicht Durchführung des Projektes kommen. \\ \hline

            R6 & Fortschrittsverlust durch falsche Anwendung von Hilfsmitteln (z.B. falsche Versionierung von Dokumenten und Code) & Es geht sehr viel Zeit verloren, um die verlorene Arbeit wieder aufzuholen. Projektziele und Projektplan könnten damit gefährdet sein. \\ \hline

            R7 & COVID-19 & Die aktuelle Pandemie kann noch unbekannte Risiken und Einschränkungen mit sich ziehen. \\
		\end{tabular}

        \caption{Risiken}
        \label{fig:risiken}
    \end{table}

	Naturkatastrophen sind in der Schweiz ein kleines Risiko. Können sich aber trotzdem, z.B. Überschwemmungen oder Stürme, stark auf den Projektverlauf auswirken.

	Den Coronavirus (COVID-19) haben wir mit als Risiko aufgenommen da es die Aktuelle Lage stark beeinflusst. Wir können nicht vorhersagen welche Folgen der Coronavirus noch mit sich ziehen wird. Zum Beispiel wie stark die Wirtschaft einbrechen und den Fortschritt des Projekts in jeder Hinsicht beeinflussen wird. Oder jeder Beteiligte am Projekt Privat beeinflusst wird.

	\newpage

	\subsubsection{Massnahmen}
	\begin{table}[h]
		\centering
		\rowcolors{1}{gray!30}{gray!10}
		\begin{tabular}{p{0.03\textwidth}|p{0.3\textwidth}|p{0.2\textwidth}|p{0.1\textwidth}|p{0.2\textwidth}}
			\textbf{Nr.} & \textbf{Was?} & \textbf{Wer mit wem?} & \textbf{Wie viel?} & \textbf{Wann?} \\ \hline

            M1 & Regelmässige Kontrolle der Kostenfortschritte sowie die Einplanung von Reserven & Oliver Egloff, Leonardo Wiedemeier & & Immer zu Beginn einer neuen Phase \\ \hline

			M2 & Einfordern eines Ferienabwesenheitsantrages sowie die Planung aller Ressourcen. Zudem sollte ein regelmässiger Know-How Transfer stattfinden. & Oliver Egloff, Leonardo Wiedemeier & & Bei Ferienabwesenheit \\ \hline

			M3 & Hardware angemessen behandel und mit dem Kunden (ABBTS) im Falle eines Ausfalls sofort reagieren & Ganzes Team & & Während des ganzes Projektes \\ \hline

			M4 & Einplanen von zeitlichen Reserven & Ganzes Team & 1x pro Woche & Zu Beginn der Planung und am Anfang einer neuen Phase \\ \hline

			M6 & Regelmässige Backups der Lösung und klare Definierungen wie gearbeitet wird. & Samuel Salomon, Leonardo Wiedemeier & 1x pro Woche & Ab Beginn Projekt \\ \hline

			M7 & Über die Situation mit COVID-19 auf dem Laufenden bleiben und persönliche wie auch geschäftliche Entscheidungen frühzeitig angehen. & Alle am Projekt beteiligten Personen & Täglich & Täglich \\
		\end{tabular}
		\caption{Massnahmen}
		\label{fig:massnahmen}
	\end{table}

	Die meisten Massnahmen haben mit genauer Planung, Kommunikation und mit konzentrierten und genauen Arbeiten zu tun. Naturkatastrophen, Coronavirus und auch Teils Hardwareausfall können uns treffen, dürfen aber nicht überraschen. Mit den hier aufgelisteten Massnahmen gehen wir gegen alle möglichen Risiken gegen an und definieren wie im Falle eins Auftretens reagiert und sich darauf vorbereitet wird.

	\newpage

	\subsubsection{Risikomatrix}
	\begin{figure}[!hbt]
		\centering
		\includegraphics[width=\columnwidth]{risikomatrix.png}
		\caption{Risikomatrix}
		\label{fig:risiko_matrix}
	\end{figure}

	Die Risikomatrix zeigt auf wie, die von uns definierten und eingeschätzten, die Risiken (gelbe Kugeln) nach Anwendung der Massnahmen (Schwarze Pfeile) entschärft und neu eingestuft werden (blaue Kugeln).

	\newpage


\section{Abschluss}


\section{Abbildungsverzeichnis}
\listoffigures


\section{Tabellenverzeichnis}
\listoftables


\section{Selbstständigkeitserklärung}


\section{Anhang}
\newpage

\subsubsection{Dokumente}
%\caption{Pflichtenheft}
% commented out because of compiling speed
%\includepdf[pages=-, landscape=true, pagecommand={}]{../../05_Anhänge/Dokumente/01_Pflichtenheft/OEG‐LWI‐SSA_Pflichtenheft_0.1_Draft_4.pdf}

%\caption{Java Coding Guidelines}
% commented out because of compiling speed
% \includepdf[pages=-, landscape=true, pagecommand={}]{../../05_Anhänge/Dokumenten/01_Backend Java Guidelines/OEG‐LWI‐SSA_Backend_Coding_Guidelines_0.1_Draft_4.pdf}

\subsubsection{Präsentationen}
%\caption{Kick-Off Meeting vom 13.05.2020}
    % commented out because of compiling speed
    % \includepdf[pages=-, landscape=true, pagecommand={}]{../../05_Anhänge/01_Präsentationen/01_KickOff_Meeting/Gruppe_1_SA_KickOff_Meeting_VisualBoardGmbH.pdf}

\end{document}

