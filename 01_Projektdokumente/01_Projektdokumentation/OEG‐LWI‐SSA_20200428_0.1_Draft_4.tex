\documentclass[11pt, a4paper, twoside]{article}

\usepackage[table]{xcolor}
\usepackage{geometry} %Set page layouts
\usepackage{pdfpages} %Include pdf files
\usepackage{fancyhdr} %Header and Footer
\usepackage{graphicx} %Inlcude graphics
\usepackage{lastpage} %Reference the number of pages in your LATEX
\usepackage[german]{babel}
\usepackage{fontspec}
\usepackage{float}
\usepackage{hyperref}
\usepackage{tocloft}
\usepackage{tabularx}
\usepackage{import}
\usepackage{lscape}
\usepackage{multirow}
\usepackage{makecell}
\usepackage{pdflscape}
\usepackage{listings}
\usepackage{longtable}
% \usepackage{showframe} %For debugging

\hypersetup{
    colorlinks=true,
    linkcolor=black,
    filecolor=magenta,
    urlcolor=cyan,
}


%json color setup
\colorlet{punct}{red!60!black}
\definecolor{background}{HTML}{EEEEEE}
\definecolor{delim}{RGB}{20,105,176}
\colorlet{numb}{magenta!60!black}

\lstdefinelanguage{json}{
	basicstyle=\normalfont\ttfamily,
	numbers=left,
	numberstyle=\scriptsize,
	stepnumber=1,
	numbersep=8pt,
	showstringspaces=false,
	breaklines=true,
	frame=lines,
	backgroundcolor=\color{background},
	literate=
	*{0}{{{\color{numb}0}}}{1}
	{1}{{{\color{numb}1}}}{1}
	{2}{{{\color{numb}2}}}{1}
	{3}{{{\color{numb}3}}}{1}
	{4}{{{\color{numb}4}}}{1}
	{5}{{{\color{numb}5}}}{1}
	{6}{{{\color{numb}6}}}{1}
	{7}{{{\color{numb}7}}}{1}
	{8}{{{\color{numb}8}}}{1}
	{9}{{{\color{numb}9}}}{1}
	{:}{{{\color{punct}{:}}}}{1}
	{,}{{{\color{punct}{,}}}}{1}
	{\{}{{{\color{delim}{\{}}}}{1}
	{\}}{{{\color{delim}{\}}}}}{1}
	{[}{{{\color{delim}{[}}}}{1}
	{]}{{{\color{delim}{]}}}}{1},
}


\geometry{top=0.75cm, bottom=0.8cm, right=2.5cm, left=2.5cm, headheight=35pt, includeheadfoot, portrait}
\setmainfont{Calibri}

%Path relative to the .tex file containing the \includegraphics command
\graphicspath{{01_Grafiken/}}

\tocloftpagestyle{fancy}

%Define variables for document
\newcommand{\shortAuthor}{OEG-LWI-SSA}
\newcommand{\dstate}{Draft} %Draft, Progress, Closed
\newcommand{\dversion}{0.1}
\newcommand{\dname}{\shortAuthor\_20200428\_\dversion\_\dstate\_\pageref{LastPage}}
%\renewcommand\font{\normalsize}
\begin{document}


%Define header & footer for twoside documents
\fancyhead{} %Clear all header fields
\fancyhead[L]{\includegraphics{header_img_abb2.png}}
\fancyfoot{} %Clear all footer fields
\fancyfoot[RO, LE]{\small \thepage \textbar \pageref{LastPage}}
\fancyfoot[LO, RE]{\small \dname}
\renewcommand{\footrulewidth}{0.4pt}
\renewcommand{\headrulewidth}{0.4pt}
\pagestyle{fancy}

\begin{large}
	\begin{tabular}{p{5cm}l l }
		\textbf{Bildungsgang:} & Klasse B20-if4.1 Informatik \\
		\textbf{Fach:} & Datenbank \& Web Engineering \\
		\textbf{Semester:} & Semester 4 – 2019/2020 \\
		\textbf{Autoren:} & Oliver Egloff, Leonardo Wiedemeier, Samuel Salomon
	\end{tabular}
\end{large}


\vspace{1cm}
\hrule
\vspace{1cm}
\begin{center}
	\centering
	\LARGE{\textbf{Semesterarbeit – Visual Board GmbH}}
\end{center}


\vspace{20pt}


\begin{figure}[h]
	\centering
	\includegraphics{title.png}
\end{figure}
\begin{center}
	\centering
	\section*{Projektdokumentation}
\end{center}

\newpage


	\section{Dokumentenmanagement}

	\begin{tabularx}{\columnwidth}{l X}
		\textbf{Erstellungsdatum:} & 28.04.2020 \\
		\textbf{Autotren:} & Samuel Salomon (SSA), Oliver Egloff (OEG), Leonardo Wiedemeier (LWI) \\
		\textbf{Dateiname:} &\dname
	\end{tabularx}


\section{Änderungsverzeichnis}
	\begin{tabularx}{\columnwidth}{l l l X}
		\textbf{Version} & \textbf{Datum} & \textbf{Autor} & \textbf{Beschreibung}\\
		0.1 & 28.04.2020 & SSA & Dokumentation und Gliederung erstellt \\
		0.2 & 04.05.2020 & LWI & Einfügen der Risikoanalyse \\
		0.3 & 06.05.2020 & LWI & Einfügen der Situationsanalyse \\
		0.4 & 07.05.2020 & LWI & Einfügen der Rollenbeschreibung + kleine Design-Änderungen \\
		0.5 & 08.05.2020 & LWI & Einfügen der Ressourcenplanung \\
		0.6 & 11.05.2020 & LWI & Änderung R3/M3 in der Risikoanalyse \\
		0.7 & 12.05.2020 & LWI & Änderung in der Risikoanalyse + Grafik \\
		0.8 & 13.05.2020 & OEG & Einfügen des Kommunikationsplans \\
		0.9 & 15.05.2020 & OEG & Einfügen Anhang Kick-Off-Meeting \\
		0.10 & 19.05.2020 & LWI & Risikomatrix mit Texten erweitern \\
		0.11 & 06.06.2020 & LWI & Lösungssuche und Lösungswahl vorbereitet \\
		0.12 & 08.06.2020 & SSA, OEG, LWI & Mind Map für die Lösungssuche erstellt\\
		0.13 & 16.06.2020 & LWI & fertigstellen der Lösungssuche und Lösungswahl \\
		0.14 & 17.06.2020 & SSA & Zieldefinition \\
		0.15 & 23.06.2020 & LWI & Funktionendiagramm \\
		0.16 & 17.08.2020 & LWI & 9 funktionale Tests eingetragen \\
		0.17 & 19.08.2020 & LWI & Schnittstellen erfasst \\
		0.18 & 20.08.2020 & LWI & Klassendiagramm Sektion bearbeitet \\
		0.19 & 21.08.2020 & LWI & Ablaufdiagramm eingefügt + Verwendete Tools erstellt + Produktkonzept + Entwichkungsumgebung und Tool-Chain + Dokumentationstruktur ergänzt \\
		0.20 & 22.08.2020 & LWI & SWOT-Analyse + Mengengerüst + Produktkonzept + Datenstruktur \\
		0.21 & 23.08.2020 & LWI & Management Summary \\
		0.22 & 24.08.2020 & LWI & ERM hinzugefügt + Management Summary \\
        0.23 & 24.08.2020 & OEG & Anhänge angefügt \\
        0.24 & 25.08.2020 & OEG & Abschliessen der Projekdokumentation \\
        0.25 & 25.08.2020 & OEG & Letzte Korrekturen \\
        1.0 & 25.08.2020 & OEG, SSA, LWI & Abschluss und Druck des Dokumentes \\
    \end{tabularx}
	
	\renewcommand{\cftsecleader}{\cftdotfill{\cftdotsep}}

    \newpage

	\tableofcontents

\newpage

\section{Managment Summary}
\paragraph{Ausgangslage}
Die Visual Board GmbH ist ein Unternehmen, welches sich auf die Herstellung und Ansteuerung von visuellen Panels für Anzeigen aller Art spezialisiert.
Durch die aktuelle Lage mit der COVID-19-Pandemie fängt die Firma an sich auch auf Webaplikationen zu spezialisieren, welche es möglich machen, die Panels auf beliebige Arten anzusteuern.
Im vorgängigen Projekt wurde bereits eine Machbarkeitsstudie durchgeführt, ob man in Zukunft auf die Technologie von LED-RGB-Matrizen setzen kann.
Darauf aufbauend soll ein Tic-Tac-Toe Spiel umgesetzt werden, welches auf den Matrizen wiedergeben werden kann.

\paragraph{Situation}
Die Aufgabe ist es ein Multiplayer Tic-Tac-Toe Spiel zu entwickeln, bei dem es die Möglichkeit gibt, laufende Spiele auf beliebigen LED-RGB-Matrizen anzeigen zu lassen.\\
Es soll ein Frontend erstellt werden, welches dem User alle nötigen Funktionalitäten zur Verfügung stellt, um ein Tic-Tac-Toe Spiel gegen eine andere Person abzuwickeln.
Dieses soll dann via MQTT-Protokoll mit einem GameServer kommunizieren, welcher die komplette Business Logik beinhaltet.
Der GameServer sollte zudem die nötigen Infos an alle beteiligten Matrizen senden, damit die laufenden Spiele dort visuell dargestellt werden können.

\paragraph{Herausforderungen}
Das Projekt wird einige Herausforderungen an das gesamte Projektteam stellen.\\
Das orchestrieren der Kommunikation zwischen dem Frontend, GameServer und der LED-RGB-Matrix,
das designen des Frontendes und des GamerServers, so dass alle Anforderungen erfüllt werden können,
wie auch die Implementierung von Schnittstellen mit MQTT werden die grössten Herausforderungen darstellen.

\paragraph{Ziele}
Die Visual Board GmbH entwickelt einen GamerServer mit separatem Frontend, welches den Anforderungen für das Multiplayer Tic-Tac-Toe-Spiel entspricht.
Zudem soll es möglich sein, das laufende Spiel auf einer bis mehreren Matrizen darstellen zu lassen. Dies ist aber kein Muss und kann auch lediglich im Web stattfinden.

\paragraph{Lösung/Optionen}
Unsere Lösung baut im Backend auf die Programmiersprache Java in Kombination mit einer RestAPI und MQTT Topics auf.
Hierzu wird eine MYSQL-Datenbank für das Backend und ein Mosquitto-MQTT-Server für die Kommunikation mit dem Frontend aufgesetzt.
Das Frontend wurde seinerseits mit Bootstrapp, JavScript und dem Vue.js Framework umgesetzt. \\
Die API und das Datenbankhandling wurden mit Spring Boot umgesetzt und das MQTT-Protokoll mithilfe der MQTT v3 Client API Java Bibliothek implementiert.
Durch das Vorgänger Projekt musste an der LED-RGB-Matrix nichts verändert werden und kann von unserem GameServer normal angesprochen werden.\\

Es bestand noch die Möglichkeit das Projekt mit PHP im Backend und im Frontend mit dem React Framework umzusetzen.
Es wurde sich aber gegen diese Variante entschieden, da in den Nutzwertanalysen hervorging, dass das Umsetzungsteam in Java mehr Kompetenzen besitzt als in PHP. Für Vue.js wurde sich entschieden, da hier der Entwicklungsaufwand kleiner ausfällt und die Dokumentation sich von React abhebt.

\newpage

\paragraph{Nutzen \& Konsequenzen}
Durch den getrennten und unabhängigen Aufbau der einzelnen Applikationen ist die Lösung sehr modular.
Das Tic-Tac-Toe-Spiel kann mit und ohne Matrix durchgeführt werden.
Auch kann das Frontend sowie der GameServer unabhängig von dem anderen zwei Services einfach ausgetauscht werden.\\
\begin{figure}[!hbt]
	\centering
	\includegraphics[width=\columnwidth]{01_Grafiken/architektur_konzept.png}
	\caption{Architektur}
	\label{fig:architektur_konzept_management_summary}
\end{figure}
Durch den Einsatz von MQTT waren einige Punkte erschwert umzusetzen, da das Protokoll für diese Use-Cases nicht wirklich geeignet ist.
Zum Beispiel wurde entschieden, das Registrieren und den Login von User über HTTP Requests zu lösen. Wie auch das Abholen user-spezifischer Informationen.\\
Ein hohes Risiko war COVID-19: der daraus resultierende Mitarbeiterausfall hätte die Einhaltung des Abgabetermins erschwert. \\ \\
Das Produkt funktioniert und erfüllt alle Anforderungen, wird aber keine grosse Chance auf dem Markt haben. Es ist eine gute Demo, zum Zeigen, was mit den LED-RGB-Matrizen möglich ist und wie man weitere Applikationen entwickeln könnte, um diese vielseitig anzusteuern. Auch zeigt es auf, wie einfach eine LED-RGB-Matrix an ein bestehendes Projekt angebunden werden kann.\\

\newpage

\section{Aufgabenstellung}
\includepdf[pages=-, landscape=false, pagecommand={}, scale=0.88]{../02_Dokumente/DO_2020_SS19-20_if_Semesterarbeit_WEG_DBN_V1_0.pdf}

\newpage

\section{Projektmanagement}
    \subsection{Projektorganisation}
    \begin{figure}[!hbt]
        \centering
        \includegraphics[width=\columnwidth]{Projektorganisation.png}
        \caption{Projektorganisation}
        \label{fig:projektorganisation}
    \end{figure}

	\newpage
   
    \begin{landscape}
        \subsection{Meeting- und Kommunikationsplan}
        \documentclass[11pt, a4paper, twoside]{article}

\usepackage{geometry} %Set page layouts
\usepackage{pdfpages} %Include pdf files
\usepackage{fancyhdr} %Header and Footer
\usepackage{graphicx} %Inlcude graphics
\usepackage{lastpage} %Reference the number of pages in your LATEX
\usepackage[german]{babel}
\usepackage{fontspec}
\usepackage{float}
\usepackage{hyperref}
\usepackage{tocloft}
%\usepackage{showframe} %For debugging

\hypersetup{
    colorlinks=true,
    linkcolor=black,
    filecolor=magenta,
    urlcolor=cyan,
}

\geometry{top=0.75cm, bottom=0.8cm, right=2.5cm, left=2.5cm, headheight=35pt, includeheadfoot, landscape}
\setmainfont{Calibri}

%Path relative to the .tex file containing the \includegraphics command
\graphicspath{ {01_Grafiken/}}

%Set the pagestyle of content tables from plain to fancy
\tocloftpagestyle{fancy}

%Define variables for document
\newcommand{\shortAuthor}{OEG-LWI-SSA}
\newcommand{\dstate}{Draft} %Draft, Progress, Closed
\newcommand{\dversion}{0.1}
\newcommand{\dname}{\shortAuthor\_YYYYMMDD\_\dversion\_\dstate\_\pageref{LastPage}}

\begin{document}
    %Define header & footer for twoside documents
    \fancyhead{} %Clear all header fields
    \fancyhead[L]{\includegraphics{header_img_abb2.png}}
    \fancyfoot{} %Clear all footer fields
    \fancyfoot[RO, LE]{\small \thepage \textbar \pageref{LastPage}}
    \fancyfoot[LO, RE]{\small \dname}
    \renewcommand{\footrulewidth}{0.4pt}
    \renewcommand{\headrulewidth}{0.4pt}

    %Define header & footer for oneside documents
    %\fancyhead{} %Clear all header fields
    %\fancyhead[L]{\includegraphics{01_Grafiken/header_img_abb2.png}}
    %\fancyfoot{}
    %\fancyfoot[R]{\small \thepage \textbar \pageref{LastPage}}
    %\fancyfoot[L]{\dname}
    %\renewcommand{\footrulewidth}{0.4pt}
    %\renewcommand{\headrulewidth}{0.4pt}
    \pagestyle{fancy}

    \begin{table}[h]
        \centering
        \rowcolors{1}{gray!30}{gray!10}
        \begin{tabular}{p{0.02\textwidth}|p{0.45\textwidth}|p{0.45\textwidth}}
            \textbf{Id} & \textbf{Risiko Beschreibung} & \textbf{Möglicher Schade} \\ \hline

            R1 & Kosteneinschätzungen werden überschritten & Projekt wird teurer als ursprünglich geplant. Projektziele können nicht eingehalten werden. \\ \hline

            R2 & Mitarbeiterausfall & Projektziele können nicht eingehalten werden. Die anderen Mitarbeiter müssen vermehrt Überzeit machen. \\ \hline

            R3 & Probleme mit Hostinsg / Hardware & Unstimmigkeiten oder Probleme vom Hosting Partner können zu Verschiebungen im Zeitplan oder Mehrkosten führen. \\ \hline

            R4 & Projektplan wurde zu sportlich geschätzt. & Verlust von Hardware, Code, Räumlichkeiten. Ausfall von Mitarbeitern. Es kann vorkommen das der Projektplan nicht eingehalten werden kann oder es zur nicht Durchführung des Projektes kommen. \\ \hline

            R5 & Naturkatastrophe & Verlust von Hardware, Code, Räumlichkeiten. Ausfall von Mitarbeitern. Es kann vorkommen das der Projektplan nicht eingehalten werden kann oder es zur nicht Durchführung des Projektes kommen. \\ \hline

            R6 & Fortschrittsverlust durch falsche Anwendung von Hilfsmitteln (z.B. falsche Versionierung von Dokumenten und Code) & Es geht sehr viel Zeit verloren, um die verlorene Arbeit wieder aufzuholen. Projektziele und Projektplan könnten damit gefährdet sein. \\ \hline

            R7 & COVID-19 & Die aktuelle Pandemie kann noch unbekannte Risiken und Einschränkungen mit sich ziehen. \\
        \end{tabular}

        \caption{Risiken}
        \label{fig:risiken}
    \end{table}

\end{document}




    \end{landscape}
    \begin{figure}[!hbt]
        \centering
        \includegraphics[width=\columnwidth]{02_Dokumente/02_Kommunikationsplan/01_Grafiken/Kommunikations- und Dokumentationssytem.png}
        \caption{Kommunikationswege}
        \label{fig:kommunikations_wege}
    \end{figure}

    \newpage

    \subsubsection{Rollenbeschreibung}
    \begin{table}[h]
        \centering
        \rowcolors{1}{gray!30}{gray!10}
        \begin{tabular}{p{0.3\textwidth}|p{0.65\textwidth}}
            \textbf{Kürzel} & \textbf{Beschreibung} \\ \hline
            Auftraggeber & Er ist der Kunde und liefert das Lastenheft. \\ \hline

            Projektsteuerungs-Gremium & Trifft wichtige Entscheidungen und ist bei Problemen umgehend zu informieren. \\ \hline

            Projektportfolio-Controller & Hat alle laufenden Projekte im Überblick und ist Ansprechpartner für den Projektleiter und die Projektträger. \\ \hline

            Projektleiter & Die führende Kraft im Projektteam. Ist Ansprechpartner für die Involvierten und Projektträger. \\ \hline

            Projektmitarbeiter & Alle Betroffenen und Involvierten Personen/Organisation, die am Projekt mitarbeiten. \\ \hline

            Frontend Entwicklung & Hauptverantwortlicher des Frontend. \\ \hline

            Backend Entwicklung & Hauptverantwortlicher des Backend. \\ \hline

            Entwicklungsleitung & Hat Übersicht des Zusammenspiels von Front- und Backend. \\ \hline

            Software \& System Architekt & Übersicht der eingesetzten Tools und des Systemaufbaus. \\ \hline

            Datenbank-Manager \& Architekt & Hauptverantwortlicher der Datenbank. \\ \hline

            Dokumentent-Verantwortlicher & Ist verantwortlich für den Aufbau und die Struktur der Dokumente. \\ \hline

            Requirements \& Business Engineering & Hat die Übersicht über alle Anforderungen und ihrer Einhaltung. \\ \hline

            Quality \& Testing & Verantwortlich für die Qualität des Codes und dass er getestet ist. \\
        \end{tabular}
        \caption{Rollenbeschreibung}
        \label{fig:rollenbeschreibung}
    \end{table}

	\newpage

	\subsubsection{Funktionendiagramm}
	\begin{center}
		\rowcolors{1}{gray!30}{gray!10}
		\begin{longtable}{|p{0.1\textwidth}|p{0.34\textwidth}|p{0.1\textwidth}|p{0.1\textwidth}|p{0.1\textwidth}|p{0.1\textwidth}|}
			\hline \textbf{Nr.} & \textbf{Leistung} & \textbf{ABBTS} & \textbf{SSA} & \textbf{OEG} & \textbf{LWI} \\ \hline
			\textbf{L1} & \textbf{Projektanalyse} & & & &  \\ \hline
			AP1.1 & Situationsanalyse & I & M & M & D  \\ \hline
			AP1.2 & Anforderungskatalog  & I & D & M & M  \\ \hline
			AP1.3 & Ziele definieren & I & D & M & M \\ \hline
			AP1.4 & Risikoanalyse & I & M & M & D  \\ \hline
			
			\textbf{L2} & \textbf{Projektplanung}  & & & &  \\ \hline
			AP2.1 & Funktionendiagramm  & I & M & M & D  \\ \hline
			AP2.3 & Kommunikationsplan  & I & M & D & M  \\ \hline
			AP2.2 & Rollenbeschreibung & I & M & M & D  \\ \hline
			AP2.4 & Projektorganisation & I & M & D & M  \\ \hline
			AP2.5 & Projektrollen definieren & I & M & M & D \\ \hline
			AP2.6 & Projektstrukturplan & I & D & D & D  \\ \hline
			AP2.7 & Projektzeitplan & I & D & M & D \\ \hline
			AP2.8 & Kostenplanung & I & M & D & M  \\ \hline
			AP2.9 & Ressourcenplanung & I & D & D & D \\ \hline
			
			\textbf{L3} & \textbf{Lösungsevaluation} & & & &  \\ \hline
			AP3.1 & Lösungssuche & I & D & D & D  \\ \hline
			AP3.2 & Lösungsevaluation & I & D & D & D  \\ \hline
			AP3.3 & Lösungsentscheid & I & D & D & D \\ \hline
			
			\textbf{L4} & \textbf{Pflichtenheft} & & & &  \\ \hline
			AP4.1 & Chancen \& Risiken analysieren & I & M & M & D \\ \hline
			AP4.2 & Massnahmen definieren & I & M & M & D \\ \hline
			AP4.3 & Technische Umsetzung planen & I & D & D & D  \\ \hline
			AP4.4 & Vorbereitung Abnahme Pflichtenheft & I & M & D & D \\ \hline
			AP4.5 & Pflichtenheft Dokument erstellen & I & M & D & M \\ \hline
			AP4.6 & Mockups für das Pflichtenheft erstellen & I & D & M & M \\ \hline
			
			\textbf{L5} & \textbf{Kommunikationsschnittstellen definieren (Backend)} & & & & \\ \hline
			AP5.1 & Datenstrukturen definieren & I & D & D & D \\ \hline
			AP5.2 & Datenbankanbindung & I & D & D & D \\ \hline
			AP5.3 & Kommunikation mit LED-RGB-Matrix & I & D & M & M \\ \hline
			AP5.4 & MQTT (Client) Kommunikation & I & D & D & D \\ \hline
			
			\textbf{L6} & \textbf{Spiellogik definieren (Backend)} & & & & \\ \hline
			AP6.1 & Datenbank Schema & I & M & M & D \\ \hline
			AP6.2 & Businesslogik (Spiellogik) & I & D & D & D \\ \hline
			
			\textbf{L7} & \textbf{Design \& Logik definieren (Frontend)} & & & & \\ \hline
			AP7.1 & Frontend nach Mockups erstellen & I & D & D & D \\ \hline
			AP7.2 & Spiellogik in Frontend einbinden & I & D & D & D \\ \hline
			
			\textbf{L8} & \textbf{Kommunikationsschnittstellen definieren (Frontend)} & & & & \\ \hline
			AP8.1 & Kommunikation mit Backend (MQTT) & I & D & D & D \\ \hline
			
			\textbf{L9} & \textbf{Projektdokumentation} & & & & \\ \hline
			AP9.1 & Klassendiagramm & I & M & D & M \\ \hline
			AP9.1 & Lösungsdokumentation & I & D & D & D \\ \hline
			\rowcolor{white} \caption{Funktionsdiagramm}
			\label{fig:funktionsdiagramm}
		\end{longtable}
	\end{center}
	
	\begin{table}[h]
		\centering
		\begin{tabular}{|p{0.46\textwidth} p{0.46\textwidth}|}
			\hline Legende & \\
			E = Entscheiden & K = Koordination, Veranlassung  \\
			I = Information & D = Durchführung, Verantwortung  \\
			M = Mitarbeiter, Beratung &  \\ \hline
		\end{tabular}
	\end{table}

	\newpage
	
	\subsection{Projektstrukturplan}
	Der komplette Projektstrukturplan ist im Anhang des Dokuments zu finden.
	
	TODO in anhang einfügen


	\subsection{Projektplan}
	Der komplette Projektplan ist im Anhang des Dokuments zu finden.
	
	TODO projektplan im anhang

	\newpage

	\subsection{Ressourcenplanung}
	\begin{table}[h]
		\centering
		\rowcolors{1}{gray!30}{gray!10}
		\begin{tabular}{p{0.17\textwidth}|p{0.1\textwidth}|p{0.1\textwidth}|p{0.17\textwidth}|p{0.15\textwidth}|p{0.1\textwidth}}
			\textbf{Ressourcenname} & \textbf{Typ} & \textbf{Initials} &\textbf{Rolle}&\textbf{Stundensatz}&\textbf{Kosten pro Stunde} \\ \hline

            Oliver Egloff & Person / Gruppe & OEG & Projektmanagement & CHF 100.00/h & CHF 0.00 \\\hline

            Leonardo Wiedemeier & Person / Gruppe & LWI & Projektmitarbeiter & CHF 100.00/h & CHF 0.00 \\\hline

            Samuel Salomon & Person / Gruppe & SSA & Projektmitarbeiter & CHF 100.00/h & CHF 0.00 \\\hline

            Stakeholder & Person / Gruppe & S & Gremium & CHF 0.00 & CHF 0.00 \\
		\end{tabular}
		\caption{Stundensatz}
		\label{fig:stundensatz}
	\end{table}

 	 TODO aktueller aufwand aller beteiligten als tabelle
 	 
 	 \newpage
 	 
 	 \subsection{Kostenplanung}

	\newpage
	
\subsection{Qualitätsmanagement}
\subsubsection{Anforderungs- und risikobasierte Qualitätsziele}
\begin{table}[h]
	\centering
	\rowcolors{1}{gray!30}{gray!10}
	\begin{tabular}{p{0.25\textwidth}|p{0.2\textwidth}|p{0.3\textwidth}|p{0.15\textwidth}}
		\textbf{Erfolgsfaktor} & \textbf{Projektanforderung} & \textbf{Risikobeurteilung} &\textbf{Qualitätsschwerpunkt} \\ \hline
		Erfüllung, Leistung und Funktionalität & Sehr wichtig & Beachtliches Risiko & Ja \\ \hline
		Akzeptanz & Wichtig & Beachtliches Risiko & Ja \\ \hline
		Wirtschaftlichkeit & Weniger wichtig & Akzeptables Risiko & Nein \\ \hline
		Kosteneinhaltung & Sehr Wichtig & Beachtliches Risiko & Ja \\ \hline
		Termintreue & Sehr wichtig & Akzeptables Risiko & Ja \\ \hline
	\end{tabular}
	\caption{Anforderungs- und risikobasierte Qualitätsziele}
	\label{fig:qualitaetsziele}
\end{table}

\subsubsection{Massnahmen}
\begin{table}[h]
	\centering
	\rowcolors{1}{gray!30}{gray!10}
	\begin{tabular}{p{0.18\textwidth}|p{0.25\textwidth}|p{0.2\textwidth}|p{0.1\textwidth}|p{0.15\textwidth}}
		\textbf{Schwerpunkt} & \textbf{Was} & \textbf{Wer mit Wem} &\textbf{Wieviele} & \textbf{Termin} \\ \hline
		Erfüllung, Leistung und Funktionalität & Regelmässige Prüfung der Arbeitspakete und Anforderungserfüllung & Oliver Egloff, Leonardo Wiedemeier & & Jede zweite Woche am Freitag \\ \hline
		Kosteneinhaltung & Prüfen der Aufwände für die geleisteten Arbeiten und somit der Kosten & Oliver Egloff & & Jede zweite Woche am Freitag \\ \hline
		Termintreue & Durchführen von Status-Meetings, damit der Arbeitsfortschritt geprüft werden kann
		- Definieren von genauen Fristen für die Arbeitspakete
		& Oliver Egloff, Samuel Salomon, Leonardo Wiedemeier & & Jede Woche Montags \\ \hline
	\end{tabular}
	\caption{Massnahmen}
	\label{fig:qualitaets_massnahmen}
\end{table}
	
	\newpage
	
	//TODO oli/liw/samuel
	\subsection{Projektcontrolling}
	Die Überprüfung der Kennzahlen des Projekts wurden bei den Meilensteinen M02, M03 sowie den Lieferobjekten L6 und L7 durchgeführt. Es wurden jeweils die folgenden Kennzahlen geprüft:
	\begin{itemize}
		\item Kosten
		\item Ressourcen / Aufwand
		\item Termin / Tage
		\item Qualität
		\item Leistung
	\end{itemize}
	
	Nachfolgend werden die Ergebnisse des Projektcontrollings pro Meilenstein festgehalten. Die detaillierten Analysen sind im Anhang ersichtlich.
	
	\begin{table}[h]
		\centering
		\rowcolors{1}{gray!30}{gray!10}
		\begin{tabular}{p{0.2\textwidth}|p{0.4\textwidth}|p{0.39\textwidth}}
			\textbf{Meilenstein} & \textbf{Ergebnis} & \textbf{Massnahme} \\ \hline
			M02 Kick off & Die Projektplanung ist abgeschlossen und die Anforderungen sind erfasst. In einem nächsten Schritt müssen die Anforderungen geplant werden. Es ist bereits mit einem Mehraufwand bei den Lieferobjekten zu rechnen. & Die Aufwände für die Lieferobjekte müssen genau definiert und in regelmässigen Abständen kontrolliert werden. \\ \hline
			M03 Abnahme Pflichtenheft & Durch die Detailplanung der Realisierung konnte festgestellt werden, dass aufgrund von fehlendem Know-How die Arbeitspakete der Realisierungsphase falsch geplant wurden. Die Ressourcen sowie die Kosten müssen aufgrund dieser Kenntnisse verändert werden. Es ist mit einem Anstieg der Projektkosten und Aufwände zu rechnen. & \\ \hline
		\end{tabular}
		\caption{Projektcontrolling}
		\label{fig:projektcontrolling}
	\end{table}
	
	\subsection{Meetingsprotokolle}
Die Meetingsprotokolle befinden sich im Anhang an dieses Dokument.

	\subsection{Arbeitsjournal}
Die Arbeitsjournale des Projektteams befinden sich im Anhang an dieses Dokument.

\newpage

\section{Situationsanalyse}
\subsection{Vorgänger-Projekt}
 Die Visual Board GmbH hat bereits mit der ABBTS (Rolf Hirschi) zusammengearbeitet und
einen Prototypen erstellt, um Bilder auf einer RGB-LED-Matrix anzuzeigen.
Das vorliegende Projekt baut auf dieses Vorgänger-Projekt (aus der Semesterarbeit des 3. Semesters) auf.
Die RGB-LED-Matrix wurde funktionsfähig gestellt und es sollte ein Programm geschrieben werden, um Bilder
wie gewünscht auf der Matrix anzuzeigen. Darunter ist das korrekte Transformieren, Aufbereiten
und Weitergeben an die Matrix zu verstehen.
Der Prototyp wurde von der Visual Board GmbH mit einer REST-Api in Java umgesetzt,
welche auf dem Raspberry Pi der Hardware oder einem externen Windows Computer läuft.
So wurde die RGB-LED-Matrix dem Netzwerk, in welchem sie sich befindet, zur Verfügung gestellt.
Es ist möglich mit den korrekten Requests, Bilder direkt oder via Webinterface
an die Matrix zu schicken.
Der Visual Board GmbH wurde die komplette Hardware durch die ABBTS zu Verfügung gestellt.
\newline
Teil der Projektorganisiton waren:
\begin{itemize}
	\item ABBTS (Rolf Hirschi) als Auftraggeber
	\item Visual Board GmbH als Arbeitnehmer
	\subitem Oliver Egloff hat die Position als Projektleiter eingenommen
	\subitem Leonardo Wiedemeier hat die Position als stellvertretender Projektleiter eingenommen
	\subitem Samuel Salomon hat die Position als Projektmitglied eingenommen
	\item Herr Jenzer jeweils als Mitglied des PSG (Projektsteuerungs-Gremium)
\end{itemize}

Das Endprodukt aus der Semesterarbeit des dritten Semesters, die RGB-LED-Matrix, gilt als Grundbaustein für dieses Projekt.
Das Produkt wird um ein Tic-Tac-Toe Spiel ausgebaut.

\subsection{Weiteres Vorgehen}
Die ABBTS hat sich entschieden auf den Prototypen der Visual Board GmbH aufzubauen und die Spiellogik für ein Tic-Tac-Toe Spiel umzusetzen.
Es soll die Möglichkeit bestehen, dass mehrere Spieler auf verschiedenen Endgeräten gegeneinander spielen, während die RGB-LED-Matrix das aktuelle Spielfeld ausgibt. Hierfür werden wir das komplette Vorgänger-Projekt verwenden, da die Matrix bereits einsatzbereit ist. Es werden allenfalls Schnittstellen angepasst oder neue erstellt.
Zusätzlich zur Implementierung der Spiellogik soll diese auch auf einen Webserver laufen und so erreichbar gemacht werden.

Die Visual Board GmbH wird daher im Rahmen des Projektes die Spiellogik mit Benutzeroberfläche implementieren und den nötigen Webserver einrichten. Für diese Arbeiten stehen 300 Arbeitsstunden zur kompletten Projektabwicklung zur Verfügung.

\newpage

\section{Zielsetzung}
Das Ziel dieses Projektes ist, ein webbasiertes Tic-Tac-Toe-Spiel zu realisieren. Bereits bestehende Erfahrungen sowie Ergebnisse des RGB-LED-Matrix-Prototypen werden wiederverwendet, um z.B. den Spielstand auf der RGB-LED-Matrix darzustellen. Ausgehend vom Projektziel werden Ziele untenstehend strukturiert nach
\begin{itemize}
	\item Systemzielen
	\item Prozesszielen
	\item Mehrwert-Zielen
\end{itemize}

aufgelistet. Sämtliche Ziele sind nach den SMART-Kriterien definiert:
\begin{itemize}
	
	\item Spezifisch
	\item Messbar
	\item Akzeptiert
	\item Realistisch
	\item Terminiert
\end{itemize}
\subsection{Zieldefinition}
Die Ziele werden in System-, Prozess- und Mehrwert-Ziele unterteilt. Die Wichtigkeit der Ziele wird anhand der Priorität eingestuft: Muss (M), Soll (S) und Kann (K)
    \subsubsection{Systemziele}
\begin{table}[h]
	\centering
	\begin{tabularx}{\columnwidth}{l X r}
		\textbf{\thead{Nr}} & \textbf{\thead{Zielbeschreibung}} & \textbf{\thead{Priorität \\ (M, S, K)}}\\ \Xhline{2pt}
		1 & Das Tic-Tac-Toe-Spiel muss multiplayerfähig sein. & M \\ \Xhline{1pt}
		2 & Spielhistorien müssen vom Benutzer abgerufen werden können. & M \\ \Xhline{1pt}
		3 & Aktuelle Spielzüge müssen auf der RGB-LED-Matrix dargestellt werden. & M \\ \Xhline{1pt}
		4 & Objekte sollen in universellen Formaten gespeichert werden, um sie für spätere Projekte wiederzuverwenden. & S \\ \Xhline{1pt}
		5 & Dem Benutzer muss ein Web-GUI zur Verfügung gestellt werden. & M \\ \Xhline{1pt}
		6 & Vom Benutzer erzeugte Daten müssen in einer zentralen MySQL-Datenbank abgelegt werden. & M \\ \Xhline{1pt}
		7 & Der Benutzer muss sich im System anmelden können. & M \\ \Xhline{1pt}
		8 & Die allgemeinen Spielregeln des offiziellen Tic-Tac-Toe-Spiels müssen abgebildet werden. & M \\ \Xhline{1pt}
	\end{tabularx}
	\caption{Systemziele}
\end{table}


\subsubsection{Prozessziele}
\begin{table}[h]
	\centering
	\begin{tabularx}{\columnwidth}{l X r}
		\textbf{\thead{Nr}} & \textbf{\thead{Zielbeschreibung}} & \textbf{\thead{Priorität \\ (M, S, K)}}\\ \Xhline{2pt}
		1 & Das System muss bis zum 26.08.2020 fertiggestellt werden. & M \\ \Xhline{1pt}
		2 & Der Prototyp soll am 26.08.2020 den Stakeholdern präsentiert werden. & M \\ \Xhline{1pt}
		3 & Das Projekt wird nach dem Wasserfallmodell unter Berücksichtigung agiler Projektmethoden für die Entwicklung umgesetzt. & M \\ \Xhline{1pt}
		4 & Am 13.05.2020 findet das Kick-off-Meeting statt, in welchem die Initialisierungsphase besprochen wird und der Übergang in die Konzeptphase stattfindet. Mit der Freigabe der Stakeholder können erste Ideen geplant und Lösungen evaluiert werden. & M \\ \Xhline{1pt}
		5 &Bis zum 26.06.2020 muss die konzeptionelle Planung abgeschlossen sein, um anschliessend mit der Realisierung zu beginnen. & M \\ \Xhline{1pt}
		6 & Folgende Personen sind zu 100\% in das Projekt involviert: Oliver Egloff, Leonardo Wiedemeier, Samuel Salomon & M \\ \Xhline{1pt}
		7 &	Die angefallenen Kosten sind zu jedem Zeitpunkt aktuell zu halten und mit dem Soll-Zustand zu vergleichen. Dies erfordert ein ständiges Nachführen der geleisteten Stunden aller Projektmitarbeiter. & M
		
	\end{tabularx}
	\caption{Prozessziele}
\end{table}

\newpage

\subsubsection{Mehrwert-Ziele}
\begin{table}[h]
	\centering
	
	\begin{tabularx}{\columnwidth}{l X r}
		\textbf{\thead{Nr}} & \textbf{\thead{Zielbeschreibung}} & \textbf{\thead{Priorität \\ (M, S, K)}}\\ \Xhline{2pt}
		1 & Know-How in der Entwicklung von Web Applikationen ausweiten & M \\ \Xhline{1pt}
		2 &	Projektmethoden richtig anwenden & M \\ \Xhline{1pt}
		3 &	Agile Methoden im GIT Umfeld sowie im Projektmanagement anwenden &	M \\ \Xhline{1pt}
		4 &	Know-How in Projektmitarbeit ausweiten & M \\ \Xhline{1pt}
		5 &	Umgang mit MQTT-Brokern sowie weiteren Web-Technologien wie HTML, CCS und Javascript verbessern & M \\ \Xhline{1pt}
		6 & Erlernen aktueller Web-Technologien & S \\ \Xhline{1pt}
		7 &	Erfahrungen mit Projektmethoden (Wasserfall und Agile) sammeln & M
	\end{tabularx}
	\caption{Mehrwert-Ziele}
\end{table}

\newpage

	\section{Lösungssuche}
	\begin{figure}[!hbt]
		\centering
		\includegraphics[width=\columnwidth]{01_Grafiken/loesungs_mind_map.png}
		\caption{Lösungs-Mind-Map}
		\label{fig:mind_map}
	\end{figure}
	
	\subsection{Vorstudie zur Erhebung der Programmiersprachen}
	\subsubsection{Backend}
	\paragraph{PHP}
	Eine weit verbreitet open-source-Skriptsprache, welche zur Laufzeit kompiliert wird. Sie zeichnet sich durch ihre Einfachheit, Geschwindigkeit und Stabilität aus. Es gibt eine umfangreiche Anzahl von Frameworks wie auch eine grosse Community, welche durch Open-source ein gewisses Mitspracherecht an der Zukunft der Sprache hat. So können simple Kommandozeilen-Scripts wie auch umfangreiche Webapplikationen erstellt werden.  
	\paragraph{Java}
	Java ist eine objektorientierte Programmiersprache, mit welcher sich plattformunabhängige Anwendungen entwickeln lassen können. Die Sprache hat sich gut durchgesetzt und ist weit verbreitet. Die Haupteigenschaften der Programmiersprache sind Sicherheit, Robustheit und Einfachheit. Java hat eine Vielzahl an Frameworks und eine grosse Community.
	\\ \\
	Beide Sprache bieten eine grosse Community und ein umfangreiches Toolset, mit dem die Anforderungen an das System erfüllt werden können. Mithilfe einer Nutzwertanalyse wird nun entschieden, mit welchen Varianten die Sprachen in der Lösungswahl berücksichtigt werden.
	
	\subsubsection{Frontend}
	\paragraph{Vue.js}
	Ist ein clientseitiges JavaScript-Webframework zum Erstellen von Single-Page-Webanwendungen nach dem MVVM-Muster. Es kann allerdings auch in Multipage Webseiten für einzelne Abschnitte verwendet werden. Serverseitiges Rendern ist auch unterstützt. \\
	Vue.js ist ein "relativ" junges Framework, hat aber trotzdem eine wachsende Community und wird immer mehr verwendet. (zum Beispiel von GitLab).
	\paragraph{React}
	Gibt es bereits etwas länger als Vue.js und bietet ähnliche Funktionalitäten als Frontend Framework. Es wurde damals von Facebook ins Leben gerufen und wird auch noch heute verwendet. Daher geniesst React eine grosse Community und ist weit verbreitet.
	\\ \\
	Beide Sprachen bieten ähnliche Funktionalitäten und Syntax. Vue.js gilt als flexiblere Sprache und daher für Neulinge besser zum Einsteigen. React.js geniesst eine grosse Community mit Facebook als Mitentwickler hinter der Sprache (ist aber wie Vue.js OpenSource). Auch hier soll durch eine Nutzwertanalyse entschieden welche Variante besser zum Projekt passt.
	
	\subsection{Black Box}
	
	\subsubsection{Frontend}
	
	\begin{figure}[!hbt]
		\centering
		\includegraphics[width=\columnwidth]{01_Grafiken/black-box_Frotend.png}
		\caption{Black Box Frontend}
		\label{fig:bb_frontend}
	\end{figure}
	
	\subsubsection{Backend}

	\begin{figure}[!hbt]
		\centering
		\includegraphics[width=\columnwidth]{01_Grafiken/black-box_Backend.png}
		\caption{Black Box Backend}
		\label{fig:bb_backend}
	\end{figure}

	\newpage

	\subsection{Lösungen}
	
	\begin{figure}[!hbt]
		\centering
		\includegraphics[width=\columnwidth]{01_Grafiken/architektur_vorschlag.png}
		\caption{Architektur}
		\label{fig:architektur}
	\end{figure}

	Es wurde eine Variante erstellt, welche durch das Vorgängerprojekt, der LED-RGB-Matrix, den Rahmenbedingungen und dem Architekturvorschlag (Abbildung: \ref{fig:architektur}) definiert wurde.
	\\ \\
	Die LED-RGB-Matrix wurde als unabhängiger Service programmiert und kann daher ohne Anpassung in das laufende Projekt integriert werden. Es handelt sich um eine Rest-API, an welche Bilder mit gewünschten Transformationsparamtern gesendet und dargestellt werden können.
	\\ \\
	Das Tic-Tac-Toe-Game (Spiellogik + Datenbank) wird das Backend bilden. Dies beinhaltet folgende Aufgaben:
	\begin{itemize}
		\item Ist für die gesamte Spiellogik zuständig
		\begin{itemize}
			\item Spielverlauf, Spielerinformation und Spielstand mit der Datenbank verwalten
			\item Assets verwalten
			\item Reagieren auf Zustandsänderungen des Frontends
		\end{itemize}
		\item Kommunizieren mit dem Frontend (über MQTT) und rendern der View
		\item Aufbereiten und senden des Spielfeldes an die LED-RGB-Matrix (REST)
	\end{itemize}
	
	Bei der Datenbank handelt es sich um MYSQL, was vom Kunden so gewünscht wurde.\\
	Das Frontend wird über MQTT mit dem Backend kommunizieren und die interaktive Spieloberfläche sein, über welche dann Tic-Tac-Toe gespielt werden kann.
    Diese reagiert dynamisch auf die Eingaben des Benutzers und passt die View an.\\
	\subsubsection{Varianten}
	Es wurde eine Lösung erarbeitet, welche mit verschiedenen Programmiersprachen umgesetzt werden kann.
	\paragraph{Frontend}
	Das Frontend kann jeweils mit Vue.js oder React.js umgesetzt werden.
	
	\paragraph{Backend}
	Im Backend besteht die Möglichkeit dieses mit Java oder PHP zum implementieren.

	\newpage
		
	\section{Lösungswahl}
	Die Lösungswahl wird aufgrund der Zielerreichung, Nutzwertanalyse und den weiteren Risiken getroffen.
	\subsection{Zielerreichung}
	In der Zielerreichung wird analysiert, ob die Lösungsvarianten die geforderten Systemziele erfüllen können. So werden ungeeignete Lösungen bereits in einer frühen Phase aus der Studie herausgefiltert.
	\\
	Es wurden jeweils nur die umgebungsrelevanten Systemziele berücksichtigt.
	
	\subsubsection{Frontend}
	
	\begin{table}[h]
		\centering
		\rowcolors{1}{gray!30}{gray!10}
		\begin{tabular}{p{0.25\textwidth}|p{0.2\textwidth}|p{0.22\textwidth}|p{0.22\textwidth}}
			\textbf{Systemziel-Nr.} & \textbf{Gewicht (M,S,K)} & \textbf{Vue.js} & \textbf{React.js} \\ \hline
			1 & M & \cellcolor{green} Ja & \cellcolor{green} Ja \\ \hline
			2 & M & \cellcolor{green} Ja & \cellcolor{green} Ja \\ \hline
			5 & M & \cellcolor{green} Ja & \cellcolor{green} Ja \\ \hline
			7 & M & \cellcolor{green} Ja & \cellcolor{green} Ja \\ \hline
			8 & M & \cellcolor{green} Ja & \cellcolor{green} Ja \\ \hline
		\end{tabular}
		\caption{Zielerreichung des Frontend}
		\label{fig:zielerreichung_frontend}
	\end{table}

	
	\subsubsection{Backend}
	
	\begin{table}[h]
		\centering
		\rowcolors{1}{gray!30}{gray!10}
		\begin{tabular}{p{0.25\textwidth}|p{0.2\textwidth}|p{0.22\textwidth}|p{0.22\textwidth}}
			\textbf{Systemziel-Nr.} & \textbf{Gewicht (M,S,K)} & \textbf{PHP} & \textbf{Java} \\ \hline
			1 & M & \cellcolor{green} Ja & \cellcolor{green} Ja \\ \hline
			2 & M & \cellcolor{green} Ja & \cellcolor{green} Ja \\ \hline
			3 & M & \cellcolor{green} Ja & \cellcolor{green} Ja \\ \hline
			4 & S & \cellcolor{green} Ja & \cellcolor{green} Ja \\ \hline
			6 & M & \cellcolor{green} Ja & \cellcolor{green} Ja \\ \hline
			7 & M & \cellcolor{green} Ja & \cellcolor{green} Ja \\ \hline
			8 & M & \cellcolor{green} Ja & \cellcolor{green} Ja \\ \hline
		\end{tabular}
		\caption{Zielerreichung des Backend}
		\label{fig:zielerreichung_backend}
	\end{table}
	
	\newpage

	\begin{landscape}
	\subsection{Nutzwertanalyse}
	Die Gewichtung wird in Form einer Zahl zwischen 1-10 definiert, wobei die 1 die tiefste und 10 die höchste Gewichtung ist.
	
	 \subsubsection{Frontend}
	
	\begin{table}[h]
		\centering
		\begin{tabular}{p{0.4\textwidth}|p{0.25\textwidth}|p{0.2\textwidth} | p{0.2\textwidth} | p{0.2\textwidth} | p{0.2\textwidth}}
			\multirow{2}{*}{\textbf{Kriterien}} & \multirow{2}{*}{\textbf{Gewichtung}} & \multicolumn{2}{c}{\textbf{Variante 1: Vue.js}} & \multicolumn{2}{c}{\textbf{Variante 2: React.js}} \\
			& & \textbf{Punkte} & \textbf{Bewertung} & \textbf{Punkte} & \textbf{Bewertung}  \\ \hline
			Vorhandenes Know-How & 6 & 2 & 12 & 2 & 12  \\ \hline
			Geringer Entwicklungsaufwand & 7 & 8 & 56 & 6 & 42 \\ \hline
			Community / Dokumentation & 9 & 9 & 81 & 8 & 72 \\ \hline
			Verfügbare Libraries & 5 & 10 & 50 & 10 & 50 \\ \hline
			Zukunftssicher / Vortlaufende Entwicklung & 8 & 10 & 80 & 10 & 80 \\ \hline
			Erweiterbarkeit & 5 & 9 & 45 & 9 & 45 \\ \hline
			\textbf{Nutzwert} & & & \cellcolor{green} 324 & & \cellcolor{orange} 301 \\ \hline
		\end{tabular}
		\caption{Nutzwerkanalyse Frontend}
		\label{fig:nutzwerkanalyse_frontend}
	\end{table}

	Unsere Recherche wurde hauptsächlich von der folgenden Seite bezogen: \href{https://www.codica.com/blog/react-vs-vue-2019/#react-vs-vue-comparison-summary}{https://codica.com}

	\end{landscape}

	\newpage
	
	\begin{landscape}
		\subsubsection{Backend}
		
		\begin{table}[h]
			\centering
			\begin{tabular}{p{0.4\textwidth}|p{0.25\textwidth}|p{0.2\textwidth} | p{0.2\textwidth} | p{0.2\textwidth} | p{0.2\textwidth}}
				\multirow{2}{*}{\textbf{Kriterien}} & \multirow{2}{*}{\textbf{Gewichtung}} & \multicolumn{2}{c}{\textbf{Variante 1: PHP}} & \multicolumn{2}{c}{\textbf{Variante 2: Java}} \\
				& & \textbf{Punkte} & \textbf{Bewertung} & \textbf{Punkte} & \textbf{Bewertung}  \\ \hline
				Vorhandenes Know-How & 6 & 7 & 42 & 8 & 48  \\ \hline
				Geringer Entwicklungsaufwand & 7 & 7 & 49 & 9 & 63 \\ \hline
				Community / Dokumentation & 9 & 9 & 81 & 8 & 72 \\ \hline
				Verfügbare Libraries & 5 & 10 & 50 & 10 & 50 \\ \hline
				Zukunftssicher / Fortlaufende Entwicklung & 8 & 10 & 80 & 10 & 80 \\ \hline
				Erweiterbarkeit & 5 & 9 & 45 & 9 & 45 \\ \hline
				\textbf{Nutzwert} & & & \cellcolor{orange} 347 & & \cellcolor{green} 358 \\ \hline
			\end{tabular}
			\caption{Nutzwerkanalyse Backend}
			\label{fig:nutzwerkanalyse_backend}
		\end{table}
	\end{landscape}

	\newpage
	
	\subsection{Weitere Risiken}
	
	\subsubsection{Frontend}
	
	\begin{table}[h]
		\centering
		\rowcolors{1}{gray!30}{gray!10}
		\begin{tabular}{p{0.24\textwidth}|p{0.34\textwidth}|p{0.34\textwidth}}
			\textbf{Beurteilungskriterien} & \textbf{Vue.js} & \textbf{React.js} \\ \hline
			Risiken & Durch das geringe Know-How kann es zu Verspätungen in der Entwicklung kommen & Durch das geringe Know-How kann es zu Verspätungen in der Entwicklung kommen \\ \hline
			Nachhaltigkeit & OpenSource & OpenSource
		\end{tabular}
		\caption{Weitere Risiken im Frontend}
		\label{fig:weitere-risiken_frontend}
	\end{table}

	\subsubsection{Backend}

	\begin{table}[h]
		\centering
		\rowcolors{1}{gray!30}{gray!10}
		\begin{tabular}{p{0.24\textwidth}|p{0.34\textwidth}|p{0.34\textwidth}}
			\textbf{Beurteilungskriterien} & \textbf{PHP} & \textbf{Java} \\ \hline
			Risiken & Durch das geringe Know-How gewisser Mitglieder, kann es zu Verspätungen in der Entwicklung kommen & Durch das geringe Know-How gewisser Mitglieder, kann es zu Verspätungen in der Entwicklung kommen \\ \hline
			Nachhaltigkeit & OpenSource & OpenSource und Proprietäre möglich
		\end{tabular}
		\caption{Weitere Risiken im Backend}
		\label{fig:weitere-risiken_backend}
	\end{table}

	\subsection{Empfehlung}
	Das Projektteam hat bereits Know-How in JavaScrit wie auch teilweise in PHP und Java.
	Da in den Nutzwertanalysen klar herausgeht, dass für dass Frontend \textbf{Vue.js} und im Backend \textbf{Java} verwendet werden sollte, wird dies auch von uns empfohlen.
	
	\newpage
	
	\section{Detailkonzept}
	
	\subsection{Mengengerüst}
	
%	Halle hier beschriebenen Komponenten sind Teil des LED-Boards oder laufe auf dem Respberry PI. Werden hier aber trotzdem in Matrix und Tic-Tac-Toe aufgeteilt.
	
	\subsubsection{LED-RGB-Matrix Hardware}
	
	\paragraph{Respberry PI 3B+ mit Extensionboard}
	\begin{itemize}
		\item Full size HDMI
		\item 4x USB 2.0 Ports
		\item Extended 40 pin GPIO header
		\item Micro SD port
		\item 5V/2.5A DC power input
		\item 16GB SD Card
		\item 1GB RAM
		\item Cortex A53 (ARMv8) 64 bit SoC @ 1.4 GHz
	\end{itemize}

	\href{https://www.arrow.com/en/products/max1000/arrow-development-tools }{Offizielle Angaben}

	\paragraph{MAXA1000 IoT Maker Board mit FPGA IO Modul}
	\begin{itemize}
		\item 1x Micro USB port
	\end{itemize}

	\href{https://www.arrow.com/en/products/max1000/arrow-development-tools }{Offizielle Angaben}
	
	\paragraph{Netzteil LVP-100-5}
	\begin{itemize}
		\item Input: 100-240V
		\item Output: +5V
	\end{itemize}
	
	\paragraph{LED-Matrix}
	\begin{itemize}
		\item 16x16 Digital LED's
	\end{itemize}
	
	\href{https://www.led-genial.de/DIGI-DOT-Panel-16x16-HD-mit-256-x-Digital-LEDs }{Offizielle Angaben}
	
	\subsubsection{LED-RGB-Matrix API}
	\paragraph{Schnittstellen}
	\begin{itemize}
		\item Bild als Datei hochladen
		\item Bild über eine URL hochladen
	\end{itemize}

	Zusätzlich haben die Schnittstellen weitere Parameter, welche gesetzt werden können, um die Bilder zu transformieren.
	
	\paragraph{Komponenten (Objekte, Services...)}
	\begin{itemize}
		\item LedMatrixController
		\item LedMatrixService
		\item ImageService
		\item FileService
		\item Gateway
	\end{itemize}
	
	\subsubsection{Tic-Tac-Toe GameServer}
	\paragraph{GameServer}
	\begin{itemize}
		\item LoginService
		\item GameService
		\item REST Schnittstellen
		\item MQTT Topics
	\end{itemize}
	
	\paragraph{Zusätzliche Services}
	\begin{itemize}
		\item MYSQL	
		\item Mosquitto (MQTT Server)
		\item Docker
	\end{itemize}
	
	\newpage
	
	\begin{landscape}
		\begin{table}[h]
\centering
\rowcolors{1}{gray!30}{gray!10}
\begin{tabular}{p{0.4\textwidth}|p{0.4\textwidth}|p{0.4\textwidth}}
	 & \textbf{Stärken}
	 \begin{itemize}
	 	\item Kompetenzen in der Programmierung allgemein
	 	\item Kompetenz beim Erstelle von API's
	 	\item Hohe Motivation der Mitarbeiter
	 \end{itemize}
 	 & \textbf{Schwächen}
 	 \begin{itemize}
 	 	\item Keine Erfahrungen mit MQTT oder Vue.js
 	 \end{itemize} \\ \hline
	 \textbf{Chancen}
	 \begin{itemize}
	 	\item Markterschliessung durch Differenzierung
	 \end{itemize} &
 	 \begin{itemize}
 	 	\item Bestehende Kompetenzen nutzen, um sich von der Konkurrenz abzuheben.
 	 	\item Motivation und Kompetenzen der Mitarbeiter nutzen, um eigenes GUI und Schnittstellenarten zu entwickeln.
 	 \end{itemize} &
  	 \begin{itemize}
  	 	\item Nötige Erfahrungen aneignen und vertiefen.
  	 	\item Die angewendeten Technologien nutzen, um eine neue Art von Lösung für eine bereits bestehende Idee zu entwickeln.
  	 \end{itemize} \\ \hline
	 \textbf{Risiken\textbackslash Gefahren}
	 \begin{itemize}
	 	\item Wenig Personal
	 	\item Es hat bereits etablierte Lösungen
	 \end{itemize} &
 	 \begin{itemize}
	  \item Einsatzbereitschaft der Mitarbeiter nutzen und motivieren, um das wenige Personal auszugleichen.
 	 \end{itemize}  & 
  	 \begin{itemize}
  	 	\item Weiterbildung der Mitarbeiter
  	 \end{itemize}
\end{tabular}
\caption{SWOT-Analyse}
\label{fig:swot-analyse}
\end{table}

	\end{landscape}


	\newpage
	
	\subsection{Logikschicht}
	
	\subsubsection{Produktkonzept}

	Im Vorläuferprojekt wurde bereits ein Programm entwickelt, mit welchem es möglich ist, eine LED-RGB-Matrix mit einer API zu auszustatten und über diese Bilder auf der Matrix anzeigen zu lassen.
	Das ausgearbeitete Konzept beinhaltet nun die Grundidee, Matrizen über ein Frontend ansteuerbar zu machen und auf diesen ein laufendes Tic-Tac-Toe-Spiel anzeigen zu lassen.
	\\
	Das Frontend wird dem User die Möglichkeiten geben, sich einzuloggen und Spiele zu erstellen, beizutreten und zu spielen. Die Kommunikation mit dem Server, welcher die komplette Businesslogik beinhalten wird, geschieht hauptsächlich über MQTT. Teile, wie z.B. das Login werden aufgrund der Sicherheit über REST gelöst.

	\begin{figure}[!hbt]
		\centering
		\includegraphics[width=\columnwidth]{01_Grafiken/architektur_konzept.png}
		\caption{Architektur}
		\label{fig:architektur}
	\end{figure}

	\newpage
	
	\subsubsection{Ablaufdiagramm}
	Im folgenden Ablaufdiagramm wird die Hauptlogik des TicTacToe Spiels als vereinfachten Codeablauf dargestellt. So wurde er geplant um die Struktur zu definieren und wie das Projekt umgesetzt werden soll.

\begin{figure}[!hbt]
	\centering
	\includegraphics[width=0.8\columnwidth]{02_Dokumente/03_Ablaufdiagramm/01_Grafiken/flussdiagramm_main.png}
	\caption{Flussdiagramm Main}
	\label{fig:flussdiagramm_main}
\end{figure}

\newpage

\begin{figure}[!hbt]
	\centering
	\includegraphics[width=0.8\columnwidth]{02_Dokumente/03_Ablaufdiagramm/01_Grafiken/flussdiagramm_login.png}
	\caption{Flussdiagramm Login}
	\label{fig:flussdiagramm_login}
\end{figure}

\newpage

\begin{figure}[!hbt]
	\centering
	\includegraphics[width=0.6\columnwidth]{02_Dokumente/03_Ablaufdiagramm/01_Grafiken/flussdiagramm_spiel-erstellen.png}
	\caption{Flussdiagramm Spiel erstellen}
	\label{fig:flussdiagramm_spiel-erstellen}
\end{figure}

\newpage

\begin{figure}[!hbt]
	\centering
	\includegraphics[width=0.3\columnwidth]{02_Dokumente/03_Ablaufdiagramm/01_Grafiken/flussdiagramm_eigenem-spiel-beitreten.png}
	\caption{Flussdiagramm Eigenem Spiel beitreten}
	\label{fig:flussdiagramm_eigenem-spiel-beitreten}
\end{figure}

\newpage

\begin{figure}[!hbt]
	\centering
	\includegraphics[width=0.9\columnwidth]{02_Dokumente/03_Ablaufdiagramm/01_Grafiken/flussdiagramm_spiellogik.png}
	\caption{Flussdiagramm Spiellogik}
	\label{fig:flussdiagramm_spiellogik}
\end{figure}

\newpage

\begin{figure}[!hbt]
	\centering
	\includegraphics[width=0.6\columnwidth]{02_Dokumente/03_Ablaufdiagramm/01_Grafiken/flussdiagramm_spiel-endet.png}
	\caption{Flussdiagramm Spiel endet}
	\label{fig:flussdiagramm_spiel-endet}
\end{figure}

	
	\newpage
	
	\subsection{Entwicklungssicht}
	
	\subsubsection{Verwendete Tools}
	\begin{table}[h]
\centering
\rowcolors{1}{gray!30}{gray!10}
\begin{tabular}{p{0.18\textwidth}|p{0.18\textwidth}|p{0.63\textwidth}}
	\textbf{Tool} & \textbf{Anwendungsgebiet} & \textbf{Beschreibung} \\ \hline
	Jira mit BigGantt & Projektmanagement & Jira ist eine Projekt Management Software von Atlassian. Bietet z.B. ein Kanban Board zum Issues verwalten und weitere Methoden um auch Agile zu arbeiten. BigGantt ist eine App (Erweiterung) für Jira um den ganzen Projektablauf zu illustrieren und zu verwalten.  \\ \hline
	Slack & Kommunikation u. Projektmanagement & Slack ist eine Business kommunikations Platform. Sie ermöglicht mehreren Teammitglieder zusammen zu chatten, telefonieren oder den Bildschirm zu teilen und kollaborativ zu arbeiten. \\ \hline
	Microsoft Teams &  Kommunikation & Business kommunikations Platform von Microsoft. Wird benutzt um mit dem Kunden im Kontakt zu bleiben. \\ \hline
	LaTeX & Dokumentation & LaTeX ist ein Satzsystem für die Erstellung von technischen und wissenschaftlichen Dokumentationen.  \\ \hline
	Git & Dokumentation u. Code & Git ist ein verteiltes version-control System. Jeweils die komplette Dokumentation in LaTeX und der Code sind mit Git versioniert.\\ \hline
\end{tabular}
\caption{Tools}
\label{fig:tools}
\end{table}

	
	\subsubsection{Entwicklungsumgebung und Tool-Chain}
	
	\paragraph{Maven Webprojekt}
	Maven ist ein Projekt-Build-Tool, das unter der Lizenz von Apache steht. Es gibt eine grosse Anzahl an Bibliotheken, welche im Maven-Repository verfügbar sind. Die im Projekt verwendeten Bibliotheken werden über Maven verwaltet.
	Folgende Bibliotheken werden im Projekt verwendet:
	
	\paragraph{Spring Framework}
	Die Kernfunktionen des Frameworks können von jeder Java-Anwendung genutzt werden; es gibt jedoch Erweiterungen beispielsweise für die Erstellung von Webanwendungen. Obwohl das Framework kein spezielles Programmiermodell vorschreibt, ist es in der Java-Community als Ergänzung oder sogar als Ersatz für das Enterprise JavaBeans (EJB)-Modell populär geworden. Das Spring-Framework ist quelloffen.
	Die Applikation wird mit dem Spring Framework in der Version 5.2 entwickelt und enthält die folgenden Erweiterungen:
	\begin{itemize}
		\item spring-boot-starter
		\item spring-boot-starter-web
		\item spring-boot-starter-tomcat
	\end{itemize}

	\paragraph{jUnit 5, Mockito und Rest-Assured}
	jUnit 5, Mockito und Rest-Assured sind Bibliotheken, welche von Entwicklern verwendet werden, um Unit-Tests in Java zu implementieren, die Programmiergeschwindigkeit zu beschleunigen und die Qualität des Codes zu erhöhen.
	
	\paragraph{Entwicklungsumgebung Intellij}
	IntelliJ IDEA ist ein integrated development environment (IDE), welches in Java geschrieben ist, um Software zu entwickeln. Sie wird von JetBrains (früher bekannt als IntelliJ) entwickelt und ist als Community-Edition sowie als proprietäre kommerzielle Edition erhältlich. Beide können von kommerziellen Entwicklern verwendet werden.
	Oliver Egloff und Leonardo Wiedemeier haben bereits gute Kenntnisse in der Anwendung dieser Software.
	
	\paragraph{Vue.js Framework}
	Ist ein clientseitiges JavaScript-Webframework zum Erstellen von Single-Page-Webanwendungen nach dem MVVM-Muster, es kann allerdings auch in Multipage Webseiten für einzelne Abschnitte verwendet werden. Serverseitiges Rendern ist auch unterstützt. \\
	Vue.js ist ein relativ junges Framework, hat aber trotzdem eine wachsende Community und wird immer mehr verwendet. (zum Beispiel von GitLab).
	
	\paragraph{Entwicklungsumgebung Webstorm}
	Webstorm IDE ist ein integrated development environment (IDE), welches für JavaScript Entwicklung von JetBrains geschrieben wurde. Ist nur als kommerzielle Edition erhätlich.
	
	\paragraph{Versionierung mit Git (Gitlab)}
	Der Source Coder der Applikation wird in Git abgelegt und versioniert. Es wird nach dem Feature Branch Workflow gearbeitet (\href{https://www.atlassian.com/git/tutorials/comparing-workflows/feature-branch-workflow}{Feature Branch Workflow}). Der Code in einem Feature Branch wird neben dem Entwickler mindestens von einer weiteren Person angeschaut und kontrolliert (Code Review), bevor er in den Master Branch zusammengeführt wird.
	
	\paragraph{Docker}
	Docker ist eine Virtualisierungssoftware um sogenannte Container bereitzustellen. Container sind voneinander isoliert und bündeln ihre eigene Software, Bibliotheken und Konfigurationsdateien, zum Beispiel für die Virtualisierung und Verwendung einer MYSQL Datenbank oder eines MQTT Servers.
	
\subsubsection{Schnittstellen}
	
\begin{table}[h]
	\centering
    \rowcolors{1}{gray!30}{gray!10}
	\begin{tabular}{p{0.20\textwidth}|p{0.15\textwidth}|p{0.63\textwidth}}
		\textbf{Systeme} & \textbf{Protokoll} & \textbf{Beschreibung} \\ \hline
		Clients u. Game Server & MQTT via Port 3888 & Das Erstellen und Abhandeln von laufenden Spielen läuft zwischen Client und Server über MQTT ab. Die Daten werden jeweils in der Topic broadcasted. Server verarbeitet alle Anfragen die er bekommt. Der Client kann durch eine requestId identifizieren welche ihm gehören. \\ \hline
		Clients u. Game Server & REST via Port 8080 & Via REST werden zusätzliche für das Frontend relevante Daten übertragen (z.B. Login, Spielerinformationen).  \\ \hline
		Game Server u. API & REST via Port 8080 & Der Server kann über die REST-Schnittstelle mit der LED-REG-Matrix kommunizieren und so laufend neue Spielfelder mitteilen. Die Spielfelder werden vom Server zusammengestellt und dann mit den gewünschten Transformationsparametern an die API verschickt. So kann die API das Bild nach den Angaben verarbeiten und anschliessend auf der Matrix darstellen.  \\ \hline
		API -> LED-RGB-Panel & Binary Code via Serial Port & Die aufbereiteten Daten werden von der API an das LED-RGB-Panel gesendet.
	\end{tabular}
	\caption{Schnittstellen}
	\label{fig:schnittstellen}
\end{table}

	\newpage

	\subsubsection{Datenstrukturen}
	Die folgende Figur zeigt die einzelnen Varianten an, wie der Server und das Frontend, über MQTT, miteinander kommunizieren.
	\begin{figure}[!hbt]
		\centering
		\includegraphics[width=\columnwidth]{01_Grafiken/datenfluss.png}
		\caption{Datenflussdiagramm}
		\label{fig:datenflussdiagramm}
	\end{figure}

\begin{figure}[!hbt]
\begin{lstlisting}[language=json,firstnumber=1]
{
    "gameId": 2,
    "name": "Pending Game",
    "state": "PENDING",
    "lastModified": "<Date lastModified>",
    "matrixId": [],
    "gameData": {
        "host": 5",
        "guest": 20,
        "moves": [],
        "winner": "<HOST oder GUEST. NONE fals unentschieden und null wenn es noch läuft>"
    },
    "playerData": [{
        "player": "HOST",
        "displayName": "Spieler Host"
        },
        {
        "player": "GUEST",
        "displayName": "Spieler Guest"
        }
    ]
}
\end{lstlisting}
\caption{Datenfluss 1: Aktives Spiel vom Server zum Client}
\label{fig:datenfluss-offen-spiel-server-client}
\end{figure}

\begin{figure}[!hbt]
	\begin{lstlisting}[language=json,firstnumber=1]
{   
    "gameId": 1,
    "name": "Example Game",
    "state": "ACTIVE",
    "lastModified": "<Date lastModified>",
    "matrixId": [1,5],
    "gameData": {
        "host": 5",
        "guest": 20,
        "moves": [{
            "count": 1
            "player": "HOST",
            "gridPosition": "MID_LEFT"
        }],
        "winner": "<HOST oder GUEST. NONE fals unentschieden und null wenn es noch läuft>"
    },
    "playerData": [
        {
        "player": "HOST",
        "displayName": "Spieler Host"
        },
        {
        "player": "GUEST",
        "displayName": "Spieler Guest"
        }
    ]
}
\end{lstlisting}
\caption{Datenfluss 2: Aktives Spiel vom Server zum Client}
\label{fig:datenfluss-aktives-spiel-server-client}
\end{figure}

\begin{figure}[!hbt]
	\begin{lstlisting}[language=json,firstnumber=1]
{
    "gameId": 1,
    "state": ACTIVE,
    "matrixId": [1,5],
    "gameData": {
        "host": 5",
        "guest": 20",
        "moves": [{
            "count": 1,
            "player": "HOST",
            "gridPosition": "MID_LEFT"
        }]
    }
}
\end{lstlisting}
\caption{Datenfluss 3: Aktives Spiel vom Client zum Server}
\label{fig:datenfluss-aktives-spiel-client-server}
\end{figure}


	\newpage	

	\subsection{Pflichtenheft}
	Das Pflichtenheft ist im Anhang unter dem Abschnitt «Pflichtenheft» ersichtlich.
	
	\newpage
	
	\section{Realisierung}
	
	\begin{figure}[!hbt]
		\centering
		\includegraphics[width=\columnwidth]{OEG_20200408_Big_Picture_Draft.png}
		\caption{Big Picture - Lösungsübersicht}
		\label{fig:bigPicture}
	\end{figure}
	
	\subsection{Frontend}
	Wie bereits erwähnt wurde das Frontend nach den zuvor erstellen Mockups in Vue.js erstellt. Innerhalb von Vue.js wurden die folgenden Libraries verwendet:
	\begin{itemize}
		\item \textbf{Bootstrap \& Bootstrap-Vue} für das Layout und Responsive Design der Webapplikation. Durch Bootstrap ist die Applikation Mobile Ready.
		\item \textbf{MQTT} für die Spielkommunikation mit dem Backend
		\item \textbf{AXIOS} für Rest Abfragen
		\item \textbf{VueX} als zentraler Store, um Benutzer und Game Daten zu temporär im Frontend zu speichern
		\item \textbf{VueRouter} um erfolgreich zu navigieren
	\end{itemize}

	\subsubsection{Aufbau}
	Dem Benutzer wird zentral eine Navigation angezeigt, über welche er die Unterseiten
	\begin{itemize}
		\item Spielübersicht
		\item Spielverlauf
		\item Benutzerprofil
		\item Logout
	\end{itemize}

	Startpunkt nach dem Login ist immer die Spielübersicht auf welcher Meine Spiele und Offene Spiele angezeigt werden.
	\begin{figure}[!hbt]
		\centering
		\includegraphics[width=\columnwidth]{Spieluebersicht.png}
		\caption{Spielübersicht}
		\label{fig:playView}
	\end{figure}
	
	\paragraph{Spielverlauf}
	Das zweite Icon in der Navigation leitet den Benutzer zum Spielverlauf. Dort kann der Benutzer seinen vergangen Spiele sehen und erkennt auf einen Blick, ob er ein Spiel gewonnen, verloren oder ob das Spiel unentschieden geendet hat.
	\begin{figure}[!hbt]
		\centering
		\includegraphics[width=\columnwidth]{Game-History.png}
		\caption{Spielverlauf}
			\label{fig:gameHistory}
	\end{figure}

	\paragraph{Benutzerprofil}
	Um persönliche Informationen wie Name, Spielername, Passwort, etc. anzupassen oder etwa die Kreuze oder Kreise durch ein persönliches Bild auszutauschen, kann der Benutzer unter seinem Benutzerprofil die entsprechenden Anpassungen durchführen. Die Benutzerdaten werden nicht wie die Gamedaten über MQTT, sondern aufgrund der Datensicherheit über einen Rest Enpoint übertragen.

	\newpage

	\begin{figure}[!hbt]
		\centering
		\includegraphics[width=\columnwidth]{Benutzerprofil-1.png}
		\caption{Benutzerinformationen anpassen}
		\label{fig:userProfile}
	\end{figure}

\begin{figure}[!hbt]
	\centering
	\includegraphics[width=\columnwidth]{Benutzerprofil-1.png}
	\caption{Spielerzeichen anpassen}
	\label{fig:userProfileTileImage}
\end{figure}

	\paragraph{Registrieren, Login \& Logout}
	Zu Beginn muss sich ein Benutzer über eine Rest Schnittstelle beim Server registrieren. Nach der Registration kann er sich anmelden, wobei ein HttpOnly Session Cookie im Browser gespeichert wird. So wird eine bestmöglichste UserExpirience gewährleistet, da der Benutzer innerhalb der Session so viele Fenster mit dem Tic-Tac-Toe Spiel öffnen kann, wie er möchte. Um sich auszuloggen kann der Benutzer das Icon ganz rechts in der Navigation anwählen.
	
	\newpage
	
	\begin{figure}[!hbt]
		\centering
		\includegraphics[width=\columnwidth]{register.png}
		\caption{Registration}
		\label{fig:register}
	\end{figure}

	\begin{figure}[!hbt]
		\centering
		\includegraphics[width=\columnwidth]{login.png}
		\caption{Login}
		\label{fig:login}
	\end{figure}
	
	\paragraph{Spiel}
	Wird von einer Liste (Meine Spiele, Offene Spiele oder Spielverlauf) ein Spiel geöffnet, so wird das Spiel aus dem Speicher geladen und in einer View dargstellt. Die View enthält sämtliche Business Logik um mit dem Server über Mqtt zu kommunizieren, um die entsprechenden Spielständ abzugleichen und anzuzeigen. Folgende Elemente sind innerhalb dieser View verfügbar:
	\begin{itemize}
		\item Name des Spiels
		\item Spieleranzeige
		\item Historie aller Züge in diesem Spiel. Der Spieler kann so das gesamte Spiel nochmals erleben
		\item Spielbrett, auf welchem der Spieler seine Züge ausführen kann
	\end{itemize}

	Sollte der Benutzer ein personalisiertes Zeichen besitzen, so wird dieses anstelle des X oder O angezeigt.
	
	\begin{figure}[!hbt]
		\centering
		\includegraphics[width=\columnwidth]{Gameboard-Kein-Gewinner.png}
		\caption{Spielanzeig mit keinem Gewinner}
		\label{fig:noWinner}
	\end{figure}

	\begin{figure}[!hbt]
		\centering
		\includegraphics[width=\columnwidth]{Gameboard-Mit-Gewinner.png}
		\caption{Spielanzeig bei dem ein Gewinner  festgestellt wurde}
		\label{fig:withWinner}
	\end{figure}
	
	\begin{figure}[!hbt]
		\centering
		\includegraphics[width=\columnwidth]{Gameboard-mit-bildern.png}
		\caption{Spielanzeig bei dem personalisierte Zeichen dargestellt werden}
		\label{fig:withPicture}
	\end{figure}

	\newpage
		
	\subsection{Klassendiagramm}
	Im folgenden Klassendiagramm wurden die von uns erstellten und für die eigentliche Tic-Tac-Toe-Logik des Backends wichtigen Klassen beschrieben. Klassen, Interfaces etc. des Springframeworks oder von JPA wurden nicht beschrieben.
	
	Das Klassendiagramm ist dem Anhang beigelegt.
	
	\newpage
	
	\subsection{ERM}
	
	\begin{figure}[!hbt]
		\centering
		\includegraphics[width=\columnwidth]{erm.png}
		\caption{Entity-relationship model}
		\label{fig:erm}
	\end{figure}
	
	\newpage
	
	\subsection{Tests}
	\subsubsection{Funktionale Tests}
	Die hier aufgelisteten Funktionalen Tests wurden mit dem Anforderungskatalog als Vorlage erstellt und decken alle Anforderungen ab.
	
	\begin{table}[h]
		\centering
		\rowcolors{1}{orange!30}{orange!10}
		\begin{tabular}{p{0.25\textwidth}|p{0.74\textwidth}}
			\cellcolor{orange!80} \textbf{Testfall Nr:} & \cellcolor{orange!80} \textbf{001} \\ \hline
			\cellcolor{orange!80} \textbf{Testaufgabe:} & Raspberry PI starten und Frontend aufrufen \\ \hline
			\cellcolor{orange!80} \textbf{Erwartetes Ergebnis:} & Das Frontend ist erreichbar und die Loginmaske wird angezeigt \\ \hline
			\cellcolor{orange!80} \textbf{Effektives Ergebnis:} & Das Frontend wird mit der Loginmaske korrekt dargestellt \\ \hline
			\cellcolor{orange!80} \textbf{Resultat:} & \textcolor{olive}{Erfüllt} \\ \hline
		\end{tabular}
		\caption{Testfall Nr 001}
		\label{fig:testfall_br_001}
	\end{table}

	\begin{table}[h]
		\centering
		\rowcolors{1}{orange!30}{orange!10}
		\begin{tabular}{p{0.25\textwidth}|p{0.74\textwidth}}
			\cellcolor{orange!80} \textbf{Testfall Nr:} & \cellcolor{orange!80} \textbf{002} \\ \hline
			\cellcolor{orange!80} \textbf{Testaufgabe:} & Registration eines neuen Users \\ \hline
			\cellcolor{orange!80} \textbf{Erwartetes Ergebnis:} & Es kann ein neuer Account erstellt und sich mit diesem eingeloggt werden \\ \hline
			\cellcolor{orange!80} \textbf{Effektives Ergebnis:} & Durch Befolgen des Registrierungsformulars und anschliessender Eingabe des Namens und Passwortes kann man sich erfolgreich einloggen \\ \hline
			\cellcolor{orange!80} \textbf{Resultat:} & \textcolor{olive}{Erfüllt} \\ \hline
		\end{tabular}
		\caption{Testfall Nr 002}
		\label{fig:testfall_br_002}
	\end{table}

	\begin{table}[h]
		\centering
		\rowcolors{1}{orange!30}{orange!10}
		\begin{tabular}{p{0.25\textwidth}|p{0.74\textwidth}}
			\cellcolor{orange!80} \textbf{Testfall Nr:} & \cellcolor{orange!80} \textbf{003} \\ \hline
			\cellcolor{orange!80} \textbf{Testaufgabe:} & Korrekte Authentifizierung \\ \hline
			\cellcolor{orange!80} \textbf{Erwartetes Ergebnis:} & Als nicht eingeloggter User hat man keine Rechte auf gesicherten Routen. \\ \hline
			\cellcolor{orange!80} \textbf{Effektives Ergebnis:} & Als anonymer User der Webseite wird man immer auf die Loginseite umgeleitet. \\ \hline
			\cellcolor{orange!80} \textbf{Resultat:} & \textcolor{olive}{Erfüllt} \\ \hline
		\end{tabular}
		\caption{Testfall Nr 003}
		\label{fig:testfall_br_003}
	\end{table}

	\newpage

	\begin{table}[h]
		\centering
		\rowcolors{1}{orange!30}{orange!10}
		\begin{tabular}{p{0.25\textwidth}|p{0.74\textwidth}}
			\cellcolor{orange!80} \textbf{Testfall Nr:} & \cellcolor{orange!80} \textbf{004} \\ \hline
			\cellcolor{orange!80} \textbf{Testaufgabe:} & Ein- und Ausloggen \\ \hline
			\cellcolor{orange!80} \textbf{Erwartetes Ergebnis:} & Es kann sich mit einem bestehenden Account ein- und ausgeloggt werden \\ \hline
			\cellcolor{orange!80} \textbf{Effektives Ergebnis:} & Durch die Eingabe der korrekten Daten loggt man sich ein und wird auf die Übersichtsseite weitergeleitet. Beim Klicken auf den ``Ausloggen`` Button wird man ausgeloggt und automatisch auf die Loginseite weitergeleitet. \\ \hline
			\cellcolor{orange!80} \textbf{Resultat:} & \textcolor{olive}{Erfüllt} \\ \hline
		\end{tabular}
		\caption{Testfall Nr 004}
		\label{fig:testfall_br_004}
	\end{table}
	
	\begin{table}[h]
		\centering
		\rowcolors{1}{orange!30}{orange!10}
		\begin{tabular}{p{0.25\textwidth}|p{0.74\textwidth}}
			\cellcolor{orange!80} \textbf{Testfall Nr:} & \cellcolor{orange!80} \textbf{005} \\ \hline
			\cellcolor{orange!80} \textbf{Testaufgabe:} & Darstellung fremder laufender Spiele auf der Übersichtsseite \\ \hline
			\cellcolor{orange!80} \textbf{Erwartetes Ergebnis:} & Auf der Übersichtsseite werden laufende Spiele aktualisiert und dargestellt \\ \hline
			\cellcolor{orange!80} \textbf{Effektives Ergebnis:} & Als eingeloggter User werden auf der Übersichtsseite periodisch neue laufende Spiele angezeigt und wieder entfernt, sobald diese voll sind. \\ \hline
			\cellcolor{orange!80} \textbf{Resultat:} & \textcolor{olive}{Erfüllt} \\ \hline
		\end{tabular}
		\caption{Testfall Nr 005}
		\label{fig:testfall_br_005}
	\end{table}
	
	\begin{table}[h]
		\centering
		\rowcolors{1}{orange!30}{orange!10}
		\begin{tabular}{p{0.25\textwidth}|p{0.74\textwidth}}
			\cellcolor{orange!80} \textbf{Testfall Nr:} & \cellcolor{orange!80} \textbf{006} \\ \hline
			\cellcolor{orange!80} \textbf{Testaufgabe:} & Erstellung eines Spiels \\ \hline
			\cellcolor{orange!80} \textbf{Erwartetes Ergebnis:} & Von der Übersichtsseite aus kann ein neues Spiel erstellt werden. \\ \hline
			\cellcolor{orange!80} \textbf{Effektives Ergebnis:} & Als eingeloggter User kann von der Übersichtsseite aus über den Button das Formular gestartet werden. Nach erfolgreichem Ausfüllen wird der User in die Lobby des neuen Spiel weitergeleitet. \\ \hline
			\cellcolor{orange!80} \textbf{Resultat:} & \textcolor{olive}{Erfüllt} \\ \hline
		\end{tabular}
		\caption{Testfall Nr 006}
		\label{fig:testfall_br_006}
	\end{table}
	
	\newpage
	
	\begin{table}[h]
		\centering
		\rowcolors{1}{orange!30}{orange!10}
		\begin{tabular}{p{0.25\textwidth}|p{0.74\textwidth}}
			\cellcolor{orange!80} \textbf{Testfall Nr:} & \cellcolor{orange!80} \textbf{007} \\ \hline
			\cellcolor{orange!80} \textbf{Testaufgabe:} & Tic-Tac-Toe spielen \\ \hline
			\cellcolor{orange!80} \textbf{Erwartetes Ergebnis:} & Zwei Personen können ein Tic-Tac-Toe-Spiel durchspielen \\ \hline
			\cellcolor{orange!80} \textbf{Effektives Ergebnis:} & Zwei Personen im gleichen Spiel können abwechselnd ein komplettes Tic-Tac-Toe spiel beenden. Es kann der HOST, GUEST oder niemand gewinnen (Unentschieden). \\ \hline
			\cellcolor{orange!80} \textbf{Resultat:} & \textcolor{olive}{Erfüllt} \\ \hline
		\end{tabular}
		\caption{Testfall Nr 007}
		\label{fig:testfall_br_007}
	\end{table}
	
	\begin{table}[h]
		\centering
		\rowcolors{1}{orange!30}{orange!10}
		\begin{tabular}{p{0.25\textwidth}|p{0.74\textwidth}}
			\cellcolor{orange!80} \textbf{Testfall Nr:} & \cellcolor{orange!80} \textbf{008} \\ \hline
			\cellcolor{orange!80} \textbf{Testaufgabe:} & Laufendes Spiel verlassen \\ \hline
			\cellcolor{orange!80} \textbf{Erwartetes Ergebnis:} & Während einem laufenden Spiel kann man als GUEST oder HOST dieses verlassen und wieder beitreten. \\ \hline
			\cellcolor{orange!80} \textbf{Effektives Ergebnis:} & Ein Spielteilnehmer kann das laufende Spiel verlassen und landet auf der Übersichtsseite. Dort werden alle seine laufende Spiele angezeigt. Durch Klicken kann diesem wieder beigetreten werden. \\ \hline
			\cellcolor{orange!80} \textbf{Resultat:} & \textcolor{olive}{Erfüllt} \\ \hline
		\end{tabular}
		\caption{Testfall Nr 008}
		\label{fig:testfall_br_008}
	\end{table}
	
	\begin{table}[h]
		\centering
		\rowcolors{1}{orange!30}{orange!10}
		\begin{tabular}{p{0.25\textwidth}|p{0.74\textwidth}}
			\cellcolor{orange!80} \textbf{Testfall Nr:} & \cellcolor{orange!80} \textbf{009} \\ \hline
			\cellcolor{orange!80} \textbf{Testaufgabe:} & Die eigene Gamehistory einsehen \\ \hline
			\cellcolor{orange!80} \textbf{Erwartetes Ergebnis:} & Seine eigene Gamehistory einsehen und einzelne vergangene Spiele anschauen \\ \hline
			\cellcolor{orange!80} \textbf{Effektives Ergebnis:} & Als eingeloggter User ist es möglich beendete Spiele in der Gamehistory anzuschauen und diese Schritt für Schritt anzuschauen. \\ \hline
			\cellcolor{orange!80} \textbf{Resultat:} & \textcolor{olive}{Erfüllt} \\ \hline
		\end{tabular}
		\caption{Testfall Nr 009}
		\label{fig:testfall_br_009}
	\end{table}
	
	\newpage
	
\begin{table}[h]
    \centering
    \rowcolors{1}{orange!30}{orange!10}
    \begin{tabular}{p{0.25\textwidth}|p{0.74\textwidth}}
        \cellcolor{orange!80} \textbf{Testfall Nr:} & \cellcolor{orange!80} \textbf{010} \\ \hline
        \cellcolor{orange!80} \textbf{Testaufgabe:} & Das Spiel auf der LED Matrix anzeigen \\ \hline
        \cellcolor{orange!80} \textbf{Erwartetes Ergebnis:} & Während einem laufenden Spiel kann der Spielfortschritt auf einer LED-RGB-Matrix angezeigt werden, indem der HOST eine verfügbare Matrix auswählt. \\ \hline
        \cellcolor{orange!80} \textbf{Effektives Ergebnis:} & Spielstand wird auf der LED-RGB-Matrix angezeigt. \\ \hline
        \cellcolor{orange!80} \textbf{Resultat:} & \textcolor{olive}{Erfüllt} \\ \hline
    \end{tabular}
    \caption{Testfall Nr 010}
    \label{fig:testfall_br_010}
\end{table}
	
	\newpage
	
	\section{Wirtschaftlichkeit und Risikoanalyse}
	
	\subsection{Wirtschaftlichkeit}
	TODO mit oli und samuel
	

\subsection{Risikoanalyse}
	\subsubsection{Risiken}

\begin{table}[h]
		\centering
		\rowcolors{1}{gray!30}{gray!10}
		\begin{tabular}{p{0.02\textwidth}|p{0.45\textwidth}|p{0.45\textwidth}}
			\textbf{Id} & \textbf{Risiko Beschreibung} & \textbf{Möglicher Schaden} \\ \hline

            R1 & Kosteneinschätzungen werden überschritten & Projekt ist teurer als ursprünglich geplant. Projektbudget kann nicht eingehalten werden. \\ \hline

            R2 & Mitarbeiterausfall & Projektziele können nicht eingehalten werden. Die anderen Mitarbeiter müssen vermehrt Überzeit machen. \\ \hline

            R3 & Hardwareausfall & Nach Ausfall muss ausgetauscht / repariert werden, was zu Verschiebungen im Zeitplan führen kann. \\ \hline

            R4 & Projektplan wurde zu sportlich geschätzt. & Meilensteine/Projektziele können nicht eingehalten werden, was zu Verschiebungen im Zeitplan führen kann. \\ \hline

            R5 & Naturkatastrophe & Verlust von Hardware, Code, Räumlichkeiten. Ausfall von Mitarbeitern. Es kann vorkommen, dass der Projektplan nicht eingehalten werden oder es nicht zur Durchführung des Projektes kommen kann. \\ \hline

            R6 & Fortschrittsverlust durch falsche Anwendung von Hilfsmitteln (z.B. falsche Versionierung von Dokumenten und Code) & Es geht sehr viel Zeit verloren, um die verlorene Arbeit wieder aufzuholen. Projektziele und Projektplan könnten damit gefährdet sein. \\ \hline

            R7 & COVID-19 & Die aktuelle Pandemie kann noch unbekannte Risiken und Einschränkungen mit sich ziehen. \\
		\end{tabular}

        \caption{Risiken}
        \label{fig:risiken}
    \end{table}

	Naturkatastrophen sind in der Schweiz ein kleines Risiko. Sie können sich aber trotzdem, z.B. durch Überschwemmungen oder Stürme, stark auf den Projektverlauf auswirken.

	Den Coronavirus (COVID-19) haben wir mit als Risiko aufgenommen, da es die aktuelle Lage stark beeinflusst. Wir können nicht vorhersagen, welche Folgen der Coronavirus noch mit sich ziehen wird. Zum Beispiel, wie stark die Wirtschaft einbrechen und den Fortschritt des Projekts in jeder Hinsicht beeinflussen, oder jeder Beteiligte am Projekt Privat beeinflusst wird.

	\newpage

	\subsubsection{Massnahmen}
	\begin{table}[h]
		\centering
		\rowcolors{1}{gray!30}{gray!10}
		\begin{tabular}{p{0.03\textwidth}|p{0.3\textwidth}|p{0.2\textwidth}|p{0.1\textwidth}|p{0.2\textwidth}}
			\textbf{Nr.} & \textbf{Was?} & \textbf{Wer mit wem?} & \textbf{Wie viel?} & \textbf{Wann?} \\ \hline

            M1 & Regelmässige Kontrolle der Kostenfortschritte sowie die Einplanung von Reserven & Oliver Egloff, Leonardo Wiedemeier & & Immer zu Beginn einer neuen Phase \\ \hline

			M2 & Einfordern eines Ferienabwesenheitsantrages sowie die Planung aller Ressourcen. Zudem sollte ein regelmässiger Know-How-Transfer stattfinden. & Oliver Egloff, Leonardo Wiedemeier & & Bei Ferienabwesenheit \\ \hline

			M3 & Hardware angemessen behandeln und mit dem Kunden (ABBTS) im Falle eines Ausfalls sofort reagieren. & Ganzes Team & & Während des ganzen Projektes \\ \hline

			M4 & Einplanen von zeitlichen Reserven & Ganzes Team & 1x pro Woche & Zu Beginn der Planung und am Anfang einer neuen Phase \\ \hline

			M6 & Regelmässige Backups der Lösung und klare Definierungen, wie gearbeitet wird. & Samuel Salomon, Leonardo Wiedemeier & 1x pro Woche & Ab Beginn Projekt \\ \hline

			M7 & Über die Situation mit COVID-19 auf dem Laufenden bleiben und persönliche wie auch geschäftliche Entscheidungen frühzeitig angehen. & Alle am Projekt beteiligten Personen & Täglich & Täglich
		\end{tabular}
		\caption{Massnahmen}
		\label{fig:massnahmen}
	\end{table}
    \noindent
    Die meisten Massnahmen haben mit genauer Planung, Kommunikation und mit konzentriertem und genauem Arbeiten zu tun.
    Naturkatastrophen, Coronavirus und auch teils Hardwareausfall können uns treffen, dürfen aber nicht überraschen.
    Mit den hier aufgelisteten Massnahmen gehen wir gegen alle möglichen Risiken an und definieren, wie im Falle eins Auftretens reagiert und sich darauf vorbereitet wird.

	\newpage

	\subsubsection{Risikomatrix}
	\begin{figure}[!hbt]
		\centering
		\includegraphics[width=\columnwidth]{risikomatrix.png}
		\caption{Risikomatrix}
		\label{fig:risiko_matrix}
	\end{figure}
\noindent
	Die Risikomatrix zeigt auf, wie die von uns definierten und eingeschätzten Risiken (gelbe Kugeln) nach Anwendung der Massnahmen (Schwarze Pfeile) entschärft und neu eingestuft werden (blaue Kugeln).

	\newpage


\section{Abschluss}


\section{Abbildungsverzeichnis}
\listoffigures


\section{Tabellenverzeichnis}
\listoftables

\newpage

\section{Selbstständigkeitserklärung}
Hiermit erklären wir, dass wir die vorliegende Diplomarbeit selbstständig geschrieben und keine anderen als die angegebenen Hilfsmittel verwendet haben.
Sämtliche verwendeten Textausschnitte, Zitate oder Inhalte anderer Verfasser wurden ausdrücklich als solche gekennzeichnet.
\\
\\

\begin{tabularx}{\columnwidth}{l l l X}
    \textbf & \textbf{Datum, Ort} & \textbf{Visum} \\
    & & &   \\
    \textbf{Oliver Egloff} & ......................................................... & .........................................................  \\
    & & &   \\
    \textbf{Leonardo Wiedemeier} & ......................................................... & .........................................................  \\
    & & &   \\
    \textbf{Samuel Salomon} & ......................................................... & ......................................................... \\
\end{tabularx}

\newpage

\section{Anhang}
\newpage

\subsection{Dokumente}

\subsubsection{Pflichtenheft}
% commented out because of compiling speed
\includepdf[pages=-, landscape=false, pagecommand={}, scale=0.91]{../../02_Pflichtenheft/Pflichtenheft_OEG‐LWI‐SSA_20200626_1.0_Closed_43.pdf} % 05_Anhänge/Dokumente/01_Pflichtenheft/OEG‐LWI‐SSA_Pflichtenheft_0.1_Draft_4.pdf

\newpage

\subsubsection{Java Coding Guidelines}
% commented out because of compiling speed
\includepdf[pages=-, landscape=false, pagecommand={}, scale=0.91]{../../05_Guidelines/01_Backend Coding Guidelines/OEG‐LWI‐SSA_Backend_Coding_Guidelines_0.1_Draft_4.pdf}

\newpage

\subsubsection{Frontend Coding Guidelines}
% commented out because of compiling speed
\includepdf[pages=-, landscape=false, pagecommand={}, scale=0.91]{../../05_Guidelines/02_Frontend Coding Guidelines/OEG‐LWI‐SSA_Frontend_Coding_Guidelines_0.1_Draft_4.pdf}

\newpage

\subsubsection{Handbuch für Installation und Inbetriebnahme}
% commented out because of compiling speed
\includepdf[pages=-, landscape=false, pagecommand={}, scale=0.91]{../../07_Handbuecher/Installation und Inbetriebnahme.pdf}

\newpage

\subsubsection{Sitzungsprotokolle}
% commented out because of compiling speed
\includepdf[pages=-, landscape=true, pagecommand={}, scale=0.91]{../../03_Protokolle/Sitzungs_Protokolle.pdf}

\newpage

\subsubsection{Schriftliche Kommunikation mit Auftraggeber}
% commented out because of compiling speed
\includepdf[pages=-, landscape=false, pagecommand={}, scale=0.91]{../../03_Protokolle/Kommunikation mit Auftraggeber.pdf}

\newpage

\subsection{Präsentationen}
\subsubsection{Kick-Off Meeting vom 13.05.2020}
    % commented out because of compiling speed
     \includepdf[pages=-, landscape=true, pagecommand={}, scale=0.91]{../../06_Präsentationen/01_KickOff_Meeting/Gruppe_1_SA_KickOff_Meeting_VisualBoardGmbH.pdf}

\newpage

\subsection{Jira Export}
\subsubsection{Export}
% commented out because of compiling speed
%\includepdf[pages=-, landscape=false, pagecommand={}, scale=0.91]{../../08_Jira/JIRA_Export.pdf}
\end{document}

