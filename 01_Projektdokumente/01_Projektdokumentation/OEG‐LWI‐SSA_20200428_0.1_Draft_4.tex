%& -job-name=autogenerated_file_by_Latex
\documentclass[11pt, a4paper, twoside]{article}

\usepackage[table]{xcolor}
\usepackage{geometry} %Set page layouts
\usepackage{pdfpages} %Include pdf files
\usepackage{fancyhdr} %Header and Footer
\usepackage{graphicx} %Inlcude graphics
\usepackage{lastpage} %Reference the number of pages in your LATEX
\usepackage[german]{babel}
\usepackage{fontspec}
\usepackage{float}
\usepackage{hyperref}
\usepackage{tocloft}
\usepackage{tabularx}
\usepackage{showframe} %For debugging

\hypersetup{
	colorlinks=true,
	linkcolor=black,
	filecolor=magenta,
	urlcolor=cyan,
}

\geometry{top=0.75cm, bottom=0.8cm, right=2.5cm, left=2.5cm, headheight=35pt, includeheadfoot, portrait}
\setmainfont{Calibri}

%Path relative to the .tex file containing the \includegraphics command
\graphicspath{{01_Grafiken/}}

\tocloftpagestyle{fancy}

%Define variables for document
\newcommand{\shortAuthor}{OEG-LWI-SSA}
\newcommand{\dstate}{Draft} %Draft, Progress, Closed
\newcommand{\dversion}{0.1}
\newcommand{\dname}{\shortAuthor\_20200428\_\dversion\_\dstate\_\pageref{LastPage}}


\begin{document}


%Define header & footer for twoside documents
\fancyhead{} %Clear all header fields
\fancyhead[L]{\includegraphics{header_img_abb2.png}}
\fancyfoot{} %Clear all footer fields
\fancyfoot[RO, LE]{\small \thepage \textbar \pageref{LastPage}}
\fancyfoot[LO, RE]{\small \dname}
\renewcommand{\footrulewidth}{0.4pt}
\renewcommand{\headrulewidth}{0.4pt}
\pagestyle{fancy}

	
	\section*{Dokumentenmanagement}
	
	\begin{tabularx}{\columnwidth}{l X}
		\textbf{Erstellungsdatum:} & 28.04.2020 \\
		\textbf{Autotren:} & Samuel Salomon (SSA), Oliver Egloff (OEG), Leonardo Wiedemeier (LWI) \\
		\textbf{Dateiname:} &\dname
	\end{tabularx}
	
	
	\section*{Änderungsverzeichnis}
	\begin{tabularx}{\columnwidth}{l l l X}
		\textbf{Version} & \textbf{Datum} & \textbf{Autor} & \textbf{Beschreibung}\\
		0.1 & 28.04.2020 & SSA & Dokumentation und Gliederung erstellt
	\end{tabularx}
	
	\renewcommand{\cftsecleader}{\cftdotfill{\cftdotsep}}
	\tableofcontents
	
	\clearpage
	
	\section{Managment Summary}
	\section{Aufgabenstellung}
	\section{Projektmanagement}
	\section{Situationsanalyse}
	\section{Zielsetzung}
	\section{Lösungssuche}
	\section{Lösungswahl}
	\section{Detailkonzept}
	\section{Realisierung}
	\newpage
	\section{Wirtschaftlichkeit und Risikoanalyse}
	
	\subsection{Risikoanalyse}
	\subsubsection{Risiken}
	
	\begin{table}[h]
		\centering
		\rowcolors{1}{gray!30}{gray!10}
		\begin{tabular}{p{0.02\textwidth}|p{0.45\textwidth}|p{0.45\textwidth}}
			\textbf{Id} & \textbf{Risiko Beschreibung} & \textbf{Möglicher Schade} \\ \hline
			
			R1 & Kosteneinschätzungen werden überschritten & Projekt wird teurer als ursprünglich geplant. Projektziele können nicht eingehalten werden. \\ \hline
			
			R2 & Mitarbeiterausfall & Projektziele können nicht eingehalten werden. Die anderen Mitarbeiter müssen vermehrt Überzeit machen. \\ \hline
			
			R3 & Probleme mit Hosting / Hardware & Unstimmigkeiten oder Probleme vom Hosting Partner können zu Verschiebungen im Zeitplan oder Mehrkosten führen. \\ \hline
			
			R4 & Projektplan wurde zu sportlich geschätzt. & Verlust von Hardware, Code, Räumlichkeiten. Ausfall von Mitarbeitern. Es kann vorkommen das der Projektplan nicht eingehalten werden kann oder es zur nicht Durchführung des Projektes kommen. \\ \hline
			
			R5 & Naturkatastrophe & Verlust von Hardware, Code, Räumlichkeiten. Ausfall von Mitarbeitern. Es kann vorkommen das der Projektplan nicht eingehalten werden kann oder es zur nicht Durchführung des Projektes kommen. \\ \hline
			
			R6 & Fortschrittsverlust durch falsche Anwendung von Hilfsmitteln (z.B. falsche Versionierung von Dokumenten und Code) & Es geht sehr viel Zeit verloren, um die verlorene Arbeit wieder aufzuholen. Projektziele und Projektplan könnten damit gefährdet sein. \\ \hline
			
			R7 & COVID-19 & Die aktuelle Pandemie kann noch unbekannte Risiken und Einschränkungen mit sich ziehen. \\
		\end{tabular}

		\caption{Risiken}
		\label{fig:risiken}
	\end{table}

	\newpage

	\subsubsection{Massnahmen}
	\begin{table}[h]
		\centering
		\rowcolors{1}{gray!30}{gray!10}
		\begin{tabular}{p{0.03\textwidth}|p{0.3\textwidth}|p{0.2\textwidth}|p{0.1\textwidth}|p{0.2\textwidth}}
			\textbf{Nr.} & \textbf{Was?} & \textbf{Wer mit wem?} & \textbf{Wie viel?} & \textbf{Wann?} \\ \hline
			
			M1 & Regelmässige Kontrolle der Kostenfortschritte sowie die Einplanung von Reserven & Oliver Egloff & & Immer zu Beginn einer neuen Phase \\ \hline
			
			M2 & Einfordern eines Ferienabwesenheitsantrages sowie die Planung aller Ressourcen. Zudem sollte ein regelmässiger Know-How Transfer stattfinden. & Oliver Egloff, Leonardo Wiedemeier & & Bei Ferienabwesenheit \\ \hline
			
			M3 & Mit dem Hosting in Kontakt bleiben und dafür sorgen das die eingesetzten Tools optimal laufen. & & & Ab der Realisierungsphase \\ \hline
			
			M4 & Einplanen von zeitlichen Reserven & Samuel Salomon, Oliver Egloff, Leonardo Wiedemeier & & Zu Beginn der Planung und am Anfang einer neuen Phase \\ \hline
			
			M6 & Regelmässige Backups der Lösung und klare Definierungen wie gearbeitet wird. & Samuel Salomon, Leonardo Wiedemeier & 1x pro Woche & 1x pro Woche \\ \hline
			
			M7 & Über die Situation mit COVID-19 auf dem Laufenden bleiben und persönliche wie auch Geschäftliche Entscheidungen Frühzeitig angehen. & Alle am Projekt beteiligten Personen & Täglich & Täglich \\
		\end{tabular}
		\caption{Massnahmen}
		\label{fig:massnahmen}
	\end{table}

	\newpage

	\subsubsection{Risikomatrix}
	\begin{figure}[!hbt]
		\centering
		\includegraphics[width=\columnwidth]{risikomatrix.png}
		\caption{Risikomatrix}
		\label{fig:risiko_matrix}
	\end{figure}

	\newpage
	
	\section{Abschluss}
	\renewcommand\listfigurename{}
	\section{Abbildungsverzeichnis}
	\listoffigures
	
	\renewcommand\listoftables{}
	\section{Tabellenverzeichnis}
	\listoftables
	
	
	\section{Selbstständigkeitserklärung}
	
	\section{Anhang}
	
	
	
\end{document}

