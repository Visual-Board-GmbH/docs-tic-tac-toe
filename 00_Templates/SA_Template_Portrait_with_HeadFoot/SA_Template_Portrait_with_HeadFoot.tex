\documentclass[11pt, a4paper, twoside]{article}

\usepackage[table]{xcolor}
\usepackage{geometry} %Set page layouts
\usepackage{pdfpages} %Include pdf files
\usepackage{fancyhdr} %Header and Footer
\usepackage{graphicx} %Inlcude graphics
\usepackage{lastpage} %Reference the number of pages in your LATEX
\usepackage[german]{babel}
\usepackage{fontspec}
\usepackage{float}
\usepackage{hyperref}
\usepackage{tocloft}
%\usepackage{showframe} %For debugging

\hypersetup{
	colorlinks=true,
	linkcolor=black,
	filecolor=magenta,
	urlcolor=cyan,
}



\geometry{top=0.75cm, bottom=0.8cm, right=2.5cm, left=2.5cm, headheight=35pt, includeheadfoot, portrait}
%\setmainfont{Calibri}

%Path relative to the .tex file containing the \includegraphics command
\graphicspath{ {01_Grafiken/}}

%Set the pagestyle of content tables from plain to fancy
\tocloftpagestyle{fancy}

%Define variables for document
\newcommand{\shortAuthor}{OEG-LWI-SSA}
\newcommand{\dstate}{Draft} %Draft, Progress, Closed
\newcommand{\dversion}{0.1}
\newcommand{\dname}{\shortAuthor\_YYYYMMDD\_\dversion\_\dstate\_\pageref{LastPage}}


\begin{document}
	%Define header & footer for twoside documents
	\fancyhead{} %Clear all header fields
	\fancyhead[L]{\includegraphics{header_img_abb2.png}}
	\fancyfoot{} %Clear all footer fields
	\fancyfoot[RO, LE]{\small \thepage \textbar \pageref{LastPage}}
	\fancyfoot[LO, RE]{\small \dname}
	\renewcommand{\footrulewidth}{0.4pt}
	\renewcommand{\headrulewidth}{0.4pt}

	%Define header & footer for oneside documents
	%\fancyhead{} %Clear all header fields
	%\fancyhead[L]{\includegraphics{01_Grafiken/header_img_abb2.png}}
	%\fancyfoot{}
	%\fancyfoot[R]{\small \thepage \textbar \pageref{LastPage}}
	%\fancyfoot[L]{\dname}
	%\renewcommand{\footrulewidth}{0.4pt}
	%\renewcommand{\headrulewidth}{0.4pt}
	\pagestyle{fancy}
	
	\renewcommand{\cftsecleader}{\cftdotfill{\cftdotsep}}
	\tableofcontents
	\clearpage
	
	\section{Section}
	\subsection{Subsection}
	\subsubsection{SubsubSection}
	\paragraph{Paragraph}
	\subparagraph{Subparagraph}
	
	% Table example
	\begin{table}[h]
		\centering
		\rowcolors{1}{gray!30}{gray!10}
		\begin{tabular}{p{0.02\textwidth}|p{0.45\textwidth}|p{0.45\textwidth}}
			\textbf{Id} & \textbf{Risiko Beschreibung} & \textbf{Möglicher Schade} \\ \hline
			test & test & test \\
		\end{tabular}
		
		\caption{Caption}
		\label{fig:figure}
	\end{table}
	
	\clearpage
	\listoffigures
	
	\clearpage
	\listoftables
\end{document}
